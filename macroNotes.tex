%\documentclass[twoside,openright]{report}
\documentclass{book}

\usepackage{amsmath,empheq}

\usepackage{graphicx}

\usepackage{enumitem}

\DeclarePairedDelimiter\abs{\lvert}{\rvert}%
\DeclarePairedDelimiter\norm{\lVert}{\rVert}%

% Swap the definition of \abs* and \norm*, so that \abs
% and \norm resizes the size of the brackets, and the 
% starred version does not.
\makeatletter
\let\oldabs\abs
\def\abs{\@ifstar{\oldabs}{\oldabs*}}
%
\let\oldnorm\norm
\def\norm{\@ifstar{\oldnorm}{\oldnorm*}}
\makeatother

\graphicspath{ {images/} }
 
\begin{document}

\begin{titlepage}

\newcommand{\HRule}{\rule{\linewidth}{0.5mm}} % Defines a new command for the horizontal lines, change thickness here

\center % Center everything on the page

\text{\LARGE  }\\[2cm] % Add some space

\HRule \\[0.6cm]
{ \huge \bfseries M200 Macroeconomics}\\[0.4cm] % Title of your document
\HRule \\[1.5cm]
 
\text{\LARGE Fran\c{c}ois Stiennon}\\[1.5cm] % Name of your university/college
\text{\LARGE University of Cambridge}\\[1.5cm] % Name of your university/college

{\large \today}\\[3cm] % Date, change the \today to a set date if you want to be precise


\vfill % Fill the rest of the page with whitespace

\end{titlepage}

\tableofcontents{}

\chapter{The Solow Model}

\section{Model Assumptions}

\begin{enumerate}
	\item Time is continuous. 
	\item Demographics. Initially there are $L(0)$ people alive and population grows at rate $n$:

	\[
		L(t) = L(0)e^{nt} 
	\]

	Normalising $L(0)$ = 1,
	\[
		L(t) = e^{nt}
	\]

	We observe that 
	\[
		\frac{\dot{L}(t)}{L(t)} = \frac{ne^{nt}}{e^{nt}} = n
	\]
	where
	\[
		\dot{L}(t) = \frac{dL(t)}{dt}
	\]

	\item Two types of agents: households and firms. Households who own the production factors ($K(t)$ and $L(t)$), and save a constant fraction $s$ of their income. The savings rate $s$ is exogenous and strictly positive ($s>0$).

	\item The labour supply in inelastic. There is perfect competition in the good and factor markets. The price of the final good is the numeraire.
	
	\item There is no government ($G = 0$) and the economy is closed ($NX = 0$): 
	\[
		Y = C + I
	\]
	
	\item The capital stock $K(t)$ depreciates by a constant fraction $\delta > 0$ in every period. We therefore have: 
	\[
		\dot{K}(t) = I(t) - \delta K(t)
	\]
	As the economy is closed, investment equals saving $I(t) = S(t)$. Savings are a constant fraction $s$ of income: 
	\[
		S(t) = sY(t)
	\]
	
	The remaining income constitutes consumption: 
	\[
		C(t) = (1-s)Y(t)
	\]
	
	We can plug in $I(t) = S(t) = sY(t)$: 
	\[
		\dot{K}(t) = sY(t) - \delta K(t)
	\]

\end{enumerate}

\section{Production Technology} 

There are 3 ways to include technological (TP) progress in the production function: 

\begin{enumerate}
	\item Hicks-Neutral TP: 
		\[
			F(K(t), L(t)) = \textbf{B(t)}F(K(t), L(t))
		\]
	\item Labour augmenting TP:
		\[
			F(K(t), L(t)) = F(K(t), \textbf{A(t)}L(t))
		\]
	\item Capital augmenting TP: 
		\[
			F(K(t), L(t)) = F(\textbf{C(t)}K(t), L(t))
		\]
\end{enumerate}
For Cobb-Douglas technology, the three forms are equivalent: 
	\begin{equation*} 
	\begin{split}
			F(K(t), L(t)) & = B(t)K(t)^\alpha L(t)^{1-\alpha} = K(t)^\alpha {(A(t)L(t))}^{1-\alpha} = {(C(t)K(t))}^\alpha {L(t)}^{1-\alpha} \\
 			& = B(t)K(t)^\alpha L(t)^{1-\alpha} = A(t)^{1-\alpha}K(t)^\alpha {L(t)}^{1-\alpha} = C(t)^\alpha{K(t)}^\alpha {L(t)}^{1-\alpha} 
	\end{split}
	\end{equation*}

where 
	\[
		B(t) = A(t)^{1-\alpha} = C(t)^\alpha
	\]
	\\
We will assume Labour augmenting TP, with a production given by 
	\[
		Y(t) = F[K(t), L(t)] = F[K(t), A(t)L(t)]
	\]
where
\begin{enumerate}
	\item $A(t)$ corresponds to technological progress.
	\item $A(t)L(t)$ is called "effective labour". 
	\item Technology $A(t)$ grows at rate $g$: 
		\[
			\frac{\dot{A}(t)}{A(t)} = g \iff A(t) = A(0)e^{gt}
		\]	
\end{enumerate}
Note: Technology must be labour augmenting to ensure the existence of a Balanced Growth Path.\\
\\
Assumptions about the production function: 

\begin{enumerate}
	\item $F(.)$ exhibits positive and diminishing returns in both factors: 	
		\begin{align*}
			\frac{\partial F}{\partial K} &>0 &  \frac{\partial ^2F}{\partial K^2} &<0\\
			\\
			\frac{\partial F}{\partial L} &>0 & \frac{\partial ^2F}{\partial L^2} &<0
		\end{align*}
	
	\item $F(.)$ exhibits constant returns to scale: 
		\[
			F(\lambda K(t), \lambda A(t)L(t)) = \lambda F(K(t), A(t)L(t))
		\]
	\item It satisfies the \textit{inada} conditions:
		\[
			\lim_{K \to 0}{\frac{\partial F}{\partial K}} = \lim_{L \to 0}{\frac{\partial F}{\partial L}} = \infty
		\]
		\\
		\[
			\lim_{K \to \infty}{\frac{\partial F}{\partial K}} = \lim_{L \to \infty}{\frac{\partial F}{\partial L}} = 0
		\]
		\\
	
\end{enumerate}
The Cobb Douglas production function:
	\[
		F(K(t), L(t)) = K(t)^\alpha {(A(t)L(t))}^{1-\alpha} 
	\]
with  $0 <\alpha < 1$, satisfied all of these conditions, which is shown as follows.
\\
\begin{enumerate}
	\item Positive and diminishing returns in both factors: 	
		\begin{equation*}
		\begin{split}
			\frac{\partial F}{\partial K} & = \alpha K(t)^{\alpha - 1} {(A(t)L(t))}^{1-\alpha}\\
			& = \alpha \bigg(\frac{1}{K(t)}\bigg)^{1-\alpha} {(A(t)L(t))}^{1-\alpha}\\
			& = \alpha \bigg(\frac{A(t)L(t)}{K(t)}\bigg)^{1 - \alpha} > 0
		\end{split}
		\end{equation*}
		
		\begin{equation*}
		\begin{split}
			\frac{\partial ^2F}{\partial K^2} & = \frac{\partial \alpha \bigg(\frac{A(t)L(t)}{K(t)}\bigg)^{1 - \alpha}}{\partial K(t)} \\
			& = \alpha \frac{\partial \bigg(\frac{A(t)L(t)}{K(t)}\bigg)^{1 - \alpha}}{\partial \frac{A(t)L(t)}{K(t)}} \frac{\partial \frac{A(t)L(t)}{K(t)}}{\partial K(t)}\\
			& = \alpha(1-\alpha)\bigg(\frac{A(t)L(t)}{K(t)}\bigg)^{-\alpha}A(t)L(t)\frac{-1}{K(t)^2}\\
			& = - \alpha(1-\alpha)\frac{A(t)L(t)^{1-\alpha}}{K(t)^{2 - \alpha}} < 0
		\end{split}
		\end{equation*}
		
		\begin{equation*}
		\begin{split}
			\frac{\partial F}{\partial AL}& = (1-\alpha) K(t)^{\alpha} {(A(t)L(t))}^{-\alpha}\\
			& = (1-\alpha) \bigg(\frac{K(t)}{A(t)L(t)}\bigg)^{\alpha} > 0
		\end{split}
		\end{equation*}
		\\
		\begin{equation*}
		\begin{split}
			\frac{\partial ^2F}{\partial (AL)^2} & = \frac{\partial (1-\alpha) K(t)^{\alpha} {(A(t)L(t))}^{-\alpha}}{\partial A(t)L(t)} \\
			& = (1-\alpha) K(t)^{\alpha} \frac{\partial {(A(t)L(t))}^{-\alpha}}{\partial A(t)L(t)}\\
			& = (1-\alpha) K(t)^{\alpha} (-\alpha) (A(t)L(t))^{-\alpha -1}\\
			& =  -\alpha(1-\alpha)\frac{K(t)^\alpha}{ A(t)L(t)^{\alpha + 1} } < 0
		\end{split}
		\end{equation*}
	\\
	\item Constant returns to scale: 
		
			\begin{equation*}
			\begin{split}
				F(\lambda K(t), \lambda A(t)L(t)) & = (\lambda K(t))^\alpha {(\lambda A(t)L(t))}^{1-\alpha}\\
				& = \lambda^\alpha K(t)^\alpha \lambda^{1-\alpha} {( A(t)L(t))}^{1-\alpha}\\
				& = \lambda K(t)^\alpha {(A(t)L(t))}^{1-\alpha}\\
				& = \lambda F(K(t), A(t)L(t))
			\end{split}
			\end{equation*}
		
	\item \textit{Inada} conditions:
		\[
			\lim_{K \to 0}{\frac{\partial F}{\partial K}} = \lim_{K \to 0}{\alpha \bigg(\frac{A(t)L(t)}{K(t)}\bigg)^{1 - \alpha}} = \infty
		\]
		\\
		\[
			\lim_{AL \to 0}{\frac{\partial F}{\partial AL}} = \lim_{AL \to 0}{(1-\alpha) \bigg(\frac{K(t)}{A(t)L(t)}\bigg)^{\alpha}} = \infty
		\]
		\\
		\[
				\lim_{K \to \infty}{\frac{\partial F}{\partial K}} = \lim_{K \to \infty}{\alpha \bigg(\frac{A(t)L(t)}{K(t)}\bigg)^{1 - \alpha}} = 0
		\]	
		\\
		\begin{equation*}
		\begin{split}
			\lim_{AL \to \infty}{\frac{\partial F}{\partial AL}} = \lim_{AL \to \infty}{ (1-\alpha) \bigg(\frac{K(t)}{A(t)L(t)}\bigg)^{\alpha}} = 0
		\end{split}
		\end{equation*}
		\\
\end{enumerate}

\section{The Firms}
Firms maximise profits whilst taking factor prices as given as there is perfect competition in factor markets: 
	\[
		\pi = \max_{K(t), L(t)}  F(K(t), A(t)L(t)) - wL(t) - r^KK(t) 
	\]

Therefore, factors are paid their marginal products: 

	\begin{gather*}
		\frac{\partial F(K(t), A(t)L(t)) - wL(t) - r^KK(t)}{\partial K(t)} = 0 \\
		\iff \frac{\partial F(K(t), A(t)L(t))}{\partial K(t)} - r^K = 0\\
		\iff r^K  = F'_{K}(K(t), A(t)L(t))
	\end{gather*}	

	\begin{gather*}
		\frac{\partial F(K(t), A(t)L(t)) - wL(t) - r^KK(t)}{\partial L(t)} = 0 \\
		\iff \frac{\partial F(K(t), A(t)L(t))}{\partial L(t)} - w = 0\\
		\iff w = \frac{\partial F(K(t), A(t)L(t))}{\partial A(t)L(t)}\frac{\partial A(t)L(t)}{\partial L(t)}\\
		\iff w = A(t)F'_{AL}(K(t), A(t)L(t))
	\end{gather*}
\\
\textbf{Euler's Theorem (unproven):} If F(K, AL) is homogenous of degree 1, then $F'_{K}$ and $F'_{AL}$ are homogenous of degree 0. 
\\This means that the partial derivatives stay constant if we scale both inputs by some scalar: 
	
	\begin{gather*}
		r^K = F'_{K}(K(t), A(t)L(t)) = F'_{K}\bigg(\frac{K(t)}{A(t)L(t)}, 1\bigg)\\
		w = A(t)F'_{AL}(K(t), A(t)L(t)) = AF'_{AL}\bigg(\frac{K(t)}{A(t)L(t)}, 1\bigg)
	\end{gather*}
\\
We can show that this is true for Cobb Douglas technology. For $F'_{K}$:
\\
	\[
		F'_{K}(K(t), A(t)L(t)) = \alpha \bigg(\frac{A(t)L(t)}{K(t)}\bigg)^{1 - \alpha}
	\]		
Hence:
	
	\begin{equation*}
	\begin{split}
		F'_{K}\bigg(\frac{K(t)}{A(t)L(t)}, 1\bigg) & = \alpha \Bigg( \frac{1}{ \frac{K(t)}{A(t)L(t)}}\Bigg)^{1 - \alpha}\\
		& = \alpha \bigg(\frac{A(t)L(t)}{K(t)}\bigg)^{1 - \alpha}\\
		& = F'_{K}(K(t), A(t)L(t))
	\end{split}
	\end{equation*}
\\
Similarly for $F'_{AL}$: 
	\[
		F'_{AL}(K(t), A(t)L(t)) = (1-\alpha) \bigg(\frac{K(t)}{A(t)L(t)}\bigg)^{\alpha}
	\]
Hence:
	
	\begin{equation*}
	\begin{split}
		F'_{AL}\bigg(\frac{K(t)}{A(t)L(t)}, 1\bigg) & = (1-\alpha) \Bigg(\frac{\frac{K(t)}{A(t)L(t)}}{1}\Bigg)^{\alpha}\\
		& = (1-\alpha) \bigg(\frac{K(t)}{A(t)L(t)}\bigg)^{\alpha}\\
		& = F'_{AL}(K(t), A(t)L(t))
	\end{split}
	\end{equation*}
\\

\section{Balanced Growth Path}

A \textbf{Balanced Growth Path} (BGP) is an equilibrium path (or steady-state) for capital $K_t$, output $Y_t$, consumption $C_t$, wages $w_t$ and the return on capital $r^K$, such that these variables grow at a \textbf{constant rate}: 

\begin{align*}
	\frac{\dot{K}(t)}{K(t} &= g_K & \frac{\dot{Y}(t)}{Y(t} &= g_Y & \frac{\dot{C}(t)}{C(t} &= g_C
\end{align*}
\begin{align*}
	\frac{\dot{w}(t)}{w(t} &= g_w & \frac{\dot{r^K}(t)}{r^K(t} &= g_{r^K}
\end{align*}
\\
These properties are an accurate characterisation of most industrialised economies; all variables grow at a constant rate in every period and therefore grow exponentially over time. Can we find a balanced growth path in the Solow model? We start with the law of motion of capital: 

\begin{equation*}
\begin{split}
\dot{K}(t) & = sF(K(t), A(t)L(t)) - \delta K(t)\\
\implies \frac{\dot{K}(t)}{K(t)} & = s\frac{F(K(t), A(t)L(t))}{K(t)} - \delta\\
\implies g_K + \delta & = s\frac{F(K(t), A(t)L(t))}{K(t)}\\
\implies g_K + \delta & = sF\bigg(1, \frac{A(t)L(t)}{K(t)} \bigg) \text{ (by constant returns to scale)}
\end{split}
\end{equation*}
\\
By assumption, $g_K$ and $\delta$ are constant along the balanced growth path (in every period). This implies that $\frac{A(t)L(t)}{K(t)}$ is constant in the balanced growth path. Therefore: 


\begin{gather*}
	 \frac{A(t)L(t)}{K(t)} = const\\
	 \implies \ln\bigg(\frac{A(t)L(t)}{K(t)}\bigg) = \ln(const)\\
	 \implies \ln(A(t)) + \ln{L(t)} - \ln{K(t)} = \ln(const)\\
	 \implies \frac{\partial \ln(A(t)) + \ln{L(t)} - \ln{K(t)}}{\partial t} = \frac{\partial \ln(const)}{\partial t}\\
	 \implies \frac{\partial \ln(A(t)}{\partial t} + \frac{\partial \ln(L(t)}{\partial t} - \frac{\partial \ln(K(t)}{\partial t} = 0\\
\end{gather*}
Using the fact that $\frac{\partial \ln(X(t)}{\partial t} = \frac{\dot{X}(t)}{X(t}$, since $\frac{\partial \ln(X(t)}{\partial t} = \frac{\partial \ln(X(t)}{\partial X(t)}\frac{\partial X(t)}{\partial t} = \frac{1}{X(t)}\dot{X}(t) = \frac{\dot{X}(t)}{X(t}$:
\[
	\frac{\dot{A}(t)}{A(t)} + \frac{\dot{L}(t)}{L(t} - \frac{\dot{K}(t)}{K(t} = 0\\ 
\]
We recall that by assumption, $\frac{\dot{A}(t)}{A(t)} = g$ and $\frac{\dot{L}(t)}{L(t)} = n$, therefore:
\begin{gather*}
 	g + n - g_K = 0\\
	\implies g_K = g + n
\end{gather*}
We have found that in the balanced growth path, the capital stock $K(t)$ grows at a constant rate $g + n$. We now use this fact to derive growth rates of the other variables. 


\begin{equation*}
\begin{split}
	Y(t) & = F(K(t), A(t)L(t))\\
	\implies \frac{Y(t)}{K(t)} & = F\bigg(1, \frac{A(t)L(t)}{K(t)}\bigg) \text{ (by constant returns to scale)}\\
\end{split}
\end{equation*}
\\
As $\frac{A(t)L(t)}{K(t)}$ is constant along the balanced growth path, so is $F\bigg(1, \frac{A(t)L(t)}{K(t)}\bigg)$. Therefore, $\frac{Y(t)}{K(t)}$ is also constant. $Y(t)$ must grow at the same rate as $K(t)$ to keep $\frac{Y(t)}{K(t)}$ constant: 
\[
	g_Y = g_K = g + n
\]
\\
We now derive the growth rate of consumption.

\begin{equation*}
\begin{split}
	C(t) = (1-s)Y(t)\\
	\implies \frac{C(t)}{Y(t)} = 1 - s
\end{split}
\end{equation*}
\\
As $s$ is constant along the balanced growth path, so is $\frac{C(t)}{Y(t)}$. Therefore, $C(t)$ must grow at the same rate as $Y(t)$ (and therefore $K(t)$) to keep $\frac{C(t)}{Y(t)}$ constant: 
\[
	g_C = g_Y = g_K = g + n
\]
\\
We now derive the growth rate of wages.

\begin{equation*}
\begin{split}
	w(t) = A(t)F'_L(K(t), A(t)L(t))\\
\end{split}
\end{equation*}
\\
As the partial derivative of the production function is homogenous of degree 0 (Euler's Theorem): 

\begin{equation*}
\begin{split}
	w(t) = A(t)F'_L\bigg(1, \frac{A(t)L(t)}{K(t)}\bigg)\\
\end{split}
\end{equation*}
\\
Since $\frac{A(t)L(t)}{K(t)}$ is constant along the balanced growth path, so is $F'_L\bigg(1, \frac{A(t)L(t)}{K(t)}\bigg)$. Therefore, $w(t)$ grows at the same rate as $A(t)$:
\[
	g_w = n
\]
\\
We now derive the growth rate of the return on capital.
\begin{equation*}
\begin{split}
r^K  &= F'_{K}(K(t), A(t)L(t))\\
& = F'_{K}\bigg(1, \frac{A(t)L(t)}{K(t)}\bigg) \text{ (by Euler's Theorem)}
\end{split}
\end{equation*}
\\
Since $\frac{A(t)L(t)}{K(t)}$ is constant along the balanced growth path, so is $F'_K\bigg(1, \frac{A(t)L(t)}{K(t)}\bigg)$. Therefore, $r^K$ is also constant: 
\[
	g_{r^K} = 0
\]
\\
Our assumptions have allowed us to derive a growth path in which all of the variables grow at a constant rate.
\\

\section{Dynamics to the Balanced Growth Path}

We start by writing the production function in intensive form: 

\[
	\widetilde{y}(t) = \frac{Y(t)}{A(t)L(t)} = \frac{F(K(t), A(t)L(t))}{A(t)L(t)} = F\bigg(\frac{K(t)}{A(t)L(t)}, 1\bigg) = f(\widetilde{k}(t))
\]
where $\widetilde{y}(t)$ is output per efficient unit of labour and $\widetilde{k}(t)$ is capital per efficient unit of labour. We will show that the economy converges to the balanced growth path in equilibrium. On one hand we have: 
\begin{gather*}
	\dot{K}(t) = sY(t) - \delta K(t)\\
	\implies \frac{\dot{K}(t)}{A(t)L(t)} = s\frac{Y(t)}{A(t)L(t)} - \delta \frac{K(t)}{A(t)L(t)}\\
	\implies \frac{\dot{K}(t)}{A(t)L(t)} = sf(\widetilde{k}(t)) - \delta \widetilde{k}(t)
\end{gather*}
\\
On the other hand we have: 

\begin{equation*}
\begin{split}
	\dot{\widetilde{k}}(t) = \frac{d \widetilde{k}(t)}{dt} = \frac{d \frac{K(t)}{A(t)L(t)}}{dt} & = \frac{ \frac{d K(t)}{dt} A(t)L(t) - K(t)\frac{d A(t)L(t)}{dt} }{(A(t)L(t))^2}\\
	& = \frac{ \dot{K}(t)A(t)L(t) - K(t)\Big(\frac{dA(t)}{dt}L(t) + A(t)\frac{dL(t)}{dt} \Big)}{(A(t)L(t))^2}\\
	& = \frac{ \dot{K}(t)A(t)L(t) - K(t)\Big(\dot{A}(t)L(t) + A(t)\dot{L}(t) \Big)}{(A(t)L(t))^2}\\
	& = \frac{\dot{K}(t)} {A(t)L(t)} - \frac{K(t)}{A(t)L{t}}\bigg(\frac{\dot{A}(t)L(t)}{A(t)L(t)} + \frac{ A(t)\dot{L}(t)} {A(t)L(t)}\bigg)\\
	& = \frac{\dot{K}(t)} {A(t)L(t)} - \widetilde{k}(t)\bigg(\frac{\dot{A}(t)}{A(t)} + \frac{\dot{L}(t)} {L(t)}\bigg)\\
	& = \frac{\dot{K}(t)} {A(t)L(t)} - \widetilde{k}(t)\big(g + n\big)\\
	& = sf(\widetilde{k}(t)) - \delta \widetilde{k}(t) - \widetilde{k}(t)\big(g + n\big) \text{ (from the above expression)}\\
	\implies \dot{\widetilde{k}}(t) & = sf(\widetilde{k}(t)) - \big(n + g + \delta\big)\widetilde{k}(t)
\end{split}
\end{equation*}
\\
\\
We can see that the change in capital per effective unit of labour equals the fraction of output per effective unit of about that is saved, minus the fraction of capital that depreciates in ever period.
\\
\begin{center}
	\includegraphics[scale=0.7]{dynamic}
\end{center}


We observe that :

\begin{equation*}
\begin{split}
	\dot{\widetilde{k}}(t) = sf(\widetilde{k}(t)) - \big(n + g + \delta\big)\widetilde{k}(t)\\
	\implies \gamma_{\widetilde{k}} = \frac{\dot{\widetilde{k}}(t)}{\widetilde{k}(t)} = s 
	\frac{f(\widetilde{k}(t))}{\widetilde{k}} - (n + \delta + g)
\end{split}
\end{equation*}
\\
where $\gamma_{\widetilde{k}}$ is the growth rate of capital per effective unit of labour. By constant returns to scale, we have: 

\begin{equation*}
\begin{split}
	\gamma_{\widetilde{k}} = \frac{\dot{\widetilde{k}}(t)}{\widetilde{k}(t)} 	& = s \frac{F(\widetilde{k}(t), 1)}{\widetilde{k}} - (n + \delta + g)\\
	& = sF\bigg(1, \frac{1}{\widetilde{k}(t)}\bigg) - (n + \delta + g)\\
	\implies \frac{\partial \gamma_{\widetilde{k}}}{\partial \widetilde{k}(t)} & = s \frac{\partial F\bigg(1, \frac{1}{\widetilde{k}(t)}\bigg)}{\partial \widetilde{k}(t)}\\
	& =  s \frac{\partial F\bigg(1, \frac{1}{\widetilde{k}(t)}\bigg)}{\partial \frac{1}{\widetilde{k}(t)}} \frac{\partial \frac{1}{\widetilde{k}(t)}}{\partial \widetilde{k}(t)}\\
	& = s F'_L \frac{-1}{\widetilde{k}(t)^2}\\
	& = -s F'_L \frac{1}{\widetilde{k}(t)^2} < 0
\end{split}
\end{equation*}
\\
Moreover, by l'Hopital rule:
\[
	\lim_{\widetilde{k}(t) \to 0}{ \bigg[\frac{sf(\widetilde{k}(t)) }{\widetilde{k}} \bigg]} = \lim_{\widetilde{k}(t) \to 0}{ \bigg[\frac{sf'(\widetilde{k}(t)) }{1} \bigg]} = \infty
\]
\[
	\lim_{\widetilde{k}(t) \to \infty}{ \bigg[\frac{sf(\widetilde{k}(t)) }{\widetilde{k}} \bigg]} = \lim_{\widetilde{k}(t) \to \infty}{ \bigg[\frac{sf'(\widetilde{k}(t)) }{1} \bigg]} = 0
\]
\\
Therefore the model is globally stable and there is a unique $\widetilde{k}(t)^*$ such that: 


\begin{equation*}
\begin{split}
	\frac{\dot{\widetilde{k}}(t)}{\widetilde{k}(t)} = 0 \implies \frac{sf(\widetilde{k}(t)^*)}{\widetilde{k}(t)^*} = n + g + \delta
\end{split}
\end{equation*}

\begin{center}
	\includegraphics[scale=0.3]{converg}
\end{center}
\begin{enumerate}
	\item   If $\widetilde{k}(t) < \widetilde{k}(t)^*$, saving/investment exceeds depreciation, thus capital per efficient unit of labour grows: $\gamma_{\widetilde{k}} > 0$. 
	\item If $\widetilde{k}(t) > \widetilde{k}(t)^*$, saving/investment is lower than depreciation, thus capital per efficient unit of labour decreases: $\gamma_{\widetilde{k}} < 0$. 
	\item As $f(\widetilde{k}(t))$ satisfies continuity, concavity and the Inada conditions, by the Intermediate Value Theorem (IVT) there must exist an unique $\widetilde{k}(t)^*$ such that $sf(\widetilde{k}(t)^*) = (n + g + \delta)\widetilde{k}(t)^*$.
	\item In the Balanced Growth Path equilibrium, capital, output and consumption per efficient unit of labour are constant: $\gamma_{\widetilde{k}} = \gamma_{\widetilde{y}} = \gamma_{\widetilde{c}} = 0$.
\end{enumerate}We now use these results to derive the growth rates of aggregate and per capita variables. 	

\begin{equation*}
\begin{split}
	\widetilde{k}(t) & = \frac{K(t)}{A(t)L(t)}\\
	\implies K(t) & = A(t)L(t)\widetilde{k}(t)\\
	\implies \ln{K(t)} & = \ln{\Big(A(t)L(t)\widetilde{k}(t)}\Big)\\
	\implies \ln{K(t)} & = \ln{A(t)} + \log{L(t)} + \ln{\widetilde{k}(t)}\\
	\implies \frac{\partial \ln{K(t)}}{\partial t} & = \frac{\partial \ln{A(t)}}{\partial t} + \frac{\partial \ln{L(t)}}{\partial t} + \frac{\partial \ln{\widetilde{k}(t)}}{\partial t}\\
	\implies \frac{\dot{K}(t)}{K(t)} & = \frac{\dot{A}(t)}{A(t)} + \frac{\dot{L}(t)}{L(t)} + \frac{\dot{\widetilde{k}}(t)}{\widetilde{k}(t)}\\
	\implies \frac{\dot{K}(t)}{K(t)} & = g + n + \gamma_{\widetilde{y}}\\
	\implies \frac{\dot{K}(t)}{K(t)} & = g + n \text{ (as $\gamma_{\widetilde{y}} = 0$ in the BGP)}\\
\end{split}
\end{equation*}
\\
We find that aggregate output grows at rate g + n. Similarly, we derive the growth rate of output per capita: 

\begin{equation*}
\begin{split}
	k(t) = \widetilde{k}(t)A(t) & = \frac{K(t)}{L(t)} \text{ (capital per unit of labour)}\\
	\implies \ln{k(t)} & = \ln{\Big(\widetilde{k}(t)A(t)\Big)}\\
	\implies \ln{k(t)} & = \ln{\widetilde{k}(t)} + \ln{A(t)}\\
	\implies \frac{\partial \ln{k(t)}}{\partial t} & = \frac{\partial \Big(\widetilde{k}(t) + \ln{A(t)}\Big)}{\partial t}\\
	\implies \frac{\partial \ln{k(t)}}{\partial t} & = \frac{\partial \widetilde{k}(t)}{\partial t} + \frac{\partial \ln{A(t)}}{\partial t}\\
	\implies \frac{\dot{k}(t)}{k(t)} & = \frac{\dot{\widetilde{k}}(t)}{\widetilde{k}(t)} + \frac{\dot{A}(t)}{A(t)}\\
	\implies \frac{\dot{k}(t)}{k(t)} & = \gamma_{\widetilde{y}} + g\\
	\implies \frac{\dot{k}(t)}{k(t)} & = g \text{ (as $\gamma_{\widetilde{y}} = 0$ in the BGP)}
\end{split}
\end{equation*}
\\
We find that output per capita grows at rate g (the rate of technological progress). Furthermore, changes in the parameters of the model such as $s$, $n$ or $\delta$ will affect the levels of $k^*$, $y^*$ and $c^*$ as they shift the depreciation curve or the investment per unit of effective labour function. However, they do not affect the growth rates of these variables is the balanced growth path.
		
\section{Golden Rule and Dynamic Inefficiency}

The \textbf{golden rule saving rate} maximises consumption in the long run (balanced growth path). We maximise consumption per unit of labour $C(t)$ with respect to $s$, which is equivalent to maximising $\frac{C(t)}{A(t)L(t)} = \widetilde{c}(t)$ with respect to $s$:

\[
	s = \arg\max_{s} \widetilde{c}(t)^* 
\]
\[
	\text{ where } \widetilde{c}(t)^* = (1 - s)f(\widetilde{k}(t)^*) = f(\widetilde{k}(t)^*) - (n + g + \delta )\widetilde{k}^*
\]
\\
We differentiate $\widetilde{c}(t)^*$ with respect to $s$:

\begin{equation*}
\begin{split}
	\frac{\partial \widetilde{c}(t)^*}{\partial s} & = \frac{\partial f(\widetilde{k}(t)^*) - (n + g + \delta )\widetilde{k}(t)^*}{\partial s} = 0\\
	& \implies \frac{\partial f(\widetilde{k}(t)^*)}{\partial s} -  (n + g + \delta )\frac{\partial \widetilde{k}(t)^*}{\partial s} = 0\\
	& \implies \frac{\partial f(\widetilde{k}(t)^*)}{\partial \widetilde{k}(t)^*}\frac{\partial \widetilde{k}(t)^*}{\partial s} -  (n + g + \delta )\frac{\partial \widetilde{k}(t)^*}{\partial s} = 0\\
	& \implies \bigg(\frac{\partial f(\widetilde{k}(t)^*)}{\partial \widetilde{k}(t)^*} -  (n + g + \delta )\bigg)\frac{\partial \widetilde{k}(t)^*}{\partial s} = 0\\
	& \implies \frac{\partial f(\widetilde{k}(t)^*)}{\partial \widetilde{k}(t)^*} -  (n + g + \delta ) =  0\\
	& \implies f'(\widetilde{k}(t)^*_{GR}) -  (n + g + \delta ) =  0\\
\end{split}
\end{equation*}

\begin{center}
	\includegraphics[scale=0.5]{golden}
\end{center}
We find that for the golden rule savings rate, the capital stock is such that the marginal product of capital is equal to the depreciation rate of capital $n + g + \delta$. Given $\widetilde{k}(t)^*_{GR}$ we can use $sf(\widetilde{k}(t)^*_{GR}) = (n + g + \delta)\widetilde{k}(t)^*_{GR}$ to find $s_{GR}$: 


\begin{equation*}
\begin{split}
	sf(\widetilde{k}(t)^*_{GR}) = (n + g + \delta)\widetilde{k}(t)^*_{GR}\\
	\implies s_{GR} = (n + g + \delta)\frac{\widetilde{k}(t)^*_{GR}}{f\big(\widetilde{k}(t)^*_{GR}\big)}
\end{split}
\end{equation*}
\\
\\
If $s < s_{GR}$, then increases in $s$ would increase $\widetilde{c}(t)^*$ in the long run: 
\begin{center}
	\includegraphics[scale=0.5]{increase}
\end{center}

If $s > s_{GR}$, then increases in $s$ would decrease $\widetilde{c}(t)^*$ in the long run (the economy is dynamically inefficient): 
\begin{center}
	\includegraphics[scale=0.5]{decrease}
\end{center}
We can represent the relationship between the savings rate and consumption per unit of effective labour in the balanced growth path as follows: 
\begin{center}
	\includegraphics[scale=0.3]{savings}
\end{center}

\section{Effect of a Change in the Savings Rate}

Suppose that initially the economy is in its balanced growth path equilibrium, $\widetilde{k}(t)^*_1$. We recall the fundamental equation*: 

\[
	\dot{\widetilde{k}}(t) = sf(\widetilde{k}) - (n + g + \delta)\widetilde{k}(t)
\]

\begin{enumerate}
	\item If the savings rate $s$ increases, investment exceeds depreciation: 
	
	\begin{equation*}
	\begin{split}
		sf(\widetilde{k}(t)^*_1) & > (n + g + \delta)\widetilde{k}(t)^*_1\\
		\implies  \dot{\widetilde{k}}(t) = sf(\widetilde{k}) & - (n + g + \delta)\widetilde{k}(t) > 0 
	\end{split}
	\end{equation*}
	
	\item The capital stock $\widetilde{k}$ grows until it reaches a new higher balanced growth path. 
	
	\item Along the transition, $\widetilde{k}$ and $\widetilde{y}$ rises, but the their growth rates slow down. 
	
	\item In the new balanced growth path, $\widetilde{k}$, $\widetilde{y}$ and $\widetilde{c}$ are constant. Per capital variables $\frac{K}{L}$, $\frac{Y}{L}$ and $\frac{C}{L}$ grow at rate $g$.	
\end{enumerate}
We can plot the evolution of variables over time following an increase in the savings rate as follows: 

\begin{center}
	\includegraphics[scale=0.6]{evol}
\end{center}

\section{Speed of Convergence}

The transition speed towards the steady state as 
\begin{enumerate}
	\item If it is rapid, we can focus on the steady state.
	\item If it is slow, we need to consider transitional dynamics (such as hysteresis) to evaluate long-term outcomes.\\
\end{enumerate}
Assuming Cobb-Douglas technology: 
\begin{equation*}
\begin{split}
	f(\widetilde{k}(t)) & = \widetilde{y} = \frac{Y(t)}{A(t)L(t)} = \frac{K(t)^\alpha\big(A(t)L(t)\big)^{1-\alpha}}{A(t)L(t)} = \frac{K(t)^\alpha}{\big(A(t)L(t)\big)^\alpha} = \widetilde{k}(t)^\alpha
\end{split}
\end{equation*}
\\
We use this to rewrite the derivative of capital per effective unit of labour. 
\begin{equation*}
\begin{split}
	\dot{\widetilde{k}}(t) & = sf(\widetilde{k}(t)) - \big(n + g + \delta\big)\widetilde{k}(t)\\
	\implies \dot{\widetilde{k}}(t) & = s\widetilde{k}(t)^\alpha - \big(n + g + \delta\big)\widetilde{k}(t)\\
\end{split}
\end{equation*}
\\
We take the first order Taylor approximation of this function around $\widetilde{k}^*$ to evaluate the speed of convergence around the steady state. We define the first order Taylor approximation as: 
\begin{equation*}
\begin{split}
	g(x) \Big\rvert_{x^*} = g(x^*) + g'(x^*)(x - x^*)
\end{split}
\end{equation*}
We apply this to $\dot{\widetilde{k}}(t)$ as follows:
\begin{equation*}
\begin{split}
	\dot{\widetilde{k}}(t) \Big\rvert_{\widetilde{k}(t)^*} & = \dot{\widetilde{k}}(t)^* + \frac{\partial \dot{\widetilde{k}}(t)^*}{\partial  \widetilde{k}(t)^*}\Big(\widetilde{k}(t) - \widetilde{k}(t)^*\Big)\\
\end{split}
\end{equation*}
We can write $\dot{\widetilde{k}}(t)^*$ as: 
\begin{equation*}
\begin{split}
\dot{\widetilde{k}}(t)^* = s(\widetilde{k}(t)^*)^\alpha - \big(n + g + \delta\big)\widetilde{k}(t)^*
\end{split}
\end{equation*}
Which we can differentiate with respect to $\widetilde{k}(t)^*$: 
\begin{equation*}
\begin{split}
\frac{\partial \dot{\widetilde{k}}(t)^*}{\partial  \widetilde{k}(t)^*} = s\alpha(\widetilde{k}(t)^*)^{\alpha -1} - \big(n + g + \delta\big)
\end{split}
\end{equation*}
We can now plug this into the Taylor approximation as follows:
\begin{equation*}
\begin{split}
	\dot{\widetilde{k}}(t) \Big\rvert_{\widetilde{k}(t)^*} = s(\widetilde{k}(t)^*)^\alpha - \big(n + g + \delta\big)\widetilde{k}(t)^* + \Big(s\alpha(\widetilde{k}(t)^*)^{\alpha -1} - \big(n + g + \delta\big)\Big)\bigg(\widetilde{k}(t) - \widetilde{k}(t)^*\bigg) \\
\end{split}
\end{equation*}
\\
Furthermore, in the steady state: 
\begin{equation*}
\begin{split}
	\dot{\widetilde{k}}(t)^* = s(\widetilde{k}(t)^*)^\alpha - \big(n + g + \delta\big)\widetilde{k}(t)^* &= 0\\
	\implies s(\widetilde{k}(t)^*)^\alpha & = \big(n + g + \delta\big)\widetilde{k}(t)^*\\
	\implies s\frac{s(\widetilde{k}(t)^*)^\alpha}{\widetilde{k}(t)^*} & = n + g + \delta\\
	\implies s(\widetilde{k}(t)^*)^{\alpha - 1} & = n + g + \delta\\
	\implies \alpha s(\widetilde{k}(t)^*)^{\alpha - 1} & = \alpha(n + g + \delta)
\end{split}
\end{equation*}
\\
Therefore, we can rewrite $\dot{\widetilde{k}}(t) \Big\rvert_{\widetilde{k}(t)^*}$ without the first term, since $\dot{\widetilde{k}}(t)^* = 0$:
\begin{equation*}
\begin{split}
	\dot{\widetilde{k}}(t) \Big\rvert_{\widetilde{k}(t)^*} =  \Big(s\alpha(\widetilde{k}(t)^*)^{\alpha -1} - \big(n + g + \delta\big)\Big)\bigg(\widetilde{k}(t) - \widetilde{k}(t)^*\bigg) \\
\end{split}
\end{equation*}
\\
We now plug in $\alpha s(\widetilde{k}(t)^*)^{\alpha - 1} = \alpha(n + g + \delta)$: 
\begin{equation*}
\begin{split}
	\dot{\widetilde{k}}(t) \Big\rvert_{\widetilde{k}(t)^*} & = \Big(\alpha(n + g + \delta) - \big(n + g + \delta\big)\Big)\bigg(\widetilde{k}(t) - \widetilde{k}(t)^*\bigg)\\
	& = (\alpha - 1)(n + g + \delta)\bigg(\widetilde{k}(t) - \widetilde{k}(t)^*\bigg) \\
	& = - (1 - \alpha)(n + g + \delta)\bigg(\widetilde{k}(t) - \widetilde{k}(t)^*\bigg) \\
\end{split}
\end{equation*}
\\
Which we can now rewrite as: 
\begin{equation*}
\begin{split}
	\dot{\widetilde{k}}(t) \Big\rvert_{\widetilde{k}(t)^*} = - \beta\bigg(\widetilde{k}(t) - \widetilde{k}(t)^*\bigg) \\
\end{split}
\end{equation*}
\\
where $\beta = (1 - \alpha)(n + g + \delta)$. We now solve this differential equation. We re-write it as: 
\begin{equation*}
\begin{split}
	\dot{\widetilde{k}}(t) = - \beta \widetilde{k}(t) + \beta \widetilde{k}(t)^* \\
\end{split}
\end{equation*}
\\
This differential equation is in the form $\dot{x}(t) = ax(t) + b$.  We first find the particular solution: 
\begin{equation*}
\begin{split}
	\dot{\widetilde{k}}(t) =  - \beta \widetilde{k}(t) + \beta \widetilde{k}(t)^* =  0 \\
	\implies - \beta\bigg(\widetilde{k}(t) - \widetilde{k}(t)^*\bigg) = 0\\
	\implies \widetilde{k}(t) - \widetilde{k}(t)^* = 0\\
	\implies \widetilde{k}(t)^P = \widetilde{k}(t)^*
\end{split}
\end{equation*}
\\

We then find the complementary solution. We solve: 
\begin{equation*}
\begin{split}
	\dot{\widetilde{k}}(t) =  - \beta \widetilde{k}(t) =  0\\
	\implies \widetilde{k}(t)^H = Ce^{-\beta t}
\end{split}
\end{equation*}
\\
We combine the two solutions to express the general solution: 
\begin{equation*}
\begin{split}
	\widetilde{k}(t)^G &= \widetilde{k}(t)^H + \widetilde{k}(t)^P\\
	&= Ce^{-\beta t} + \widetilde{k}(t)^*
\end{split}
\end{equation*}
\\
We need to pin down C. W notice that: 
\begin{equation*}
\begin{split}
	\widetilde{k}(0) &= Ce^{-\beta 0} + \widetilde{k}(t)^*\\
	\implies \widetilde{k}(0) & = C + \widetilde{k}(t)^*\\
	\implies C &= \widetilde{k}(0) - \widetilde{k}(t)^*
\end{split}
\end{equation*}
\\
We can plug this into the general solution: 
\begin{equation*}
\begin{split}
	\widetilde{k}(t) &= \big(\widetilde{k}(0) - \widetilde{k}(t)^*\big)e^{-\beta t} + \widetilde{k}(t)^*\\
	&= \widetilde{k}(0)e^{-\beta t} - \widetilde{k}(t)^*e^{-\beta t} + \widetilde{k}(t)^*\\
	&= \widetilde{k}(0)e^{-\beta t} + (1 - e^{-\beta t})\widetilde{k}(t)^*\\
\end{split}
\end{equation*}
\\
We define $t = t_{half}$ such that: 
\begin{equation*}
\begin{split}
	\widetilde{k}(t_{half}) - \widetilde{k}(0) = \frac{\widetilde{k}(t)^* - \widetilde{k}(0)}{2} \\
\end{split}
\end{equation*}
\\
where $t_{half}$ is the time it takes for capital to reach the half-way point between it's initial level $\widetilde{k}(0)$ and the steady-state $\widetilde{k}(t)^*$. It is an indication of the speed of convergence. We can solve for $t_{half}$:
\begin{equation*}
\begin{split}
	 \widetilde{k}(0)e^{-\beta t_{half}} + (1 - e^{-\beta t_{half}})\widetilde{k}(t)^* - \widetilde{k}(0) &= \frac{\widetilde{k}(t)^* - \widetilde{k}(0)}{2} \\\
	 \implies  (1 - e^{-\beta t_{half}})\widetilde{k}(t)^* + \widetilde{k}(0)e^{-\beta t_{half}} - \widetilde{k}(0) &= \frac{\widetilde{k}(t)^* - \widetilde{k}(0)}{2} \\
	 \implies  (1 - e^{-\beta t_{half}})\widetilde{k}(t)^* + (e^{-\beta t_{half}} - 1)\widetilde{k}(0) &= \frac{\widetilde{k}(t)^* - \widetilde{k}(0)}{2} \\
	 \implies  (1 - e^{-\beta t_{half}})\widetilde{k}(t)^* - (1 - e^{-\beta t_{half}})\widetilde{k}(0) &= \frac{\widetilde{k}(t)^* - \widetilde{k}(0)}{2} \\
	 \implies  (1 - e^{-\beta t_{half}})\Big(\widetilde{k}(t)^* - \widetilde{k}(0)\Big) &= \frac{1}{2}\Big( \widetilde{k}(t)^* - \widetilde{k}(0) \Big) \\
	 \implies  (1 - e^{-\beta t_{half}}) &= \frac{1}{2}\\
	 \implies  - e^{-\beta t_{half}} &= \frac{-1}{2}\\
	 \implies  e^{-\beta t_{half}} &= \frac{1}{2}\\
	 \implies  -\beta t_{half} &= \ln{\frac{1}{2}}\\
	 \implies  -\beta t_{half} &= \ln{1} - \ln{2}\\
	 \implies  -\beta t_{half} &= - \ln{2}\\
	 \implies  \beta t_{half} &= \ln{2}\\
	 \implies  t_{half} &= \frac{\ln{2}}{\beta}\\
\end{split}
\end{equation*}
\\
An increase in $\beta$ leads to a decrease in the time it takes for capital to reach the half-way point, and therefore an increase in the speed of convergence. We note that $\beta = (1 - \alpha)(n + g + \delta)$, which determines the speed of convergence, does not depend on the savings rate $s$. In the United States, $\alpha = \frac{1}{3}$, $n = 0.01$, $\delta = 0.06$ and $g = 0.02$. Therefore, $\beta = 0.0533$ and $t_{half} = 13.125$.


\section{Picketty (2014)}

\begin{enumerate}
	\item Piketty's first law of capitalism. Capital's share of national income is (by definition): 
	
\begin{equation*}
\begin{split}
	\alpha_{K} = r^K \frac{K}{Y} \\
\end{split}
\end{equation*}
\\

	\item Piketty's second law of capitalism. In the long run the capital-to-output ratio is (by theory): 
	
\begin{equation*}
\begin{split}
	\frac{K}{Y} &= \frac{s}{g} \\
	\implies \alpha_{K} &= r^K \frac{s}{g}\\
\end{split}
\end{equation*}
\\
	
Therefore, if the rate of technological progress $g$ falls to 0, then capital's share of national income $\alpha_{K}$ goes to infinity: 

\begin{equation*}
\begin{split}
	\lim_{g \to 0}{\alpha_{K}} = \lim_{g \to 0}{ r^K \frac{s}{g} } = \infty
\end{split}
\end{equation*}
\\

If $g$ declines to zero, then capital's share of income would increase explosively.

	\item Piketty's key observation: 
\begin{equation*}
\begin{split}
	r^K > g
\end{split}
\end{equation*}
The rental rate of capital is larger than the long run growth rate and rate of technological growth $g$.
\end{enumerate}

\section{Piketty's model}

We consider a standard solow model with $Y(t) = C(t) + I(t)$, $Y(t) = F(K(t), A(t)L(t))$ and $I(t) = sY(t)$. We write the law of motion of capital: 
\begin{equation*}
\begin{split}
	\dot{K}(t) &= I(t) - \delta K(t)\\
	\implies \dot{K}(t) &= sY(t) - \delta K(t)\\
	\implies \frac{\dot{K}(t)}{K(t)} &= \frac{sY(t) - \delta K(t)}{K(t)}\\
	\implies \frac{\dot{K}(t)}{K(t)} &= s\frac{Y(t)}{K(t)} - \delta 
\end{split}
\end{equation*}
Along the Balanced Growth Path, capital, output and technology grows at a constant growth rate: $\frac{\dot{K}(t)}{K(t)} = g_K = g_A + n = g$. Therefore, 
\begin{equation*}
\begin{split}
	g = s\frac{Y(t)}{K(t)} - \delta \\
	\implies g + \delta = s\frac{Y(t)}{K(t)}\\
	\implies \frac{g + \delta}{s} = \frac{Y(t)}{K(t)}\\
	\implies \frac{K(t)}{Y(t)} = \frac{s}{g + \delta}
\end{split}
\end{equation*}
We find that as the rate of technological progress $g$ falls to 0, capital's share of national income $\frac{K(t)}{Y(t)}$ increases. However, there is a limit on to how large the ratio can grow. For example, if $g = 0$, $s = 0.24$ and $\delta = 0.08$, then: 
\begin{equation*}
\begin{split}
	\Big(\frac{K(t)}{Y(t)}\Big)_{MAX} = \frac{0.24}{0.08} = 3 \\
\end{split}
\end{equation*}
\\
We now define:
\begin{equation*}
\begin{split}
	\widetilde{I}(t) = \widetilde{s}\widetilde{Y}(t) &= \widetilde{s}\Big(F\big(K(t), A(t)L(t)\big) - \delta K(t)\Big)\\
	&= \widetilde{s}F\big(K(t), A(t)L(t)\big) - \widetilde{s}\delta K(t)\\
\end{split}
\end{equation*}
where 
\begin{equation*}
\begin{split}
	\widetilde{Y}(t) &= F\big(K(t), A(t)L(t)\big) - \delta K(t)\\
	\implies \widetilde{Y}(t) &= Y(t) - \delta K(t)
\end{split}
\end{equation*}
\\
and similarly we have: 
\begin{equation*}
\begin{split}
	\widetilde{I}(t) &= I(t) - \delta K(t)\\
	\implies I(t) = \widetilde{I}(t) + \delta K(t) &= \widetilde{s}F\big(K(t), A(t)L(t)\big) - \widetilde{s}\delta K(t) + \delta K(t)\\
	\implies I(t) &= \widetilde{s}F\big(K(t), A(t)L(t)\big) + (1 - \widetilde{s})\delta K(t)
\end{split}
\end{equation*}
\\
We recall that $\dot{K}(t) = I(t) - \delta K(t)$, so we can write: 
\begin{equation*}
\begin{split}
	\dot{K}(t) = I(t) - \delta K(t) &= \widetilde{s}F\big(K(t), A(t)L(t)\big) + (1 - \widetilde{s})\delta K(t)	- \delta K(t)\\
	&= \widetilde{s}F\big(K(t), A(t)L(t)\big) + (1 - \widetilde{s} - 1)\delta K(t) \\
	&= \widetilde{s}F\big(K(t), A(t)L(t)\big) - \widetilde{s} \delta K(t) \\	
	&= \widetilde{s}\Big(F\big(K(t), A(t)L(t)\big) - \delta K(t)\Big) \\	
\end{split}
\end{equation*}
\\
We write the law of motion of capital: 
\begin{equation*}
\begin{split}
	\dot{K}(t) &= \widetilde{s}F\big(K(t), A(t)L(t)\big) - \widetilde{s} \delta K(t) \\	
	\implies \frac{\dot{K}(t)}{K(t)} &= \frac{\widetilde{s}F\big(K(t), A(t)L(t)\big) - \widetilde{s} \delta K(t)}{K(t)}\\
	\implies \frac{\dot{K}(t)}{K(t)} &= \frac{\widetilde{s}Y(t)}{K(t)} - \frac{\widetilde{s} \delta K(t)}{K(t)}\\
	\implies \frac{\dot{K}(t)}{K(t)} &= \widetilde{s}\frac{Y(t)}{K(t)} - \widetilde{s} \delta  \\
\end{split}
\end{equation*}
\\
Along the Balanced Growth Path, capital, output and technology grow at a constant growth rate: $\frac{\dot{K}(t)}{K(t)} = g_K = g_A + n = g$. Therefore, 
\begin{equation*}
\begin{split}
	g = \widetilde{s}\frac{Y(t)}{K(t)} - \widetilde{s} \delta  \\
	\implies 	g +  \widetilde{s} \delta = \widetilde{s}\frac{Y(t)}{K(t)}  \\
	\implies 	\frac{g +  \widetilde{s} \delta}{\widetilde{s}} = \frac{Y(t)}{K(t)}  \\
	\implies 	\frac{K(t)}{Y(t)} = \frac{\widetilde{s}}{g +  \widetilde{s} \delta}  \\
\end{split}
\end{equation*}
\\
We find that as the rate of technological progress $g$ falls to 0, capital's share of national income $\frac{K(t)}{Y(t)}$ increases. However, there is a limit on to how large the ratio can grow. For example, if $g = 0$, $\widetilde{s} = 0.24$ and $\delta = 0.08$, then: 

\begin{equation*}
\begin{split}
	\Big(\frac{K(t)}{Y(t)}\Big)_{MAX} = \frac{\widetilde{s}}{g + \widetilde{s} \delta} = \frac{0.24}{0.24*0.08} = \frac{1}{0.08} = 12.5 \\
\end{split}
\end{equation*}
\\
Under this modified model, we have: 
\begin{equation*}
\begin{split}
	C(t) = Y(t) - I(t) &= Y(t) - \Big( \widetilde{s}F\big(K(t), A(t)L(t)\big) + (1 - \widetilde{s})\delta K(t)  \Big) \\
	&= Y(t) - \widetilde{s}Y(t) - (1 - \widetilde{s})\delta K(t) \\ 
	&= (1 - \widetilde{s})\big(Y(t) - \delta K(t)\big) \\ 
	\implies \frac{C(t)}{Y(t)} &= (1 - \widetilde{s})\frac{Y(t) - \delta K(t)}{Y(t)}\\
	&= (1 - \widetilde{s})\big(1 - \delta \frac{K(t)}{Y(t)}\big)
\end{split}
\end{equation*}
\\
We plug in $\frac{K(t)}{Y(t)} = \frac{\widetilde{s}}{g +  \widetilde{s} \delta}$: 
\begin{equation*}
\begin{split}
	\frac{C(t)}{Y(t)} &= (1 - \widetilde{s})\Bigg(1 - \delta \frac{\widetilde{s}}{g +  \widetilde{s} \delta} \Bigg)\\
	&= (1 - \widetilde{s})\Bigg(\frac{g +  \widetilde{s} \delta}{g +  \widetilde{s} \delta} - \frac{\delta \widetilde{s}}{g +  \widetilde{s} \delta} \Bigg)\\
	&= (1 - \widetilde{s})\Bigg(\frac{g}{g +  \widetilde{s} \delta} \Bigg)
\end{split}
\end{equation*}
\\

The savings rate is $s = \frac{Y(t) - C(t)}{Y(t)} = 1 - \frac{C(t)}{Y(t)}$. Therefore: 
\begin{equation*}
\begin{split}
	s = 1 - \frac{C(t)}{Y(t)} &= 1 - (1 - \widetilde{s})\Bigg(\frac{g}{g +  \widetilde{s} \delta} \Bigg)\\
	&= 1 - \frac{(1 - \widetilde{s})g}{g +  \widetilde{s} \delta} \\
	&= \frac{g +  \widetilde{s} \delta}{g +  \widetilde{s} \delta} - \frac{(1 - \widetilde{s})g}{g +  \widetilde{s} \delta} \\
	&= \frac{g +  \widetilde{s} \delta - (1 - \widetilde{s})g}{g +  \widetilde{s} \delta}  \\
	&= \frac{g +  \widetilde{s} \delta - g + \widetilde{s}g}{g +  \widetilde{s} \delta}  \\
	&= \frac{\widetilde{s} \delta + \widetilde{s}g}{g +  \widetilde{s} \delta}  \\
	\implies s(g) &= \frac{\widetilde{s} (\delta + g)}{g +  \widetilde{s} \delta}
\end{split}
\end{equation*}
\\
We differentiate $s$ with respect to $g$: 
\begin{equation*}
\begin{split}
	s(g) &= \frac{\widetilde{s}\delta + \widetilde{s}g}{g +  \widetilde{s} \delta}  \Big(= \frac{u}{v}\Big) \text{ where $u' = \widetilde{s}$ and $v' = 1$}  \\
	\implies s'(g) &= \frac{u'v - uv'}{v^2} = \frac{\widetilde{s}(g +  \widetilde{s} \delta) - (\widetilde{s}\delta + \widetilde{s}g)1}{(g +  \widetilde{s} \delta)^2}\\
	&= \frac{\widetilde{s}g +  \widetilde{s}^2 \delta - \widetilde{s}\delta - \widetilde{s}g}{(g +  \widetilde{s} \delta)^2}\\
	&= \frac{- \widetilde{s}\delta + \widetilde{s}^2 \delta}{(g +  \widetilde{s} \delta)^2}\\
	&= \frac{- \widetilde{s}\big(\delta - \widetilde{s} \delta\big)}{(g +  \widetilde{s} \delta)^2}\\
	\implies s'(g) &= \frac{- \widetilde{s}\delta\big(1 - \widetilde{s}\big)}{(g +  \widetilde{s} \delta)^2} < 0\\
\end{split}
\end{equation*}
\\
The savings rate $s$ is decreasing in the rate of technological progress $g$. Furthermore, as $g$ falls to 0, the savings rate increases such that $s = 1$ and the consumption/output ratio $\frac{C(t)}{Y(t)}$ falls to 0.
\\
\\
Piketty goes even further and assumes that $\delta = 0$. This implies that: 
\begin{equation*}
\begin{split}
	\frac{K(t)}{Y(t)} = \frac{\widetilde{s}}{g + \widetilde{s}\delta} = \frac{\widetilde{s}}{g}\\
	\implies \lim_{g \to 0}{\frac{K(t)}{Y(t)} }  = \lim_{g \to 0}{ \frac{\widetilde{s}}{g} } = \infty
\end{split}
\end{equation*}
The capital/output ratio $\frac{K(t)}{Y(t)}$ diverges to infinity as the $g$ falls to 0.









\chapter{Real Business Cycles}

\newcommand{\Lagr}{\mathcal{L}}


\section{Background}

We examine fluctuations around the long-run trends captured by the Solow model. 
\\
\begin{center}
	\includegraphics[scale=0.2]{cycle}
\end{center}
\underline{Lucas' definition of business cycles:} Recurrent fluctuations of output about trend and the comovement among other aggregate time series (consumption, employment, investment).
\\
\\
We will analyse these two characteristics of business cycles: 
\begin{enumerate}
	\item Differences in the volatility of time series.
	\item The comovement of time series (correlations and serial correlations).
\end{enumerate}
These can be summarised as follows for the UK between 1956 and 1990: 

\begin{center}
	\includegraphics[scale=0.4]{table}
\end{center}
We can see that:
\begin{enumerate}
	\item Consumption fluctuates much less than output.
	\item Investment fluctuates much more than output.
	\item The capital stock fluctuates much less than output and is largely uncorrelated with output.
	\item Productivity is slightly pro-cyclical but varies considerably less than output.
	\item Wages vary less than productivity.
	\item Employment fluctuates almost as much as output and hours of work, while average weekly hours fluctuate considerably less.
	\item Government expenditures are essentially uncorrelated with output. Imports are more strongly pro-cyclical than export.
\end{enumerate}
We will build a model which captures these features of the economy.


\section{Model Assumptions}

\begin{enumerate}
	\item Perfect competition, no market imperfections (such as externalities, imperfect information) and fully flexible nominal prices. Money is neutral: monetary policy has no effect. 
	
	\item It is a neoclassical growth model in which saving determines capital accumulation and output. 
		
	\item There are infinite identical households, who live forever, provide labour to firms at a real wage rate $w_t$, and own capital which they lend to firms at a rental rate $r_t$.
	
	\item There are infinite firms who produce a homogeneous good, rent capital and labour (all identical) and are price takers.

	\item Technology shocks and their propagation create fluctuations around the steady state of all macro variables.
	
	\item There is technology growth, but no population growth. The total population is normalised to 1, thus all quantities are per-capita. The model is solved analytically by calibrating the parameters appropriately. 
	
	\item We will use the following symbols: 
	
	\begin{center}
		\includegraphics[scale=0.4]{symbols}
	\end{center}
	
	\item $A_t$ denotes technological progress and grows at an exponential with a stochastic trend: 
	\begin{equation*}
	\begin{split}
		\ln{A_t} = \bar{A} + gt + \widetilde{A}_t
	\end{split}
	\end{equation*}
	where $\widetilde{A}_t$ is a stationaly AR(1) process: 
	\begin{equation*}
	\begin{split}
		 \widetilde{A}_t = \rho_{\widetilde{A}_t}  \widetilde{A}_{t-1} + \epsilon_t 
	\end{split}
	\end{equation*}
with $\epsilon_t \sim \mathcal{N}(0,\,\sigma^{2})$  and $0 < \rho < 1$.
	
	\item Law of motion of capital: 
	\begin{equation*}
	\begin{split}
		 K_{t-1} = I_t + (1 - \delta)K_t \text{ with } \delta \in (0, 1)
	\end{split}
	\end{equation*}
	
	\item Resource constraints: 
	\begin{equation*}
	\begin{split}
		x_t + L_t = 1 \\
		Y_t = C_t + I_t
	\end{split}
	\end{equation*}
	The economy is closed $(NX = 0)$ and there is no government $(G = 0)$.
\\	
	
\end{enumerate}


\section{Households and Firms}

The representative household chooses consumption and savings in all periods to maximise the expected discounted lifetime utility, which is given by: 

	\begin{equation*}
	\begin{split}
		u_t &= E_t \Bigg( \sum_{j=0}^{\infty} \beta_j u(C_{t+j}, x_{t+j})  \Bigg )\\
		&= E_t \bigg( u(C_0, x_0)  + \beta u(C_1, x_1) + \beta^2 u(C_2, x_2) + ... \bigg)\\
		&= E_t \big( u(C_0, x_0) \big) + \beta E_t \big( u(C_1, x_1) \big) + \beta^2 E_t \big( u(C_2, x_2) \big) + ... \\
	\end{split}
	\end{equation*}
\\
with $\beta \in (0, 1)$ and $\frac{\partial u_t}{\partial C_t} > 0$, $\frac{\partial u_t}{\partial x_t} > 0$, u(., .) is strictly concave in both arguments and twice
continuously di�erentiable and satisfying the Inada conditions. The households face the following constraints: 
	\begin{equation*}
	\begin{split}
		C_t + S_t &= r^K_t K_t + w_t L_t \text{  for all } t = 0, 1, 2, ...\\
		x_t + L_t &= 1
	\end{split}
	\end{equation*}
In every period, households can either consume $C_t$ or save $S_t$ the sum of their capital income $r^K_t K_t$ and labour income $w_t L_t$. They are endowed with one unit of time in every period which they can use for either leisure $x_t$ or labour $L_t$.
\\
\\
The representative household chooses capital and labour inputs in all periods to maximise the expected discounted lifetime profits. As there is no intertemporal trade-off, the objective is equivalent to maximising period-by-period real profits. Therefore, the maximisation is done in every period. For example, in period t: 
	\begin{equation*}
	\begin{split}
		\max_{K_t, L_t}  Y_t - wL_t - r^K K_t\\
		\text{where } Y_t = K_t^\alpha\big(A_t L_t\big)^{1-\alpha}
	\end{split}
	\end{equation*}
 The production function is concave in $K_t$ and $A_t L_t$ and twice continuously differentiable in both arguments and satisfying the Inada conditions. As production $Y_t$ is a function of $A_t$ which is stochastic, $Y_t$ is stochastic. Good shocks improve production (good weather makes harvest better) whist bad shocks worsen production (bad weather destroys harvest). 


\section{Market Equilibrium} 
By assumption, markets clear immediately as there are not price rigidities. In the goods market, investment follows the law of motion of capital: 
	\begin{equation*}
	\begin{split}
			 K_{t+1} - K_t &= I_t + \delta K_t \\
		   	 \implies K_{t+1}  &= S_t + (1 - \delta)K_t \\
	\end{split}
	\end{equation*}
In the credit market, we assume that investment equals savings $I_t = S_t$. Therefore, 
	\begin{equation*}
	\begin{split}
			 K_{t+1} &= I_t + (1 - \delta)K_t \\
			 \implies K_{t+1} &= S_t + (1 - \delta)K_t \\
			  \implies S_t &= K_{t+1} - (1 - \delta)K_t \\
	\end{split}
	\end{equation*}

We recall the consumer's budget constraint
	\begin{equation*}
	\begin{split}
		C_t + S_t &= r^K_t K_t + w_t L_t \text{  for all } t = 0, 1, 2, ...\\
	\end{split}
	\end{equation*}
	
in which we plug our expression for $S_t$: 
	\begin{equation*}
	\begin{split}
		C_t + K_{t+1} - (1 - \delta)K_t &= r^K_t K_t + w_t L_t \\
		\implies C_t + K_{t+1} &= r^K_t K_t  + (1 - \delta)K_t  + w_t L_t \\
		\implies C_t + K_{t+1} &= (1 + r^K_t- \delta)K_t  + w_t L_t \\
		\implies C_t + K_{t+1} &= (1 + r_t)K_t  + w_t L_t \text{  where $r_t = r^K_t- \delta$}
	\end{split}
	\end{equation*}
\\
We need to combine: 
\begin{enumerate}
	\item The optimality conditions for equilibrium consumption from the household's maximisation problem.
	\item The optimality conditions for equilibrium prices of production factors (rental and wage rate) from firm's maximisation problem.
\end{enumerate}

This will give us a set of equilibrium conditions which will determine the evolution of all variables over time and react to shocks. 

\section{Intratemporal Problem of Labour}
We consider the following single-period maximisation problem for households, who derive utility from consumption and leisure. 
	\begin{equation*}
	\begin{split}
		\max_{C_t, L_t} \big\{ u(C_t, 1-L_t) \big\} &= \max_{C_t, L_t} \Bigg\{ \frac{C_t^{1-\sigma}}{1-\sigma} + b\frac{(1-L_t)^{1-\sigma}}{1-\sigma}  \Bigg\}	 \text{ with $\sigma, b>0$}\\
		\text{s.t. } C_t &= wL_t
	\end{split}
	\end{equation*}
\\
We write the Lagrangian: 
	\begin{equation*}
	\begin{split}
		\Lagr (C_t, L_t, \lambda) = \frac{C_t^{1-\sigma}}{1-\sigma} + b\frac{(1-L_t)^{1-\sigma}}{1-\sigma} + \lambda (wL_t - C_t)
	\end{split}
	\end{equation*}
Taking first order conditions (FOCs) with respect to $C_t$ and $L_t$: 
	\begin{gather*}
		\frac{\partial \Lagr (C_t, L_t, \lambda)}{\partial C_t} = \frac{1-\sigma}{1-\sigma}C_t^{1-\sigma - 1} - \lambda = 0 \\
		\implies C_t^{-\sigma} = \lambda \\
		\frac{\partial \Lagr (C_t, L_t, \lambda)}{\partial L_t} = \frac{b}{1-\sigma} \frac{\partial (1-L_t)^{1-\sigma}}{\partial L_t} + \lambda w = 0\\
		\implies \frac{b}{1-\sigma} \frac{\partial (1-L_t)^{1-\sigma}}{\partial (1-L_t)} \frac{\partial (1-L_t)}{\partial L_t} + \lambda w = 0\\
		\implies - \frac{b}{1-\sigma} (1-\sigma)(1-L_t)^{1-\sigma -1} + \lambda w = 0\\
		\implies - b (1-L_t)^{-\sigma} + \lambda w = 0\\
		\implies b (1-L_t)^{-\sigma} = \lambda w\\
	\end{gather*}
\\
We combine the two FOCs: 
	\begin{equation*}
	\begin{split}
		\frac{b (1-L_t)^{-\sigma}}{C_t^{-\sigma}} = \frac{\lambda w}{\lambda}\\
		\implies b\bigg(\frac{1-L_t}{C_t}\bigg)^{-\sigma} = w\\
		\implies b\bigg(\frac{C_t}{1-L_t}\bigg)^{\sigma} = w
	\end{split}
	\end{equation*}
We can now solve for $L_t$: 
	\begin{equation*}
	\begin{split}
		b\frac{C_t^{\sigma}}{(1-L_t)^{\sigma}} &= w\\
		\implies C_t^{\sigma} &= \frac{w (1-L_t)^{\sigma} }{b}\\
		\implies {\big(C_t^{\sigma}\big)}^{\frac{1}{\sigma}} &= \bigg( \frac{w (1-L_t)^{\sigma} }{b} \bigg )^{\frac{1}{\sigma}}  \\
		\implies C_t &=  \frac{w^{\frac{1}{\sigma}}  (1-L_t) }{b^{\frac{1}{\sigma}} } \\
	\end{split}
	\end{equation*}
\\
We plug in the constraint $C_t = wL_t$: 
	\begin{equation*}
	\begin{split}
		wL_t &=  \frac{w^{\frac{1}{\sigma}}  (1-L_t) }{b^{\frac{1}{\sigma}} } \\
		\implies L_t b^{\frac{1}{\sigma}} &=  w^{\frac{1}{\sigma}-1} (1-L_t) \\
		\implies L_t b^{\frac{1}{\sigma}} &=  w^{\frac{1}{\sigma}-\frac{\sigma}{\sigma}} (1-L_t) \\
		\implies L_t b^{\frac{1}{\sigma}} &=  w^{\frac{1-\sigma}{\sigma}} (1-L_t) \\
		\implies L_t b^{\frac{1}{\sigma}} &=  w^{\frac{1-\sigma}{\sigma}}- w^{\frac{1-\sigma}{\sigma}} L_t \\
		\implies L_t b^{\frac{1}{\sigma}} + w^{\frac{1-\sigma}{\sigma}} L_t &= w^{\frac{1-\sigma}{\sigma}} \\
		\implies L_t (b^{\frac{1}{\sigma}} + w^{\frac{1-\sigma}{\sigma}}) &= w^{\frac{1-\sigma}{\sigma}} \\
		\implies L_t &= \frac{w^{\frac{1-\sigma}{\sigma}} } {b^{\frac{1}{\sigma}} + w^{\frac{1-\sigma}{\sigma}}}\\
		\implies L_t &= \frac{w^{\frac{1-\sigma}{\sigma}} } {w^{\frac{1-\sigma}{\sigma}}\bigg(1 + \frac{b^{\frac{1}{\sigma}}}{w^{\frac{1-\sigma}{\sigma}}}\bigg)}\\
		\implies L_t &= \frac{1} {1 + b^{\frac{1}{\sigma}} w^{\frac{\sigma - 1}{\sigma}}  }\\
	\end{split}
	\end{equation*}
\\
We now derive the elasticity of labour with respect to wages. 
	\begin{equation*}
	\begin{split}
		\epsilon_{L_t, w} = \frac{\partial L_t}{\partial w}\frac{w}{L_t} &= \frac{\partial \frac{1} {1 + b^{\frac{1}{\sigma}} w^{\frac{\sigma - 1}{\sigma}} }}{\partial w}\frac{w}{L_t} \\
		&= \frac{\partial \frac{1} {1 + b^{\frac{1}{\sigma}} w^{\frac{\sigma - 1}{\sigma}}  }}{\partial \big(1 + b^{\frac{1}{\sigma}} w^{\frac{\sigma - 1}{\sigma}} \big) }\frac{\partial \big(1 + b^{\frac{1}{\sigma}} w^{\frac{\sigma - 1}{\sigma}} \big) }{\partial w} \frac{w}{L_t}\\
		&=  \frac{-1} {\big(1 + b^{\frac{1}{\sigma}} w^{\frac{\sigma - 1}{\sigma}} \big)^2 }  \bigg( b^{\frac{1}{\sigma}} \frac{\sigma - 1}{\sigma} w^{\frac{\sigma - 1}{\sigma}-1} \bigg) \frac{w}{L_t}\\
		&=  \frac{-1} {\big(1 + b^{\frac{1}{\sigma}} w^{\frac{\sigma - 1}{\sigma}} \big)^2 }  \frac{\sigma - 1}{\sigma} b^{\frac{1}{\sigma}} w^{\frac{\sigma - 1}{\sigma}}  \frac{1}{L_t}\\
	\end{split}
	\end{equation*}
\\
Plugging in $L_t = \frac{1} {1 + b^{\frac{1}{\sigma}} w^{\frac{\sigma - 1}{\sigma}}  } \implies \frac{1}{L_t} = 1 + b^{\frac{1}{\sigma}} w^{\frac{\sigma - 1}{\sigma}} $: 
	\begin{equation*}
	\begin{split}
		\epsilon_{L_t, w} &=  \frac{-1} {\big(1 + b^{\frac{1}{\sigma}} w^{\frac{\sigma - 1}{\sigma}} \big)^2 }  \frac{\sigma - 1}{\sigma} b^{\frac{1}{\sigma}} w^{\frac{\sigma - 1}{\sigma}}  \big(1 + b^{\frac{1}{\sigma}} w^{\frac{\sigma - 1}{\sigma}} \big) \\
		&=  \frac{-1} {1 + b^{\frac{1}{\sigma}} w^{\frac{\sigma - 1}{\sigma}}  }  \frac{\sigma - 1}{\sigma} b^{\frac{1}{\sigma}} w^{\frac{\sigma - 1}{\sigma}}  \\
		\implies \epsilon_{L_t, w} &= \frac{- \frac{\sigma - 1}{\sigma} b^{\frac{1}{\sigma}} w^{\frac{\sigma - 1}{\sigma}} } {1 + b^{\frac{1}{\sigma}} w^{\frac{\sigma - 1}{\sigma}}  } 
	\end{split}
	\end{equation*}
\\ 
If $\sigma > 1$ then $\epsilon_{L_t, w} < 0$ and if $\sigma < 1$ then $\epsilon_{L_t, w} > 0$. The elasticity of labour with respect to wages depends on the parameter $\sigma$. When $\sigma = 1$, the labour supply is independent of $w$. This is because individuals have no wealth and with logarithmic utility, income and substitution effects cancel out. We will therefore use logarithmic utility from now on. 


\section{Intertemporal Problem of Labour}
We consider the following two-period maximisation problem for households, who derive utility from consumption and leisure. 
	\begin{equation*}
	\begin{split}
		\max_{C_1, C_2, L_1, L_2} \big\{ u(C_1, 1-L_1) + \beta u(C_2, 1-L_2) \big\}\\
	\end{split}
	\end{equation*}
	\begin{equation*}
	\begin{split}
		= \max_{C_1, C_2, L_1, L_2} \big\{ \ln C_1 + b \ln (1-L_1) + \beta \big( \ln C_2 + b \ln (1-L_2) \big) \big\} \\
	\end{split}
	\end{equation*}
	\begin{equation*}
	\begin{split}
		\text{ s.t. }C_1 + \frac{C_2}{1+r} = w_1 L_1 + \frac{w_2 L_2}{1+r}
	\end{split}
	\end{equation*}
\\
We write the Lagrangian: 
	\begin{equation*}
	\begin{split}
		\Lagr (C_1, C_2, L_1, L_2, \lambda) = \ln C_1 + b \ln (1-L_1) + \beta \big( \ln C_2 + b \ln (1-L_2) \big)\\
		 + \lambda \Big(w_1 L_1 + \frac{w_2 L_2}{1+r} - C_1 - \frac{C_2}{1+r} \Big)
	\end{split}
	\end{equation*}
Taking first order conditions (FOCs) with respect to $L_1$ and $L_2$: 
	\begin{equation*}
	\begin{split}
		\frac{\partial \Lagr (C_1, C_2, L_1, L_2, \lambda)}{\partial L_1} = b \frac{1}{1-L_1}(-1) + \lambda w_1 = 0 \\
		\implies \lambda w_1 = \frac{b}{1-L_1}\\
		\implies 1-L_1 = \frac{b}{\lambda w_1}\\
		\frac{\partial \Lagr (C_1, C_2, L_1, L_2, \lambda)}{\partial L_2} = \beta b \frac{1}{1-L_2}(-1) + \frac{ \lambda w_2}{1+r} = 0 \\		
		\implies \frac{ \lambda w_2}{1+r} = \frac{\beta b}{1-L_2}\\
		\implies 1-L_2 = \frac{\beta b (1+r)}{ \lambda w_2}
	\end{split}
	\end{equation*}
\\ 
Combining these expressions: 
	\begin{equation*}
	\begin{split}
		\frac{1-L_2}{1-L_1} = \frac{\frac{\beta b (1+r)}{ \lambda w_2}}{\frac{b}{\lambda w_1}} = \frac{\frac{\beta (1+r)}{ w_2}}{\frac{1}{ w_1}} = \beta (1+r) \frac{w_1}{w_2}
	\end{split}
	\end{equation*}
\\
This implies as that as the relative wages today $\frac{w_1}{w_2}$ increases, agents will work more today relative to tomorrow. As the interest rate $r$ increases, agents will work more today relative to tomorrow. The equality must hold with $r$ and $\frac{w_1}{w_2}$ as independent, exogenous variables and $L_1, L_2$ as dependant variables. Therefore, shocks can be amplified by household's decision to work and to invest. 
			
			
\section{Optimisation under Uncertainty}
We introduce uncertainty in the two-period maximisation problem. Agents choose to consume or save their period 1 income $Y_1$, and their income in period 2 $Y_2$ is stochastic: $Y_1 \sim \mathcal{N}(\bar{Y}, \sigma_Y)$. The interest rate is also stochastic $r_1 \sim \mathcal{N}(\bar{r}, \sigma_r)$.
	\begin{equation*}
	\begin{split}
		\max_{C_0, B_0, C_1} \big\{ u(C_0) + \beta E_0 \big( u(C_1^H) \big) \big\}\\
	\end{split}
	\end{equation*}
	\begin{equation*}
	\begin{split}
		\text{s.t. } C_0 + B_0 = Y_0 \text{ and  } C_1 = Y_1 + B_0 (1+r_1)
	\end{split}
	\end{equation*}
\\
where $B_0$ denotes savings in period 0. We write the Lagrangian: 
	\begin{gather*}
		\Lagr (C_0, B_0, C_1, \lambda) = u(C_0) + \beta E_0 \big( u(C_1^H) + \lambda_1 (Y_0 - C_0 - B_0 \big)\\
		 + \lambda_2 \big(Y_1 + B_0 (1+r_1) - C_1 \big)
	\end{gather*}
Plugging in $C_1 =  Y_1 + B_0 (1+r_1) = Y_1 + (Y_0 - C_0) (1+r_1)$ :
	\begin{gather*}
		\Lagr (C_0, B_0, C_1, \lambda) = u(C_0) + \beta E_0 \Big( u\big(Y_1 + (Y_0 - C_0) (1+r_1) \big) \Big)\\
		 + \lambda_1 (Y_0 - C_0 - B_0)\\
		 + \lambda_2 \big(Y_1 + B_0 (1+r_1) - C_1 \big)
	\end{gather*}
\\
Taking first order conditions (FOCs) with respect to $C_0$ : 
	\begin{gather*}
		\frac{\partial \Lagr (C_0, B_0, C_1, \lambda)}{\partial C_0} = u'(C_0) + \beta E_0 \Bigg ( \frac{\partial u\big(Y_1 + (Y_0 - C_0) (1+r_1) \big) }{\partial C_0 } \Bigg ) - \lambda_1 = 0\\
		\implies u'(C_0) + \beta E_0 \Bigg ( \frac{\partial u\big(Y_1 + (Y_0 - C_0) (1+r_1) \big) }{\partial Y_1 + (Y_0 - C_0) (1+r_1) } \frac{\partial Y_1 + (Y_0 - C_0) (1+r_1) }{\partial C_0 } \Bigg ) = \lambda_1\\
		\implies u'(C_0) + \beta E_0 \Bigg ( u'\big(Y_1 + (Y_0 - C_0) (1+r_1) \big) \big( -(1+r_1) \big) \Bigg ) = \lambda_1\\
		\implies u'(C_0) = \beta E_0 \Big ( (1+r_1) u'\big(C_1 \big) \Big ) + \lambda_1
	\end{gather*}
\\
We thereby obtain the Euler equation: the marginal utility of consumption in the two periods is equal. If household saves one unit of extra consumption in period 0 at interest rate $r_1$, the reduction in utility in period 0 is exactly offset by increase in expected discounted utility in period 1.
\\
\\
We now assume that agents live for infinite periods and maximise expected lifetime utility as follows: 
	\begin{equation*}
	\begin{split}
		\max_{k_{t+j+1}, L_{t+j}} U &= E_t \Bigg( \sum_{j=0}^{\infty} \beta^j u\Big( C_{t+j}, 1 - L_{t+j} \Big)  \Bigg) \\
	\end{split}
	\end{equation*}
	\begin{equation*}
	\begin{split}
		\text{s. t. } C_t + S_t = r_t K_t + w_t L_t
	\end{split}
	\end{equation*}
\\
We write the law of motion of capital: 
	\begin{equation*}
	\begin{split}
		K_{t+1} = I_t + (1-\delta)K_t\\
		\implies I_t = K_{t+1} - (1-\delta)K_t
	\end{split}
	\end{equation*}
We plug this expression into the budget constraint, assuming $I_t = S_t$ : 
	\begin{equation*}
	\begin{split}
		C_t + K_{t+1} - (1-\delta)K_t = r_t K_t + w_t L_t\\
		\implies C_t = r_t K_t + (1-\delta)K_t + w_t L_t - K_{t+1} \\
		\implies C_t = (1 + r_t -\delta)K_t + w_t L_t - K_{t+1} \\
		\implies C_t + K_{t+1} = (1 + r_t -\delta)K_t + w_t L_t 
	\end{split}
	\end{equation*}
We plug this expression of $C_t$ into the household's maximisation problems: 
	\begin{gather*}
		\max_{k_{t+j+1}, L_{t+j}} U = E_t \Bigg( \sum_{j=0}^{\infty} \beta^j u\Big( (1+r_{t+j}) K_{t+j} + w_{t+j}L_{t+j} - K_{t+j+1}, 1 - L_{t+j} \Big)  \Bigg)\\
		\implies U= E_t \Bigg( u\Big( (1+r_{t}) K_{t} + w_{t}L_{t} - K_{t+1}, 1 - L_{t} \Big) \\
		+ \beta u\Big( (1+r_{t+1}) K_{t+1} + w_{t+1}L_{t+1} - K_{t+2}, 1 - L_{t+1} \Big)  \\
		+ \beta^2 u\Big( (1+r_{t+2}) K_{t+2} + w_{t+2}L_{t+j} - K_{t+3}, 1 - L_{t+2} \Big) + \beta^3...  \Bigg)\\
	\end{gather*}
\\ 

Taking first order conditions (FOCs) with respect to $K_{t+1}$ : 
	\begin{gather*}
		\frac{\partial  U}{\partial  K_{t+1}} = E_t \Bigg( \frac{\partial u\big( (1+r_{t}) K_{t} + w_{t}L_{t} - K_{t+1}, 1 - L_{t} \big)}{\partial K_{t+1}}\\
		 + \beta \frac{\partial u\big( (1+r_{t+1}) K_{t+1} + w_{t+1}L_{t+1} - K_{t+2}, 1 - L_{t+1} \big)}{\partial K_{t+1}}  \Bigg) = 0\\
		 \implies E_t \Bigg( \frac{\partial u\big( (1+r_{t}) K_{t} + w_{t}L_{t} - K_{t+1}, 1 - L_{t} \big)}{\partial (1+r_{t}) K_{t} + w_{t}L_{t} - K_{t+1}, 1 - L_{t}} \\
		 \frac{\partial (1+r_{t}) K_{t} + w_{t}L_{t} - K_{t+1}, 1 - L_{t} \big)}{\partial K_{t+1}}\\
		 + \beta \frac{\partial u\big( (1+r_{t+1}) K_{t+1} + w_{t+1}L_{t+1} - K_{t+2}, 1 - L_{t+1} \big)}{\partial (1+r_{t+1}) K_{t+1} + w_{t+1}L_{t+1} - K_{t+2}, 1 - L_{t+1}} \\
		  \frac{\partial (1+r_{t+1}) K_{t+1} + w_{t+1}L_{t+1} - K_{t+2}, 1 - L_{t+1}}{\partial K_{t+1}}   \Bigg) = 0\\
		 \implies u'_{C_t}(C_t, 1-L_t)(-1) + \beta E_t \Big( u'_{C_{t+1}}(C_{t+1}, 1-L_{t+1})(1+r_{t+1})  \Big ) = 0\\
		\implies  u'_{C_t}(C_t, 1-L_t) = \beta E_t \Big( (1+r_{t+1}) u'_{C_{t+1}}(C_{t+1}, 1-L_{t+1}) \Big ) 
	\end{gather*}
We thereby obtain the Euler equation. We further assume the following utility function: 
	\begin{equation*}
	\begin{split}
		u(C_t, 1-L_t) = \ln C_t + b \ln (1-L_t) \text{  with }b>0
	\end{split}
	\end{equation*}
\\
Differentiating with respect to $C_t$ : 
	\begin{equation*}
	\begin{split}
		u'_{C_t}(C_t, 1-L_t) = \frac{\partial \ln C_t + b \ln (1-L_t)}{\partial C_t} = \frac{1}{C_t}
	\end{split}
	\end{equation*}
Differentiating with respect to $L_t$ : 
	\begin{equation*}
	\begin{split}
		u'_{L_t}(C_t, 1-L_t) &= \frac{\partial \ln C_t + b \ln (1-L_t)}{\partial L_t}\\
		 &= \frac{\partial \ln C_t + b \ln (1-L_t)}{\partial 1-L_t}\frac{\partial 1-L_t}{\partial L_t}\\
		 &= b \frac{1}{1-L_t}(-1)\\
		 \implies u'_{L_t}(C_t, 1-L_t) &=  \frac{-b}{1-L_t}
	\end{split}
	\end{equation*}

We can plug this into the Euler equation: 
	\begin{equation*}
	\begin{split}
		\frac{1}{C_t} = \beta E_t \Big( (1+r_{t+1}) \frac{1}{C_{t+1}} \Big ) 
	\end{split}
	\end{equation*}
If households saves one unit of extra consumption in period t at (expected) interest rate $r_{t+1}$, the reduction in utility in period t is exactly offset by increase in (expected) discounted utility in period t + 1. Therefore there is consumption smoothing (which is a key propagation mechanism in RBC model).
\\
\\
We not the following transversality condition: 
	\begin{equation*}
	\begin{split}
		\lim_{t \to \infty}{\beta^t u'_c(C_t, I_t)K_{t+1}} = 0 
	\end{split}
	\end{equation*}
\\
Taking first order conditions (FOCs) with respect to $L_t$ : 
	\begin{gather*}
		\frac{\partial  U}{\partial  L_t} = \frac{\partial u\big( (1+r_{t}) K_{t} + w_{t}L_{t} - K_{t+1}, 1 - L_{t}}{\partial  L_t} = 0\\
		\implies \frac{\partial u\big( (1+r_{t}) K_{t} + w_{t}L_{t} - K_{t+1}, 1 - L_{t}\big)}{\partial  (1+r_{t}) K_{t} + w_{t}L_{t} - K_{t+1}} \frac{\partial (1+r_{t}) K_{t} + w_{t}L_{t} - K_{t+1} }{\partial  L_t} \\
		+ \frac{\partial u\big( (1+r_{t}) K_{t} + w_{t}L_{t} - K_{t+1}, 1 - L_{t}\big)}{\partial  1-L_t} \frac{\partial 1-L_t }{\partial  L_t}
		= 0\\
		\implies u'_{C_t}(C_t, 1-L_t)w_t + u'_{L_t}(C_t, 1-L_t) = 0\\
		\implies  \frac{1}{C_t}w_t + \frac{-b}{1-L_t} = 0 \\
		\implies \frac{1}{C_t}w_t = \frac{b}{1-L_t} \\
		\implies w_t = \frac{b C_t}{1-L_t}
	\end{gather*}


\section{Firm's Maximisation Problem}

Firms are price takers (as there is perfect competition). They maximise profit by choosing the optimal amount of capital and labour: 
	\begin{equation*}
	\begin{split}
		 \max_{K_t, L_t} \Big( K_t^\alpha \big( A_t L_t \big)^{1-\alpha} - w_t L_t - r_t^K K_t \Big)
	\end{split}
	\end{equation*}
\\
Taking first order conditions (FOCs) with respect to $K_t$ and $L_t$ : 

	\begin{equation*}
	\begin{split}
		 \frac{\partial \Big(K_t^\alpha \big( A_t L_t \big)^{1-\alpha} - w_t L_t - r_t^K K_t \Big)}{\partial K_t} = \alpha K_t^{\alpha-1} \big( A_t L_t \big)^{1-\alpha} - r_t^K = 0\\
		 \implies r_t^K = \alpha \Bigg( \frac{A_t L_t}{K_t} \Bigg)^{1-\alpha} = MPK
	\end{split}
	\end{equation*}
	
	\begin{equation*}
	\begin{split}
		 \frac{\partial \Big(K_t^\alpha \big( A_t L_t \big)^{1-\alpha} - w_t L_t - r_t^K K_t \Big)}{\partial L_t} = (1-\alpha)A_t K_t^\alpha \big( A_t L_t)^{-\alpha} - w_t = 0\\
		 \implies w_t =(1-\alpha)A_t  \Bigg( \frac{K_t}{A_t L_t} \Bigg)^\alpha = MPL
	\end{split}
	\end{equation*}	
		
		
\section{Equilibrium Conditions}


Household:
\begin{enumerate}
  \item $\frac{1}{C_t} = \beta E_t \Big( (1+r_{t+1}^K - \delta) \frac{1}{C_{t+1}} \Big )$
  \item $C_t + K_{t+1} = (1 + r_t -\delta)K_t + w_t L_t $
\end{enumerate}
$ $\\
$ $
Firms:
\begin{enumerate}[resume]
	\item $r_t^K = \alpha \Bigg( \frac{A_t L_t}{K_t} \Bigg)^{1-\alpha} = MPK$	
	\item $w_t =(1-\alpha)A_t  \Bigg( \frac{K_t}{A_t L_t} \Bigg)^\alpha = MPL$
	\item $Y_t = K_t^\alpha \big( A_t L_t \big)^{1-\alpha}$
\end{enumerate}
$ $\\
$ $
Markets: 
\begin{enumerate}[resume]
	\item $Y_t = C_t + I_t$ (the goods market)
	\item $I_t = L_{t+1} - (1-\delta)K_t = S_t$ (the credit market)
	\item $L_t^d = L_t^s$ (the labour market)
\end{enumerate}
$ $\\
$ $
Technological progress (exogenous): 
\begin{enumerate}[resume]
	\item $\ln{A_t} = \bar{A} + gt + \widetilde{A}_t$
	\item $\widetilde{A}_t = \rho_{\widetilde{A}_t}  \widetilde{A}_{t-1} + \epsilon_t$ with $\epsilon_t \sim \mathcal{N}(0,\,\sigma^{2})$ \\
	\\
	The initial conditions are $K_0$, $A_0$ and the terminal condition is: \\
	\\
	$\lim_{t \to \infty}{\beta^t u_c(C_t, I_t)K_{t+1}} = 0$
\end{enumerate}	
$ $
\\
The 7 endogenous variables are $C, r^K, K, w, L, Y, I$ and the 2 exogenous variables are $A_t$ and $\widetilde{A_t}$. We have 8 equations (1) to (9) to characterise equilibrium, therefore one is redundant: 
	\begin{equation*}
	\begin{split}
		 C_t + K_{t+1} &= (1 + r_t -\delta)K_t + w_t L_t \\
		 \implies C_t + K_{t+1} - (1-\delta)K_t &= r_t K_t + w_t L_t \\
		 \implies C_t + S_t &=  r_t K_t + w_t L_t \\
		 \implies C_t + I_t &= MPK_t K_t + MPL_t L_t = Y_t
	\end{split}
	\end{equation*}	
Thus the household's budget constraint is equivalent to the national income identity. We proceed to solve seven the system with 7 equations and with 7 unknowns. We obtain the laws of motion (dynamics) for all endogenous variables. In general we cannot find closed form expressions (as the system is highly non-linear). We do this by solving the approximate (log-linear or linear) system using first order Taylor expansions. 

\section{Special Case Analytical Solution}

We assume log utility ($u(C_t) = \ln C_t$), that agents do not value leisure ($b = 0$) and full depreciation ($\delta = 1$). We write out the equilibrium conditions: 

\begin{enumerate}
  \item $L_t = 1$
  \\
  \\
  Workers work full time as they do not value leisure.
  \\
  \item $C_t + I_t = Y_t \implies C_t + K_{t+1} = K_t^\alpha A_t^{1-\alpha}$ 
  \\
  \\
  As there is full depreciation, $K_{t+1} = I_t$.
 \\
  \item We can re-write the Euler equation:
 	\begin{equation*}
	\begin{split}
		 \frac{1}{C_t} = \beta E_t \Big( (1+r_{t+1}^K - \delta) \frac{1}{C_{t+1}} \Big )
	\end{split}
	\end{equation*}	
	As $ \delta = 1$:
 	\begin{equation*}
	\begin{split}
		 1+r_{t+1}^K - \delta = r_{t+1}^K &= \alpha \bigg( \frac{A_{t+1} L_{t+1}}{K_{t+1}} \bigg)^{1-\alpha}\\
		 &= \alpha \bigg( \frac{A_{t+1}}{K_{t+1}} \bigg)^{1-\alpha}\\
	\end{split}
	\end{equation*}	
  	As $L_{t+1} = 1$. We plug this expression in: 
 	\begin{equation*}
	\begin{split}
		 \frac{1}{C_t} = \beta E_t \Bigg( \alpha \bigg( \frac{A_{t+1}}{K_{t+1}} \bigg)^{1-\alpha} \frac{1}{C_{t+1}} \Bigg )
	\end{split}
	\end{equation*}	
 
\end{enumerate}
$ $
\\
We guess that $C_t = s_t Y_t$ and we will find that $s$ is a constant function of some parameters of the model. This assumption implies that: 
 	\begin{equation*}
	\begin{split}
		K_{t+1} &= Y_t - C_t\\
			&= Y_t - s_t Y_t\\
			&=  (1-s_t)Y_t
	\end{split}
	\end{equation*}	
\\
Furthermore,
 	\begin{equation*}
	\begin{split}
		\bigg( \frac{A_{t+1}}{K_{t+1}} \bigg)^{1-\alpha} &= \frac{A_{t+1}^{1-\alpha}}{K_{t+1}^{1-\alpha}} \\
			&= A_{t+1}^{1-\alpha} K_{t+1}^{-(1-\alpha)}\\
			&= A_{t+1}^{1-\alpha} K_{t+1}^{\alpha-1}\\
			&= \frac{K_{t+1}^\alpha A_{t+1}^{1-\alpha}}{K_{t+1} }\\
			&= \frac{K_{t+1}^\alpha (A_{t+1}L_{t+1})^{1-\alpha}}{K_{t+1} } \text{ since $L_{t+1} = 1$}\\
			\implies \bigg( \frac{A_{t+1}}{K_{t+1}} \bigg)^{1-\alpha} &= \frac{Y_{t+1}}{K_{t+1} } 
	\end{split}
	\end{equation*}	
\\
We plug this expressions into the Euler equation: 
 	\begin{equation*}
	\begin{split}
		 \frac{1}{C_t} = \beta E_t \Bigg( \alpha \frac{Y_{t+1}}{K_{t+1} }  \frac{1}{C_{t+1}} \Bigg )
	\end{split}
	\end{equation*}
\\	
Using the fact that $C_t = s_t Y_t$: 
 	\begin{equation*}
	\begin{split}
		 \frac{1}{s_t Y_t} &= \beta E_t \Bigg( \alpha \frac{Y_{t+1}}{K_{t+1} }  \frac{1}{s_{t+1} Y_{t+1}} \Bigg )\\
		 	&= \beta E_t \Bigg( \alpha \frac{1}{K_{t+1} s_{t+1}}  \Bigg )\\
	\end{split}
	\end{equation*}
Plugging in $K_{t+1} = (1-s_t)Y_t$: 
 	\begin{equation*}
	\begin{split}
		 \frac{1}{s_t Y_t} &= \beta E_t \Bigg( \alpha \frac{1}{(1-s_t)Y_t s_{t+1}}  \Bigg )\\
		 	&= \beta \frac{\alpha}{(1-s_t)Y_t } E_t \Bigg( \frac{1}{s_{t+1}}  \Bigg )\\
		\implies \frac{1}{s_t} &= \beta \frac{\alpha}{1-s_t } E_t \Bigg( \frac{1}{s_{t+1}}  \Bigg )\\
		\implies 1 &= \beta \frac{\alpha}{1-s_t } \text{   as $\frac{1}{s_t} = E_t \Bigg( \frac{1}{s_{t+1}}  \Bigg )$}\\ 
		\implies 1-s_t &= \alpha \beta \\
		\implies \hat{s}_t &= 1- \alpha \beta
	\end{split}
	\end{equation*}
\\
$A_t$ and $K_t$ do not enter this expression so there is a constant $\hat{s}_t$ which satisfies the expression $\hat{s}_t = 1- \alpha \beta$. We plug this expression into our expression for consumption: 
 	\begin{equation*}
	\begin{split}
		 C_t = s_t Y_t &= (1- \alpha \beta) Y_t\\
		  &= (1- \alpha \beta)K_{t}^\alpha A_{t}^{1-\alpha}
	\end{split}
	\end{equation*}
\\
Taking logs on both sides and defining $c_t = \ln(C_t)$, $k_t = \ln(K_t)$, $y_t = \ln(Y_t)$ and $a_t = \ln(A_t)$: 
 	\begin{equation*}
	\begin{split}
		 \ln(C_t) &= \ln\big( (1- \alpha \beta)K_{t}^\alpha A_{t}^{1-\alpha} \big)\\
		 	&= \ln(1- \alpha \beta) + \ln (K_{t}^\alpha) + \ln (A_{t}^{1-\alpha} )\\
			&= \ln(1- \alpha \beta) + \alpha \ln (K_{t}) + (1-\alpha) \ln (A_{t} )\\
			\implies c_t &= \ln(1- \alpha \beta) + \alpha k_t + (1-\alpha) a_t\\
	\end{split}
	\end{equation*}
\\
We re-write our expression for $K_{t+1}$: 
 	\begin{equation*}
	\begin{split}
		 K_{t+1} &= (1-s_t)Y_t\\
		 			&=  \big(1- (1- \alpha \beta)\big)Y_t\\
					&=  \alpha \beta Y_t\\
		\implies K_{t+1} &= \alpha \beta K_{t}^\alpha A_{t}^{1-\alpha}
	\end{split}
	\end{equation*}
\\
Taking logs on both sides:
 	\begin{equation*}
	\begin{split}
		\ln(K_{t+1}) &= \ln\big(\alpha \beta K_{t}^\alpha A_{t}^{1-\alpha}\big)\\
		&= \ln(\alpha \beta) + \ln(K_{t}^\alpha) + \ln( A_{t}^{1-\alpha} )\\
		&= \ln(\alpha \beta) + \alpha \ln(K_{t}) + (1-\alpha) \ln( A_{t} )\\
		\implies \ln(K_{t+1}) &= \ln(\alpha \beta) + \alpha k_t + (1-\alpha) a_t
	\end{split}
	\end{equation*}
\\
We re-write our expression for $Y_{t+1}$: 
 	\begin{equation*}
	\begin{split}
		 Y_{t+1} &= K_{t+1}^\alpha A_{t+1}^{1-\alpha}\\
		  &= (\alpha \beta Y_t)^\alpha A_{t+1}^{1-\alpha}
	\end{split}
	\end{equation*}
\\
Taking logs on both sides:
 	\begin{equation*}
	\begin{split}
		 \ln(Y_{t+1}) &= \ln\big((\alpha \beta Y_t)^\alpha A_{t+1}^{1-\alpha}\big)\\
		 	 &= \ln(\alpha \beta Y_t)^\alpha  + \ln( A_{t+1}^{1-\alpha} )\\
		 	 &= \alpha \ln (\alpha \beta Y_t)  + (1-\alpha) \ln( A_{t+1} )\\
		 	 &= \alpha \big( \ln (\alpha \beta) + \ln( Y_t)\big)  + (1-\alpha) \ln( A_{t+1} )\\
		 	 &= \alpha \ln (\alpha \beta) + \alpha \ln( Y_t)  + (1-\alpha) \ln( A_{t+1} )\\
		 	\implies y_{t+1} &= \alpha \ln (\alpha \beta) + \alpha y_t + (1-\alpha) a_{t+1}\\
	\end{split}
	\end{equation*}
\\

\section{Impulse Response Analysis}

We now express all the variables as deviations from their balanced growth path: 
	\begin{equation*}
	\begin{split}
		 \hat{y}_{t+1} = (1-\alpha) \widetilde{A}_{t+1} + \alpha \hat{y}_t
	\end{split}
	\end{equation*}
\\
where $\hat{y}_{t+1}$ is predicted output in the next period, and depends on current output $\hat{y}_t$ and the technology shock in the next period $\widetilde{A}_{t+1}$. We suppose that technology shocks are normally distributed with mean of zero: 
	\begin{equation*}
	\begin{split}`
		\widetilde{A}_t = \epsilon_t \text{ and } \epsilon_t \sim \mathcal{N}(0,\,\sigma^{2})
	\end{split}
	\end{equation*}
\\
The change in output following a technology shock $\epsilon_{t+1} = 1$ can we written as follows: 
	\begin{equation*}
	\begin{split}
		\frac{\partial \hat{y}_{t+1}}{\partial \epsilon_{t+1}} &= 1 - \alpha\\
		\frac{\partial \hat{y}_{t+2}}{\partial \epsilon_{t+1}} &= \frac{\partial (1-\alpha) \widetilde{\epsilon}_{t+2} + \alpha \hat{y}_{t+1}}{\partial \epsilon_{t+1}} = \alpha \frac{\partial \hat{y}_{t+1}}{\partial \epsilon_{t+1}} = \alpha (1 - \alpha)\\
		\frac{\partial \hat{y}_{t+3}}{\partial \epsilon_{t+1}} &= \frac{\partial (1-\alpha) \widetilde{\epsilon}_{t+3} + \alpha \hat{y}_{t+2}}{\partial \epsilon_{t+1}} = \alpha \frac{\partial \hat{y}_{t+2}}{\partial \epsilon_{t+1}} = \alpha \alpha (1 - \alpha) = \alpha^2 (1 - \alpha) \\
		\frac{\partial \hat{y}_{t+4}}{\partial \epsilon_{t+1}} &= \frac{\partial (1-\alpha) \widetilde{\epsilon}_{t+4} + \alpha \hat{y}_{t+3}}{\partial \epsilon_{t+1}} = \alpha \frac{\partial \hat{y}_{t+3}}{\partial \epsilon_{t+1}} = \alpha \alpha^2 (1 - \alpha) = \alpha^3 (1 - \alpha) \\
		...\\
		\frac{\partial \hat{y}_{t+s}}{\partial \epsilon_{t+1}} &= \frac{\partial (1-\alpha) \widetilde{\epsilon}_{t+s} + \alpha \hat{y}_{t+s-1}}{\partial \epsilon_{t+1}} = \alpha \frac{\partial \hat{y}_{t+s-1}}{\partial \epsilon_{t+1}} = \alpha \alpha^{s-2} (1 - \alpha) = \alpha^{s-1} (1 - \alpha) \\
	\end{split}
	\end{equation*}
\\
Therefore as $s \xrightarrow{} \infty$, $ \hat{y}_{t+s} \xrightarrow{} \bar{y}$. The shock has only a temporary effect on output, before we go back to the balanced growth path. The higher is $\rho_{\widetilde{A}}$ the more persistent the e�ect of the shock will be: 
	\begin{equation*}
	\begin{split}
		\hat{y}_t = (\alpha + \rho_{\widetilde{A}}) \hat{y}_{t-1} - \alpha \rho_{\widetilde{A}} \hat{y}_{t-2} + (1-\alpha)\epsilon_{t+1}
	\end{split}
	\end{equation*}
\\
We can simulate a shock and calculate future values of the variables numerically: 
\begin{center}
	\includegraphics[scale=0.3]{impulse}
\end{center}
The above impulse response functions must be compared with the empirical ones. The empirical impulse responses show two regularities:
\begin{enumerate}
	\item The response of output to a shock is hump-shaped, with a peak occurring after one or two quarters 
	\item The half-life of a shock is of about two and a half years.
\end{enumerate}
$ $
\\
The RBC model that we sketched reproduces these regularities only if technology shocks are very persistent (i.e. high $\rho_{\widetilde{A}}$). Although our simplistic model exaggerates these features, the following limitation holds more generally: RBC models do not generate enough endogenous persistence and rely heavily on technology persistence (which is unexplained). The same result emerges when one computes the summary statistics.
\\
\\
Objections:
\begin{enumerate}
	\item Ignores demand shocks which are known to contribute to output fluctuations.
	\item Needs very persistent exogenous shocks to generate reasonable output fluctuation persistence.
	\item Cannot generate the highly variable employment numbers, unless we assume a very high elasticity of labour supply (which is not true).
	\item There are weak propagation mechanism.
\end{enumerate}
$ $
\\
Contributions:
\begin{enumerate}
	\item We combine growth and business cycles.
	\item This is a hugely important methodological contribution.
\end{enumerate}



\chapter{Consumption}

Consumption models constitute one of the main building blocks of any macroeconomic model. Consumption, savings, and ultimately investments are the main drivers of the business cycle. Investments, and ultimately the capital stock largely determines our standard of living. 
\\
\\
We introduce the Keynesian consumption function: 
	\begin{equation*}
	\begin{split}
		C_t = \alpha + \beta Y_t 
	\end{split}
	\end{equation*}
where $Y_t$ is disposable income and $\beta$ the marginal propensity to consume. Consumption today is only a function of income today: agents are not forward looking. 
\\
\\
Friedman's Permanent Income Hypothesis: 
	\begin{equation*}
	\begin{split}
		C_t = \gamma Y_t^P
	\end{split}
	\end{equation*}
where $Y_t^P$ is permanent income, the expected income in the future. The household's consumption-saving decision is a forward looking and long horizon planning problem. Consumption should only adjust to changes in the permanent component of income. 




\section{Life-Cycle/Permanent Income Hypothesis}

Assumptions: 
\begin{enumerate}
	\item Time is discreet: $t = 0, 1, ..., T$. 
	
	\item Lifetime utility: 
		\begin{equation*}
		\begin{split}
			U = \sum_{t=0}^{T} u(c_t)
		\end{split}
		\end{equation*}
	where $u'(.) < 0$ and $u''(.) > 0$.
	
	\item The budget constraint: 
		\begin{equation*}
		\begin{split}
			c_t + a_{t+1} = a_t + y_t, t = 0, 1, ..., T\\
			a_{T+1} = 0
		\end{split}
		\end{equation*}
	where $a_t$ is savings in period $t$ and $a_0$ is savings in period $0$. We assume that there are no savings after the final period ($a_{T+1} = 0$). We can rewrite the intertemporal budget constraint: 
		\begin{gather*}
			c_0 + a_{1} = a_0 + y_0\\
			c_1 + a_{2} = a_1 + y_1\\
			c_2 + a_{3} = a_2 + y_2\\
			...\\
			c_T + a_{T+1} = a_T + y_T\\
			\\
			\implies \sum_{t=0}^{T} c_t + \sum_{t=1}^{T+1} a_t = \sum_{t=0}^{T} a_t + \sum_{t=0}^{T} y_t\\
			\implies \sum_{t=0}^{T} c_t + \sum_{t=1}^{T} a_t + a_{T+1} = \sum_{t=0}^{T} a_t + \sum_{t=0}^{T} y_t\\
			\implies \sum_{t=0}^{T} c_t + a_{T+1} = \sum_{t=0}^{T} a_t - \sum_{t=1}^{T} a_t  + \sum_{t=0}^{T} y_t\\
			\implies \sum_{t=0}^{T} c_t + a_{T+1} = a_0  + \sum_{t=0}^{T} y_t\\ 
			\implies \sum_{t=0}^{T} c_t = \sum_{t=0}^{T} y_t + a_0 \text{  as $a_{t+1} = 0$}\\
		\end{gather*}		
	where $a_t$ is savings in period $t$. 
	
	\item $r = 0$ and $\beta = 0$. The interest rate is $0$ and there is no time discounting. 
	
	\item Income $y_t$ is exogenous so we have a partial equilibrium.
\end{enumerate}
$ $
\\
The problem of the consumer becomes: 
		\begin{gather*}
			\max_{(C_t)_{t=0}^T} \sum_{t=0}^{T} u(c_t)\\
			s.t.  \sum_{t=0}^{T} c_t = \sum_{t=0}^{T} y_t + a_0
		\end{gather*}
\\
The representative household Lagrangian: 
		\begin{gather*}
			\mathcal{L} = \sum_{t=0}^{T} u(c_t) - \lambda \Bigg( \sum_{t=0}^{T} c_t - \sum_{t=0}^{T} y_t - a_0 \Bigg)
		\end{gather*}
\\
Taking FOCs with respect to $c_t \in {c_0, c_1, c_2, ..., c_T}$: 
		\begin{gather*}
			u'(c_0) = \lambda \\
			u'(c_1) = \lambda \\ 
			...\\
			u'(c_t) = \lambda \\
			...\\
			u'(c_T) = \lambda\\
			\\
			\implies u'(c_t) = u'(c_{t+1}) = \lambda\\ 
			\text{ then } c_0 = c_1 = ... = c_t = ... = c_T
		\end{gather*}
\\
The marginal utility of consumption is equal in every period. This implies that consumption is constant across time. From the budget constraint: 
		\begin{gather*}
			\sum_{t=0}^{T} c_t = \sum_{t=0}^{T} y_t + a_0\\
			\implies \frac{1}{T+1}\sum_{t=0}^{T} c_t = \frac{1}{T+1}\Bigg(\sum_{t=0}^{T} y_t + a_0\Bigg)\\
		\end{gather*}
where $\frac{1}{T+1}\sum_{t=0}^{T} c_t$ is average consumption over the lifetime. As we know that consumption is constant over time, it must be the case that: 
		\begin{gather*}
			c_0 = c_1 = ... = c_t = ... = c_T = \frac{1}{T+1}\Bigg(\sum_{t=0}^{T} y_t + a_0\Bigg)
		\end{gather*}
\\
We can rewrite average income as Friedman's permanent income: 
		\begin{gather*}
			Y^P = \frac{1}{T+1}\sum_{t=0}^{T} c_t 
		\end{gather*}
\\
The difference between an individual's permanent income and her current income is called her transitory income. The MPC out of permanent income is 1:
		\begin{gather*}
			c_t = Y^P 
		\end{gather*}
\\
What is the MPC out of current income? Let's increase an individual's period 0 income by $\epsilon$, such that it equals $y_0 + \epsilon$. We plug this into our pression for $c_0$: 
	\begin{equation*}
	\begin{split}
			c_0 &= \frac{1}{T+1}\Bigg(a_0 + \sum_{t=0}^{T} y_t + \epsilon \Bigg)\\
				&= \frac{1}{T+1} a_0 + \frac{1}{T+1} \sum_{t=0}^{T} y_t + \frac{1}{T+1} \epsilon\\
				&= \frac{1}{T+1} a_0 + Y^P + \frac{1}{T+1} \epsilon
	\end{split}
	\end{equation*}
Therefore the MPC out of $\epsilon$ is $\frac{1}{T+1}$. The individual spreads the transitory income accros his lifetime.
\\ 
\\
What is the marginal propensity to save out of transitory income? 
	\begin{equation*}
	\begin{split}
			s_t &= y_t - c_t\\
				&= y_t - \frac{1}{T+1}\Bigg(a_0 + \sum_{t=0}^{T} y_t \Bigg)\\
				&= y_t - \frac{1}{T+1}a_0 - \frac{1}{T+1}\sum_{t=0}^{T} y_t \\
				&= \big(y_t - Y^P\big) - \frac{1}{T+1}a_0 \\
	\end{split}
	\end{equation*}
where $y_t - Y^P$ denotes transitory income. Saving changes one-to-one with changes in transitory income. Savings are high when current income is high relative to permanent income. 
\\
\\
Keynes (1936): "the amount of aggregate consumption mainly depends on the amount of aggregate income" and argued that this relationship is "a fairly stable function".
	\begin{equation*}
	\begin{split}
			C_t = \alpha + \beta Y_t
	\end{split}
	\end{equation*}
A lot of people have tried to estimate the $\alpha$ and $\beta$ in this function. For a cross-sectional regression, we sample individuals from the population at a point in time and record their income $Y_t$ and consumption $C_t$: 
\begin{center}
	\includegraphics[scale=0.3]{cons}
\end{center}
$ $
$ $\\
The MPC out of income is small: people's consumption is weakly related to their present income. If we aggregate the income and consumption of individuals in every time period and estimate the same model: 
\begin{center}
	\includegraphics[scale=0.3]{time}
\end{center}
$ $
$ $\\
We find that the MPC out of income is close to 1. In any given time period, the average income across individuals is close to the average permanent income (according to the law of large numbers). As small differences in consumption due to transitory income cancel out in aggregate, the average consumption is close to the average permanent income. As the average permanent income grows over time, so does average consumption. 
\\
\\
We now estimate the cross-sectional model across demographic groups which have different permanent incomes: blacks and whites. We find the following: 
\begin{center}
	\includegraphics[scale=0.3]{blacks}
\end{center}
$ $
$ $\\
We find that for any given level of income, whites consume more that blacks. This is because blacks have a lower permanent income than whites, which determines consumption (rather than short run fluctuations in income). 
\\ 
\\
Suppose we wish to estimate: 
	\begin{equation*}
	\begin{split}
			C_i = \alpha + \beta Y_i + \epsilon_i 
	\end{split}
	\end{equation*}
\\
The estimate of $\beta$ is given by: 
	\begin{equation*}
	\begin{split}
			\hat{\beta} &= \frac{\text{Cov}(Y_i, C_i)}{\text{V}(Y_i)} \\ 
				&= \frac{\text{Cov}(Y_i^P + Y_i^T, C_i)}{\text{V}(Y_i^P + Y_i^T)}
	\end{split}
	\end{equation*}
as $Y_i = Y_i^P + Y_i^T$ by the permanent income hypothesis. We further assume that $Y^P$ and $Y^T$ are uncorrelated and that $E(Y^T) = 0$ and recall that, by assumption $C_i = Y^P$ : 
	\begin{equation*}
	\begin{split}
			\text{Cov}(Y_i^P + Y_i^T, C_i) &= E\bigg[\bigg( Y_i^P + Y_i^T - E\big(Y_i^P + Y_i^T\big) \bigg) \bigg( C_i - E\big(C_i\big) \bigg) \bigg] \\
			&= E\bigg[\bigg( Y_i^P + Y_i^T - \bar{Y}_i^P \bigg) \bigg( C_i - \bar{Y}_i^P \bigg) \bigg]
	\end{split}
	\end{equation*}
where $E(Y_i^P) = \bar{Y}_i^P$ is the average permanent income. Therefore: 
	\begin{equation*}
	\begin{split}
			\text{Cov}(Y_i^P + Y_i^T, C_i) &= E\bigg[ (Y_i^P)^2 - Y_i^P \bar{Y}^P + Y_i^T Y_i^P - Y_i^T \bar{Y}_i^P - \bar{Y}_i^P Y_i^P + (\bar{Y}_i^P)^2 \bigg] \\
			&= E\bigg[ (Y_i^P)^2 - 2Y_i^P \bar{Y}^P + (\bar{Y}_i^P)^2 \bigg] + E\bigg[Y_i^T Y_i^P - Y_i^T \bar{Y}_i^P \bigg] \\
			&= E\bigg[ \big(Y_i^P - \bar{Y}_i^P\big)^2 \bigg] + E\bigg[Y_i^T \big(Y_i^P - \bar{Y}_i^P \big) \bigg]\\
			&= E\bigg[ \big(Y_i^P - \bar{Y}_i^P\big)^2 \bigg] + E\big(Y_i^T\big) E\big(Y_i^P - \bar{Y}_i^P \big) + \text{Cov}(Y_i^T, Y_i^P - \hat{Y}_i^P) \\
			&= E\bigg[ \big(Y_i^P - \bar{Y}_i^P\big)^2 \bigg] + E\big(Y_i^T\big) E\big(Y_i^P - \bar{Y}_i^P \big) + \text{Cov}(Y_i^T, Y_i^P) - \text{Cov}(Y_i^T, \hat{Y}_i^P) \\
			&= E\bigg[ \big(Y_i^P - \bar{Y}_i^P\big)^2 \bigg] + E\big(Y_i^T\big) E\big(Y_i^P - \bar{Y}_i^P \big) \text{ (as both covariances equal 0)}\\
			&= E\bigg[ \big(Y_i^P - E(Y_i^P)\big)^2 \bigg] \text{ (as $E\big(Y_i^T\big) = 0$)}\\
		 \text{Cov}(Y_i^P + Y_i^T, C_i) &= \text{Var}(Y_i^P)
	\end{split}
	\end{equation*}
We can plug this into our expression for $\hat{\beta}$: 
	\begin{equation*}
	\begin{split}
			\hat{\beta} &= \frac{\text{Cov}(Y_i^P + Y_i^T, C_i)}{\text{V}(Y_i^P + Y_i^T)} \\
				&= \frac{\text{Var}(Y_i^P)}{\text{V}(Y_i^P) + \text{V}(Y_i^T) + 2\text{Cov}(Y_i^P, Y_i^T)} \\
				&= \frac{\text{Var}(Y_i^P)}{\text{V}(Y_i^P) + \text{V}(Y_i^T)} \text{ (as $\text{Cov}(Y_i^P, Y_i^T)=0$)}  \\
	\end{split}
	\end{equation*}
\\
We now find an expression for $\alpha$. As it is the intercept, we can set all variables equal to their mean and solve for $\alpha$:
	\begin{equation*}
	\begin{split}
			\bar{C}_i &= \alpha + \beta \bar{Y}_i \\
			\implies \alpha &= \bar{C}_i - \beta \bar{Y}_i \\
				&= \bar{Y}_i^P - \beta \big( \bar{Y}_i^P + \bar{Y}_i^T  \big)\\
				&= \bar{Y}_i^P - \beta \bar{Y}_i^P \\
				&= (1-\beta)\bar{Y}_i^P \\
	\end{split}
	\end{equation*}
\\

\begin{enumerate}
	\item From a cross-section perspective, a lot of variation in income is temporary due to job-loss or life-cycle considerations, so $\text{V}(Y_i^T) > \text{V}(Y_i^P)$. Therefore, we would expect to see a small $\beta$ and large $\alpha$.
	
	\item From an aggregate time-series perspective, virtually all variations in income reflects changes in long-run growth, and therefore permanent income. By the law of large numbers, the variance of the aggregate transitory income in every period is very small: $\text{V}(Y_i^T) < \text{V}(Y_i^P)$. Therefore we would expect to see a large $\beta$ and small $\alpha$.

	\item With respect to differences across groups, the variance of $Y_i^P$ and $Y_i^T$ are probably similar. However, $Y_i^P$ is substantially lower for the black population. Thus we would expect a small $\alpha$ for the black population, but a similar $\beta$.
	
\end{enumerate}


\section{Consumption under Uncertainty: Hall's Random Walk Hypothesis}

We now assume that the future exogenous income is uncertain. The household maximises: 
	\begin{gather*}
			E_0\Bigg(\sum_{t=0}^{T} u(c_t) \Bigg)\\
			\text{s.t. }	c_t + a_{t+1} = a_t + y_t\\
			a_{T+1} \geq 0 \text{ with $a_0$ given}
	\end{gather*}
\\
The household maximises the sum of all future consumption at time $t$ : 
	\begin{gather*}
			\max_{(c_j)_{j=t}^T} E_t\Bigg(\sum_{j=t}^{T} u(c_j) \Bigg) = u(c_t) + E_t \big( u(c_{t+1}) \big) + E_t \big( u(c_{t+2}) \big) + ... + E_t \big( u(c_{T}) \big) \\
    	\text{s.t  } \left\{
                \begin{array}{ll}
                  c_t + a_{t+1} = a_t + y_t \\
                  c_{t+1} + a_{t+2} = a_{t+1} + y_{t+1} \\
                  ...\\
                 c_T + a_{T+1} = a_T + y_T
                \end{array}
        \right.\\
        \implies a_{t+1} = a_t + y_t - c_t \\
        \implies c_{t+1} = a_{t+1} + y_{t+1} - a_{t+2} \\
        \implies c_{t+1} = (a_t + y_t - c_t) + y_{t+1} - a_{t+2} 
	\end{gather*}
\\
We can write the Lagrangian: 
	\begin{gather*}
		\Lagr(c_t, c_{t+1}, ..., c_T) = u(c_t) + E_t \big( u(c_{t+1}) \big) + ... + E_t \big( u(c_{T}) \big) + \sum_{j=t}^{T} \lambda_j \big( a_j + y_j + c_j - c_{j+1} \big) \\
		\implies \Lagr(c_t, c_{t+1}, ..., c_T) = u(c_t) + E_t \big( u(a_t + y_t - c_t + y_{t+1} - a_{t+2} ) \big) + ... + E_t \big( u(c_{T}) \big) \\
		+ \sum_{j=t}^{T} \lambda_j \big( a_j + y_j + c_j - c_{j+1} \big) \\
	\end{gather*}
\\
Taking first order conditions with respect to $c_t$ : 
	\begin{equation*}
	\begin{split}
		\frac{\partial \Lagr(c_t, c_{t+1}, ..., c_T)}{c_t} &= u'(c_t) + E_t \bigg( \frac{\partial u(c_{t+1})}{\partial c_{t+1}} \frac{\partial c_{t+1}}{c_t} \bigg) = 0 \\
		&= u'(c_t) + E_t \big( u'(c_{t+1})(-1) \big) = 0\\
		& \implies u'(c_t) = E_t \big( u'(c_{t+1}) \big) 
	\end{split}
	\end{equation*}
\\
We now assume that utility is quadratic: 
	\begin{equation*}
	\begin{split}
		u(c) &= c - \frac{a}{2}c^2\\
		\implies u'(c) &= 1 - \frac{a}{2} 2 c^{2-1} = 1 - ac
	\end{split}
	\end{equation*}
\\
We plug this into our identity: 
	\begin{equation*}
	\begin{split}
		u'(c_t) &= E_t \big( u'(c_{t+1}) \big) \\
		\implies 1 - ac_t &= E_t \big( 1 - ac_{t+1} ) \big) \\
		\implies 1 - ac_t &= 1- E_t \big(ac_{t+1} ) \big) \\
		\implies - ac_t &=- E_t \big(ac_{t+1} ) \big) \\
		\implies ac_t &= E_t \big(ac_{t+1} ) \big) \\
		\implies ac_t &= aE_t \big(c_{t+1} ) \big) \\
		\implies c_t &= E_t \big(c_{t+1} ) \big) \\
	\end{split}
	\end{equation*}
\\
If $c_t = E_t \big(c_{t+1} ) \big)$, then:
	\begin{equation*}
	\begin{split}
		c_{t+1} = c_t + \epsilon_t\\
		\implies c_t \sim \text{AR}(1)
	\end{split}
	\end{equation*}
\\
Furthermore, 
	\begin{equation*}
	\begin{split}
		E_t (\epsilon_t) = 0 \implies E_t (\Delta \epsilon_t) = 0
	\end{split}
	\end{equation*}
\\
We find that consumption is a random walk. Agents will immediately react to any known change in future income in order to smooth consumption. The only reason consumption will not be smooth is if there are sudden changes that were entirely unpredictable in the past.
\\
\\
The intertemporal budget constraint is: 
	\begin{equation*}
	\begin{split}
		\sum_{t=0}^{T} c_t = a_0 + \sum_{t=0}^{T} y_t
	\end{split}
	\end{equation*}
\\
Since this will hold for all the realisations of the process for $y_t$, it must also hold in expectations:
	\begin{gather*}
			E_0\Bigg(\sum_{t=0}^{T} c_t \Bigg) = a_0 + E_0\Bigg(\sum_{t=0}^{T} y_t \Bigg) 
	\end{gather*}
\\
Since $c_t = E_t \big(c_{t+1}\big)$ and $c_{t+1} = E_{t+1}\big(c_{t+2}\big)$, by the law of iterated expectations,
	\begin{gather*}
			c_t = E_t \big(c_{t+1}\big) = E_t \bigg( E_{t+1}\big(c_{t+2}\big) \bigg) = E_{t}\big(c_{t+2}\big) \\
			\implies E_0\Bigg(\sum_{t=0}^{T} c_t \Bigg) = (T + 1) c_0 \\
			\implies (T + 1) c_0 = a_0 + E_0\Bigg(\sum_{t=0}^{T} y_t \Bigg) \\
			\implies c_0 = \frac{1}{T+1}\Bigg(a_0 + E_0\Bigg(\sum_{t=0}^{T} y_t \Bigg)\Bigg)
	\end{gather*}
\\
This is called the certainty equivalence. Present consumption is the expected average income in the future. We notice that in $t=1$: 
	\begin{gather*}
			 c_1 = \frac{1}{T} \Bigg(a_0 + E_1\Bigg(\sum_{t=1}^{T} y_t \Bigg)\Bigg)
	\end{gather*}
\\
Since $a_1 = a_0 + y_0 - c_0$: 
	\begin{gather*}
			 c_1 = \frac{1}{T} \Bigg(a_0 + y_0 - c_0 + E_1\Bigg(\sum_{t=1}^{T} y_t \Bigg)\Bigg)
	\end{gather*}
\\
We also know that :
	\begin{gather*}
			  (T + 1) c_0 = a_0 + E_0\Bigg(\sum_{t=0}^{T} y_t \Bigg) \\
			  \implies a_0 = (T + 1) c_0 - E_0\Bigg(\sum_{t=0}^{T} y_t \Bigg) 
	\end{gather*}
\\
Plugging this in: 
	\begin{gather*}
			 c_1 = \frac{1}{T} \Bigg(y_0 - c_0 + (T + 1) c_0 - E_0\Bigg(\sum_{t=0}^{T} y_t \Bigg) + E_1\Bigg(\sum_{t=1}^{T} y_t \Bigg)\Bigg) \\
			 \implies c_1 = \frac{1}{T} \Bigg(y_0 + T c_0 - E_0\Bigg(\sum_{t=0}^{T} y_t \Bigg) + E_1\Bigg(\sum_{t=1}^{T} y_t \Bigg)\Bigg) \\
			 \implies c_1 = c_0 + \frac{1}{T} \Bigg(y_0 + E_1\Bigg(\sum_{t=1}^{T} y_t \Bigg) - E_0\Bigg(\sum_{t=0}^{T} y_t \Bigg)\Bigg) \\
			\implies c_1 = c_0 + \frac{1}{T} \Bigg(y_0 + E_1\Bigg(\sum_{t=1}^{T} y_t \Bigg) - E_0\Bigg(\sum_{t=1}^{T} y_t \Bigg) - y_0\Bigg) \\
			\implies c_1 = c_0 + \frac{1}{T} \Bigg( E_1\Bigg(\sum_{t=1}^{T} y_t \Bigg) - E_0\Bigg(\sum_{t=1}^{T} y_t \Bigg)\Bigg)
	\end{gather*}
\\
The change in consumption between period 0 and 1 is $\frac{1}{T}$-times the change in expected lifetime earnings between the two periods. Consumption changes in $t = 1$ because new information arrives at $t = 1$. The new information that arrives between $t = 0$ and $t = 1$ is $y_1$.
\\
\\
$y_1$ can be decomposed in two parts:
	\begin{gather*}
			  y_1 = E_0 \big( y_1 \big) + \Big( y_1 - E_0 \big( y_1 \big) \Big)
	\end{gather*}
\\
where $E_0 \big( y_1 \big)$ is forecast of 1 in period 0 and $\Big( y_1 - E_0 \big( y_1 \big) \Big)$ is the shock or new information. Consumption should react only towards the new information. Consumption today equals consumption tomorrow plus the surprises in permanent income!

\section{Campbell and Mankiw Test}

We want to test whether $c_t = E_t\big(c_{t+1}\big)$. We estimate: 
	\begin{gather*}
			  \Delta c_t = X' \beta + \epsilon_t
	\end{gather*}
where $X'$ contains variables which are known in period $t-1$ and before. This includes lags of aggregate consumption, GDP, and other variables. According to our theory, consumption changes only in response to new information. We test that past information has no influence on the change in consumption: 
	\begin{gather*}
			  H_0 : \beta = 0
	\end{gather*}
	\\
Under $H_0$: 
	\begin{gather*}
			  \Delta c_t = c_t - c_{t-1} = \epsilon_t \\
			  \implies c_t = c_{t-1} + \epsilon_t
	\end{gather*}
	\\
and therefore $c_t$ is a unit root. Empirically, we find that the null hypothesis is rejected. When X contains lagged stock returns then $\beta \neq 0$. Therefore we can predict present consumption with past stock prices, which contradicts the idea that $c_t$ is a random walk.
\\
\\
We can improve our test by assuming that a fraction $1-\lambda$ of consumers behave according to the Permanent Income Hypothesis, and therefore:
	\begin{gather*}
			  \Delta c_t = c_t - c_{t-1} = \epsilon_t \\
	\end{gather*}
whilst the other fraction $\lambda$ is more "hand-to-mouth": 
	\begin{gather*}
			  c_t = y_t\\
			  \implies \Delta c_t = \Delta y_t 
	\end{gather*}
\\
On aggregate, we should then observe: 
	\begin{equation*}
	\begin{split}
			  \Delta c_t &= \lambda \Delta y_t + (1-\lambda)\epsilon_t \\
			  &= \lambda \Delta y_t + u_t
	\end{split}
	\end{equation*}
\\
We want to estimate: 
	\begin{equation*}
	\begin{split}
			  \Delta c_t = \lambda \Delta y_t + u_t
	\end{split}
	\end{equation*}
We cannot estimate this regression because it is likely that $\Delta y_t$ and $u_t$ are correlated, since $u_t = (1-\lambda)\epsilon_t$ where $\epsilon_t$ is the shock to consumption for consumers who behave according to the PIH. A large $\epsilon_t$ is likely to be correlated with a shock in income $\Delta y_t$. To see this we can assume that income evolves as follows: 
	\begin{equation*}
	\begin{split}
			 y_t = y_{t-1} + \eta_t
	\end{split}
	\end{equation*}
We recall that : 
	\begin{gather*}
		c_t = c_{t-1} + \frac{1}{T+1-t} \Bigg( E_t\Bigg(\sum_{s=t}^{T} y_t \Bigg) - E_{t-1}\Bigg(\sum_{s=t}^{T} y_t \Bigg)\Bigg) \\
		\implies \Delta c_t = c_t - c_{t-1} = \frac{1}{T+1-t} \Bigg( E_t\Bigg(\sum_{s=t}^{T} y_t \Bigg) - E_{t-1}\Bigg(\sum_{s=t}^{T} y_t \Bigg)\Bigg) = \eta_t
	\end{gather*}
and
	\begin{equation*}
	\begin{split}
			\Delta y_t = y_t -  y_{t-1} = \eta_t
	\end{split}
	\end{equation*}
\\
Under these assumptions,
	\begin{gather*}
 		\Delta c_t = \lambda \Delta y_t + (1-\lambda)\epsilon_t = \eta_t
	\end{gather*}
$\Delta c_t = \Delta y_t$ implies that $\lambda = 1$ even if it (theoretically speaking) could be $0$ (if more consumers are "hand-to-mouth".
\\
\\
When we have a lot of variation in the innovation $\eta_t$, consumption growth will fluctuate for both PIH and myopic consumers. Thus, we will see large fluctuations in consumption growth that are not due to a large share of myopics. The solution is to find an instrument which correlates with $y_t$  and $y_{t-1}$ but which is uncorrelated with $\eta_t$. Any lagged value of income of consumption changes should be able to accomplish that. Therefore we use IV, instead of OLS. Campbell and Mankiw find a value of $\hat{\lambda} = 0.5$ for a broad range of instruments.

\section{Interest Rate and Savings}

Let's reintroduce the discount factor $\beta = \frac{1}{1+\rho}$ and interest rate $r$ on savings in the household optimisation. Households maximise lifetime utility by choosing how much to consume in every period:
	\begin{equation*}
	\begin{split}
			\max_{\big( c_t, a_{t+1} \big)_{t=0}^T} \sum_{t=0}^{T} \frac{1}{(1+p)^t} u(c_t)\\
			\text{s.t. } c_t + a_{t+1} = (1+r)a_t + y_t
	\end{split}
	\end{equation*}
\\
Using $a_{T+1} = 0$, the intertemporal budget constraint is: 
	\begin{equation*}
	\begin{split}
			\sum_{t=0}^{T} \frac{c_t}{(1+r)^t}  = (1+r)a_0 + \sum_{t=0}^{T} \frac{y_t}{(1+r)^t}
	\end{split}
	\end{equation*}
\\
The present value of lifetime consumption is equal to the sum of the present value of lifetime income and of initial savings. We write the Lagrangian:
	\begin{equation*}
	\begin{split}
		\Lagr &= \sum_{t=0}^{T} \frac{1}{(1+\rho)^t} u(c_t) + \sum_{t=0}^{T} \lambda_t \bigg( (1+r)a_t + y_t - c_t - a_{t+1} \bigg) \\
		\Lagr	&= c_0 + \frac{1}{1+\rho}c_1 + \frac{1}{(1+\rho)^2}c_2 + ... + \frac{1}{(1+\rho)^T}c_T + \lambda_0 \big( (1+r)a_0 + y_0 - c_0 - a_1 \big)\\
				& + \lambda_1 \big( (1+r)a_1 + y_1 - c_1 - a_2 \big) + ... + \lambda_T \big( (1+r)a_T + y_T - c_T - a_{T+1} \big)
	\end{split}
	\end{equation*}
\\
Taking FOCs with respect to $c_t$: 
	\begin{equation*}
	\begin{split}
		\frac{\partial \Lagr}{\partial c_t} = \frac{1}{(1+\rho)^t} u'(c_t) + \frac{1}{(1+\rho)^{t+1}} \frac{\partial u(c_{t+1}) }{\partial c_{t+1} } \frac{\partial c_{t+1} }{\partial c_t } = 0\\
	\end{split}
	\end{equation*}
\\
where
	\begin{equation*}
	\begin{split}
		c_{t+1} &= (1+r)a_{t+1} + y_{t+1} - a_{t+2} \\
		\implies c_{t+1} &= (1+r)\big( (1+r)a_t + y_t - c_t \big) + y_{t+1} - a_{t+2}\\
		\\
		\implies \frac{\partial c_{t+1} }{\partial c_t } &= -(1+r)
	\end{split}
	\end{equation*}
\\
Plugging this in we obtain the Euler equation: 
	\begin{equation*}
	\begin{split}
		\frac{\partial \Lagr}{\partial c_t} = \frac{1}{(1+\rho)^t} u'(c_t) + \frac{1}{(1+\rho)^{t+1}} & u'(c_{t+1})\big(-(1+r) \big) = 0 \\
		\implies \frac{1}{(1+\rho)^t} u'(c_t) &= \frac{1+r}{(1+\rho)^{t+1}} u'(c_{t+1}) \\
		\implies u'(c_t) &= \frac{1+r}{1+\rho} u'(c_{t+1})
	\end{split}
	\end{equation*}
\\
We further assume a Constant Relative Risk Aversion (CRRA) utility function: 
	\begin{equation*}
	\begin{split}
		u(c) &= \frac{c^{1-\sigma}}{1-\sigma}\\
		\implies u'(c) &= \frac{(1-\sigma)c^{1-\sigma - 1}}{1-\sigma} = c^{-\sigma}
	\end{split}
	\end{equation*}
\\
The Euler equation becomes: 
	\begin{equation*}
	\begin{split}
		c_t^{-\sigma} &= \frac{1+r}{1+\rho} c_{t+1}^{-\sigma} \\
		\implies \Big(c_t^{-\sigma} \Big)^{\frac{-1}{\sigma}} &= \Big(\frac{1+r}{1+\rho} c_{t+1}^{-\sigma} \Big)^{\frac{-1}{\sigma}} \\
		\implies c_t &= \Big( \frac{1+r}{1+\rho} \Big)^{\frac{-1}{\sigma}} c_{t+1}
	\end{split}
	\end{equation*}
\\
The intertemporal elasticity of substitution (IES) is defined as: 
	\begin{equation*}
	\begin{split}
		IES = \abs{\frac{\partial \ln\frac{c_{t+1}}{c_t}}{\partial \ln\frac{u'(c_{t+1})}{u'(c_t)}}} = \abs{\frac{\partial \ln\frac{c_{t+1}}{c_t}}{\partial \ln\frac{1+\rho}{1+r}}}
	\end{split}
	\end{equation*}
\\
We know that:
	\begin{equation*}
	\begin{split}
		u'(c_t) &= \frac{1+r}{1+\rho} u'(c_{t+1}) \\
		\implies \frac{u'(c_{t+1})}{u'(c_t)} &= \frac{1+\rho}{1+r}
	\end{split}
	\end{equation*}
\\
and that: 
	\begin{equation*}
	\begin{split}
		c_t = \Big( \frac{1+r}{1+\rho} \Big)^{\frac{-1}{\sigma}} c_{t+1} \\
		\implies \frac{c_{t}}{c_{t+1}} = \Big( \frac{1+r}{1+\rho} \Big)^{\frac{-1}{\sigma}} \\
		\implies \frac{c_{t+1}}{c_{t}} = \Big( \frac{1+r}{1+\rho} \Big)^{\frac{1}{\sigma}}
	\end{split}
	\end{equation*}
\\
We plug this into the formula for the $IES$: 
	\begin{equation*}
	\begin{split}
		IES &= \abs{\frac{\partial \ln  \Big( \frac{1+r}{1+\rho} \Big)^{\frac{1}{\sigma}} }{\partial \ln\frac{1+\rho}{1+r}}} \\
			&= \abs{\frac{\partial \frac{1}{\sigma} \ln \frac{1+r}{1+\rho} }{\partial \ln\frac{1+\rho}{1+r}}} \\
			&= \abs{\frac{1}{\sigma}  \frac{\partial \ln \frac{1+r}{1+\rho} }{\partial \ln\frac{1+\rho}{1+r}}} \\
			&= \frac{1}{\sigma}  \abs{ \frac{\partial \ln \frac{1+r}{1+\rho} }{\partial \ln\frac{1+\rho}{1+r}}} \\
			&= \frac{1}{\sigma}  \abs{ \frac{\partial -\ln \frac{1+\rho}{1+r} }{\partial \ln\frac{1+\rho}{1+r}}} \\
			&= \frac{1}{\sigma}  \abs{ -1 } \\
		\implies IES &=  \frac{1}{\sigma} 
	\end{split}
	\end{equation*}
\\
the elasticity of consumption growth with respect to the growth in marginal utility. From the Euler equation we know that changes in $r$ will cause changes in marginal utility growth $\frac{u'(c_{t+1})}{u'(c_t)}$. To what extent will this yield changes in consumption growth? From the $IES$ we know that a $1\%$ increase in r leads to a $\frac{1}{\sigma}\%$ increase in consumption. This, only if $\sigma$ is very low (high IES) will the interest rate have a strong effect on consumption and savings decisions.\\
\\
	\begin{equation*}
	\begin{split}
		u'(c_t) &= \frac{1+r}{1+\rho} u'(c_{t+1}) \\
	\end{split}
	\end{equation*}
\\
An increase in $r$ will unambiguously increase the ratio $\frac{c_{t+1}}{c_t}$. However, it may actually lower the level of both $c_t$ and $c_{t+1}$, as $r$ always brings a positive substitution effect. The income effect may however be both positive and negative depending if you're a borrower or a saver in period $t$.
\\
\\
We can analyse this graphically for a borrower. We start by illustrating the clockwise rotation of the budget constraint around $(Y_1, Y_2)$ as a result of an increase in $r$:
\begin{center}
	\includegraphics[scale=0.4]{increaseRgraph1}
\end{center}
We can see that the the initial consumption bundle $(C_1, C_2)$ is no longer feasible. We can illustrate the substitution effect by plotting the parallel to the new budget constraint which is tangent to the initial indifference curve: 
\begin{center}
	\includegraphics[scale=0.4]{increaseRgraph2}
\end{center}
We can see that as the interest rate $r$ increases, borrowers will substitute away from consumption in period 1 towards consumption in period 2. The opportunity cost of borrowing in period 1 in terms of period 2 consumption increases. We now examine the income effect by plotting the indifference curve which is tangent to the new budget constraint: 
\begin{center}
	\includegraphics[scale=0.4]{increaseRgraph3}
\end{center}
As the agent is a borrower, an increase in the interest rate leaves him worse off, so he consumes less in both periods.
\\
\\
We can summarise the effect of an increase in $r$ for borrowers, savers and consumers who consume their endowment as follows: 
\begin{center}
	\includegraphics[scale=0.4]{increaseRtable}
\end{center}


\section{Consumption CAPM}
We now assume now that the consumer can invest in a risky asset in period $t$ with random return in the next period $r_{t+1}$. The 2-period utility maximisation problem is as follows.
	\begin{gather*}
			\max_{c_t, c_{t+1}} u(c_t) + E_t \big( u(c_{t+1}) \big) \\
			\text{s.t. } c_{t+1} = y_{t+1} + (1+r_{t+1})(y_t + c_t)
	\end{gather*}
We write the Lagrangian: 
	\begin{equation*}
	\begin{split}
		\Lagr(c_t, c_{t+1}, \lambda) &= u(c_t) + E_t \big( u(c_{t+1}) \big) + \lambda \big( c_{t+1} - y_{t+1} - (1+r_{t+1})(y_t + c_t) \big)  \\
	\end{split}
	\end{equation*}
\\
We take FOCs with respect to $c_t$ and $c_{t+1}$: 
	\begin{equation*}
	\begin{split}
		\frac{\partial \Lagr(c_t, c_{t+1}, \lambda)}{\partial c_t} = u'(c_t) + \lambda(1+r_{t+1}) &= 0 \\
		\implies u'(c_t) &= -\lambda(1+r_{t+1}) \\
		\implies \frac{1}{1+r_{t+1}}u'(c_t) &= -\lambda \\
	\end{split}
	\end{equation*}
\\
	\begin{equation*}
	\begin{split}
		\frac{\partial \Lagr(c_t, c_{t+1}, \lambda)}{\partial c_{t+1}} = \frac{1}{1+\rho}E_t\Big( u'(c_{t+1}) \Big) + \lambda &= 0  \\
		\implies \frac{1}{1+\rho}E_t\Big( u'(c_{t+1}) \Big) &= -\lambda
	\end{split}
	\end{equation*}
\\
Combining FOCs: 
	\begin{equation*}
	\begin{split}
		\frac{1}{1+r_{t+1}}u'(c_t) = \frac{1}{1+\rho}E_t\Big( u'(c_{t+1}) \Big) \\ 
		\implies  u'(c_t) = \frac{1}{1+\rho}E_t\Big( (1+r_{t+1})u'(c_{t+1}) \Big) \\ 
	\end{split}
	\end{equation*}
\\
Using the fact that $E[XY]=E[X]E[Y]+Cov(X,Y)$ we have: 
	\begin{equation*}
	\begin{split}
		u'(c_t) = \frac{1}{1+\rho}\bigg( E_t( 1+r_{t+1} ) E_t\big( u'(c_{t+1}) \big) + \text{Cov}\big((1+r_{t+1}), u'(c_{t+1}) \big) \bigg) \\ 
	\end{split}
	\end{equation*}
\\
We now assume for simplicity the following utility function:
	\begin{equation*}
	\begin{split}
		u(c) &= c - \frac{a}{2}c^2 \\
		\implies u'(c) &= 1 - \frac{a}{2}\frac{1}{2}c \\
			&= 1 - ac 
	\end{split}
	\end{equation*}
\\
Therefore, 
	\begin{equation*}
	\begin{split}
		u'(c_t) &= \frac{1}{1+\rho}\bigg( E_t( 1+r_{t+1} ) E_t\big( u'(c_{t+1}) \big) + \text{Cov}\big((1+r_{t+1}), 1 - ac_{t+1}  \big) \bigg) \\ 
			&= \frac{1}{1+\rho}\bigg( E_t( 1+r_{t+1} ) E_t\big( u'(c_{t+1}) \big) -\text{Cov}\big((1+r_{t+1}), ac_{t+1}  \big) \bigg) \\ 
	\end{split}
	\end{equation*}
\\
A ceteris paribus higher covariance term yields a lower expected return because the asset works like insurance. Return is high when the marginal utility of returns are high. A lower covariance term yields a higher expected return.
\\
\\
We now assume now that the consumer can invest in 2 assets; a risky asset with random return $r$ and a risk-less asset with known return $r_f$. The 2-period utility maximisation problem is now as follows. 
	\begin{gather*}
			\max_{c_t, c_{t+1}} u(c_t) + E_t \big( u(c_{t+1}) \big) \\
			\text{s.t. } c_t + a_{t+1} + b_{t+1} = (1 + r_t) a_t + (1 + r_t^f) b_t + y_t
	\end{gather*}
\\
We have separate FOCs for the two assets. The FOC for the risky asset is the same as what we derived previously: 
	\begin{equation*}
	\begin{split}
		u'(c_t) = \frac{1}{1+\rho}E_t\Big( (1+r_{t+1})u'(c_{t+1}) \Big) \\ 
	\end{split}
	\end{equation*}
\\
For the risk-free asset, the FOC is the same with the exception than we can take $r_t^f$ out of the expectation operator, as it is not a random variable. 
	\begin{equation*}
	\begin{split}
		u'(c_t) = \frac{1+r_{t+1}^f}{1+\rho}E_t\Big( u'(c_{t+1}) \Big) \\ 
	\end{split}
	\end{equation*}
\\
We now solve for the expected returns in both cases. For the risky asset: 
	\begin{gather*}
		u'(c_t) = \frac{1}{1+\rho}\bigg( E_t( 1+r_{t+1} ) E_t\big( u'(c_{t+1}) \big) + \text{Cov}\big((1+r_{t+1}), u'(c_{t+1}) \big) \bigg) \\ 
		 (1+\rho) u'(c_t) =  E_t( 1+r_{t+1} ) E_t\big( u'(c_{t+1}) \big) + \text{Cov}\big((1+r_{t+1}), u'(c_{t+1}) \big)  \\ 
		 (1+\rho) u'(c_t) -  \text{Cov}\big((1+r_{t+1}), u'(c_{t+1}) \big) =  E_t( 1+r_{t+1} ) E_t\big( u'(c_{t+1}) \big)  \\ 
		 \implies E_t( 1+r_{t+1} ) = \frac{(1+\rho) u'(c_t) -  \text{Cov}\big((1+r_{t+1}), u'(c_{t+1}) \big)}{E_t\big( u'(c_{t+1}) \big)}
	\end{gather*}
\\
For the risk-free asset: 
	\begin{equation*}
	\begin{split}
		u'(c_t) = \frac{1+r_{t+1}^f}{1+\rho}E_t\big( u'(c_{t+1}) \big) \\ 
		 \frac{1+r_{t+1}^f}{1+\rho} = \frac{u'(c_t)}{E_t\big( u'(c_{t+1}) \big)} \\ 
		 \implies 1+r_{t+1}^f = \frac{(1+\rho) u'(c_t)}{E_t\big( u'(c_{t+1}) \big)} \\ 
	\end{split}
	\end{equation*}
\\
If we take the difference between the expected returns for the two assets, we obtain the expected excess return of the risky asset:
	\begin{equation*}
	\begin{split}
		 E_t(r_{t+1}) - r_{t+1}^f &= E_t( 1+r_{t+1} ) - (1+r_{t+1}^f )\\
		&= \frac{(1+\rho) u'(c_t) -  \text{Cov}\big((1+r_{t+1}), u'(c_{t+1}) \big)}{E_t\big( u'(c_{t+1}) \big)} - \frac{(1+\rho) u'(c_t)}{E_t\big( u'(c_{t+1}) \big)} \\
		&= \frac{(1+\rho) u'(c_t) -  \text{Cov}\big((1+r_{t+1}), u'(c_{t+1}) \big) - (1+\rho) u'(c_t)}{E_t\big( u'(c_{t+1}) \big)} \\
		&= -\frac{\text{Cov}\big((1+r_{t+1}), u'(c_{t+1}) \big)}{E_t\big( u'(c_{t+1}) \big)} \\
	\end{split}
	\end{equation*}
\\
If we again assume that $u(c) = c - \frac{a}{2}c^2$ and $u'(c) = 1 - ac$, then: 
	\begin{equation*}
	\begin{split}
		 E_t(r_{t+1}) - r_{t+1}^f &= -\frac{\text{Cov}\big((1+r_{t+1}), 1 - ac_{t+1} \big)}{E_t\big( u'(c_{t+1}) \big)} \\
			&= -\frac{(-a)\text{Cov}\big((1+r_{t+1}), c_{t+1} \big)}{E_t\big( u'(c_{t+1}) \big)} \\
			&= \frac{a\text{Cov}\big((1+r_{t+1}), c_{t+1} \big)}{E_t\big( u'(c_{t+1}) \big)} \\
	\end{split}
	\end{equation*}
\\
The consumption CAPM predicts that the excess return of a risky asset is proportional to the covariance between the return and consumption. An asset with a positive covariance with consumption will have a higher return than the risk free rate:  
	\begin{equation*}
	\begin{split}
		 \text{If Cov}\big((1+r_{t+1}), c_{t+1} \big) > 0 &\implies \frac{a\text{Cov}\big((1+r_{t+1}), c_{t+1} \big)}{E_t\big( u'(c_{t+1}) \big)} > 0 \\
		 & \implies  E_t(r_{t+1}) - r_{t+1}^f > 0 \\
		 & \implies E_t(r_{t+1}) >  r_{t+1}^f 
	\end{split}
	\end{equation*}
\\
An asset with a negative covariance with consumption will have a lower return than the risk free rate:  
	\begin{equation*}
	\begin{split}
		 \text{If Cov}\big((1+r_{t+1}), c_{t+1} \big) < 0 &\implies \frac{a\text{Cov}\big((1+r_{t+1}), c_{t+1} \big)}{E_t\big( u'(c_{t+1}) \big)} < 0 \\
		 & \implies  E_t(r_{t+1}) - r_{t+1}^f < 0 \\
		 & \implies E_t(r_{t+1}) <  r_{t+1}^f 
	\end{split}
	\end{equation*}
\\
\\

\section{Application: the Equity Premium Puzzle}

We assume that the consumer can invest in a risky asset $i$ with random return $r_{t+1}^i$. The FOC for this asset is: 
	\begin{equation*}
	\begin{split}
		u'(c_t) = \frac{1}{1+\rho}E_t\Big( (1+r_{t+1}^i)u'(c_{t+1}) \Big) \\ 
	\end{split}
	\end{equation*}
\\
We now assume the following utility function:
	\begin{equation*}
	\begin{split}
		u(c) &= \frac{c^{1-\sigma}}{1-\sigma} \\
		\implies u'(c) &= \frac{1}{{1-\sigma}}({1-\sigma})c^{1-\sigma - 1} \\
			&= c^{-\sigma}
	\end{split}
	\end{equation*}
\\
Therefore, 
	\begin{equation*}
	\begin{split}
		c_t^{-\sigma} = \frac{1}{1+\rho}E_t\Big( (1+r_{t+1}^i)c_{t+1}^{-\sigma} \Big) \\ 
		1+\rho = \frac{1}{c_t^{-\sigma}} E_t\Big( (1+r_{t+1}^i)c_{t+1}^{-\sigma} \Big) \\ 
		\implies 1+\rho = E_t\bigg( (1+r_{t+1}^i)\frac{c_{t+1}^{-\sigma}}{c_t^{-\sigma}} \bigg) \\ 
	\end{split}
	\end{equation*}
\\
We define $g_{t+1} = \frac{c_{t+1}-c_t}{c_t}$ as the growth rate of consumption. Thereby:
	\begin{equation*}
	\begin{split}
		1 + g_{t+1} &= 1 + \frac{c_{t+1}-c_t}{c_t} \\
			&= \frac{c_t}{c_t} + \frac{c_{t+1}-c_t}{c_t} \\
			&= \frac{c_t + c_{t+1}-c_t}{c_t} \\
			&= \frac{c_{t+1}}{c_t} \\
		\implies (1 + g_{t+1})^{-\sigma} &= \bigg(\frac{c_{t+1}}{c_t}\bigg)^{-\sigma}
	\end{split}
	\end{equation*}
\\
We plug this in and drop the time subsripts: 
	\begin{equation*}
	\begin{split}
		1+\rho = E\bigg( (1+r^i) (1 + g)^{-\sigma}  \bigg) \\ 
	\end{split}
	\end{equation*}
\\
We define $f(r^i, g) = (1+r^i) (1 + g)^{-\sigma}$ and take a second-order Taylor expansion around $r^{i*} = g^* = 0$:
	\begin{equation*}
	\begin{split}
		f(r^i, g) \approx \text{ } & f(r^{i*}, g^*) \\
			& + \frac{\partial f(r^{i*}, g^*)}{\partial r^{i*}}(r^i - r^{i*}) + \frac{\partial f(r^{i*}, g^*)}{\partial g^*}(g - g^{*}) \\
			& + \frac{1}{2}\frac{\partial f(r^{i*}, g^*)}{\partial r^{i*} \partial r^{i*}}(r^i - r^{i*})^2 +  \frac{1}{2}\frac{\partial f(r^{i*}, g^*)}{\partial g^* \partial g^*} (g - g^{*})^2 \\
			& + \frac{\partial f(r^{i*}, g^*)}{\partial r^{i*} \partial g^*} (r^i - r^{i*})(g - g^{*}) \\
			\\
			\approx \text{ } & (1+r^{i*}) (1 + g^*)^{-\sigma} \\
			& + \frac{\partial (1+r^{i*}) (1 + g^*)^{-\sigma}}{\partial r^{i*}}(r^i - r^{i*}) + \frac{\partial (1+r^{i*}) (1 + g^*)^{-\sigma}}{\partial g^*}(g - g^{*}) \\
			& + \frac{1}{2}\frac{\partial (1+r^{i*}) (1 + g^*)^{-\sigma}}{\partial r^{i*} \partial r^{i*}}(r^i - r^{i*})^2 +  \frac{1}{2}\frac{\partial (1+r^{i*}) (1 + g^*)^{-\sigma}}{\partial g^* \partial g^*} (g - g^{*})^2 \\
			& + \frac{\partial (1+r^{i*}) (1 + g^*)^{-\sigma}}{\partial r^{i*} \partial g^*} (r^i - r^{i*})(g - g^{*}) \\
			\\
			\approx \text{ } & (1+r^{i*}) (1 + g^*)^{-\sigma} \\
			& + (1 + g^*)^{-\sigma}(r^i - r^{i*}) + (1+r^{i*})\frac{\partial (1 + g^*)^{-\sigma}}{\partial (1 + g^*)}\frac{\partial 1 + g^*}{\partial g^*}(g - g^{*}) \\
			& + \frac{1}{2}\frac{\partial (1 + g^*)^{-\sigma}}{\partial r^{i*}}(r^i - r^{i*})^2 +  \frac{1}{2}\frac{\partial (1+r^{i*}) (1 + g^*)^{-\sigma}}{\partial g^* \partial g^*} (g - g^{*})^2 \\
			& + \frac{\partial (1 + g^*)^{-\sigma}}{\partial g^*} (r^i - r^{i*})(g - g^{*}) \\
			\\
			\approx \text{ } & (1+r^{i*}) (1 + g^*)^{-\sigma} \\
			& + (1 + g^*)^{-\sigma}(r^i - r^{i*}) + (1+r^{i*})(-\sigma)(1+g^*)^{-\sigma-1}(g - g^{*}) \\
			& + \frac{1}{2}(0)(r^i - r^{i*})^2 +  \frac{1}{2}\frac{\partial -\sigma(1+r^{i*})(1+g^*)^{-\sigma-1}}{\partial g^*} (g - g^{*})^2 \\
			& + \frac{\partial (1 + g^*)^{-\sigma}}{\partial (1 + g^*)} \frac{\partial 1 + g^*}{\partial g^*} (r^i - r^{i*})(g - g^{*}) \\
			\\
			\approx \text{ } & (1+r^{i*}) (1 + g^*)^{-\sigma} \\
			& + (1 + g^*)^{-\sigma}(r^i - r^{i*}) -\sigma(1+r^{i*})(1+g^*)^{-\sigma-1}(g - g^{*}) \\
			& + \frac{1}{2}(-\sigma(1+r^{i*}))\frac{\partial (1+g^*)^{-\sigma-1}}{\partial 1+g^*}\frac{\partial 1+g^*}{g^*} (g - g^{*})^2 \\
			& -\sigma (1 + g^*)^{-\sigma-1} (r^i - r^{i*})(g - g^{*}) \\
	\end{split}
	\end{equation*}
\\
	\begin{equation*}
	\begin{split}
			f(r^i, g) \approx \text{ } & (1+r^{i*}) (1 + g^*)^{-\sigma} \\
			& + (1 + g^*)^{-\sigma}(r^i - r^{i*}) -\sigma(1+r^{i*})(1+g^*)^{-\sigma-1}(g - g^{*}) \\
			& + \frac{1}{2}(-\sigma(1+r^{i*}))(-\sigma-1)(1+g^*)^{-\sigma-2} (g - g^{*})^2 \\
			& -\sigma (1 + g^*)^{-\sigma-1} (r^i - r^{i*})(g - g^{*}) \\
	\end{split}
	\end{equation*}
\\
We plug in $r^{i*} = g^* = 0$:
	\begin{equation*}
	\begin{split}
			f(r^i, g) \approx \text{ } & (1+0) (1 + 0)^{-\sigma} \\
			& + (1 + 0)^{-\sigma}(r^i - 0) -\sigma(1+0)(1+0)^{-\sigma-1}(g - 0) \\
			& + \frac{1}{2}(-\sigma(1+0))(-\sigma-1)(1+0)^{-\sigma-2} (g - 0)^2 \\
			& -\sigma (1 + 0)^{-\sigma-1} (r^i - 0)(g - 0) \\
			\\
			\approx \text{ } & 1 \\
			& + r^i -\sigma g \\
			& + \frac{1}{2}(-\sigma)(-\sigma-1) g^2 \\
			& -\sigma r^i g  \\
	\end{split}
	\end{equation*}
\\
Therefore, we can conclude that:
	\begin{equation*}
	\begin{split}
		 (1+r^i) (1 + g)^{-\sigma} & \approx 1 + r^i - \sigma g - \sigma g r^i + \frac{1}{2} \sigma(\sigma + 1)g^2
	\end{split}
	\end{equation*}
\\
As a consequence, we can use this in our previous expression:
	\begin{equation*}
	\begin{split}
		1+\rho &= E\bigg( (1+r^i) (1 + g)^{-\sigma}  \bigg) \\ 
		1+\rho &= E\bigg( 1 + r^i - \sigma g - \sigma g r^i + \frac{1}{2} \sigma(\sigma + 1)g^2  \bigg) \\ 
			&= 1 + E\big( r^i \big) - \sigma E\big( g \big) - \sigma E\big( g r^i \big)  + \frac{1}{2} \sigma(\sigma + 1)E\big( g^2 \big) \\
	\end{split}
	\end{equation*}
\\
Using the fact that $E[XY ] = E[X]E[Y ] + Cov(X, Y)$: 
	\begin{equation*}
	\begin{split}
		1+\rho &= 1 + E\big( r^i \big) - \sigma E\big( g \big) - \sigma \bigg( E\big( g \big)E\big( r^i \big) + \text{Cov}(g, r^i) \bigg)  + \frac{1}{2} \sigma(\sigma + 1) \bigg( E\big( g \big)E\big( g \big) + \text{Cov}(g, g) \bigg) \\
		\rho &= E\big( r^i \big) - \sigma E\big( g \big) - \sigma \bigg( E\big( g \big)E\big( r^i \big) + \text{Cov}(g, r^i) \bigg)  + \frac{1}{2} \sigma(\sigma + 1) \bigg( E\big( g \big)^2 + \text{Var}(g) \bigg) \\
	\end{split}
	\end{equation*}
\\
On short horizons $E(r^i)E(g)$ and $E(g)^2$ are almost zero (in continuous time they are zero), so: 
	\begin{equation*}
	\begin{split}
		\rho &= E\big( r^i \big) - \sigma 0 - \sigma \bigg( 0 + \text{Cov}(g, r^i) \bigg)  + \frac{1}{2} \sigma(\sigma + 1) \bigg( 0 + \text{Var}(g) \bigg) \\
		\rho &= E\big( r^i \big) - \sigma  \text{Cov}(g, r^i)   + \frac{1}{2} \sigma(\sigma + 1)  \text{Var}(g)  \\
		\implies E\big( r^i \big) &= \rho + \sigma  \text{Cov}(g, r^i) - \frac{1}{2} \sigma(\sigma + 1)  \text{Var}(g) 
	\end{split}
	\end{equation*}
\\
Comparing asset $i$ with asset $j$ we get:
	\begin{equation*}
	\begin{split}
		E\big( r^i \big) - E\big( r^j \big) &= \rho + \sigma  \text{Cov}(g, r^i) - \frac{1}{2} \sigma(\sigma + 1)  \text{Var}(g) - \bigg(  \rho + \sigma  \text{Cov}(g, r^j) - \frac{1}{2} \sigma(\sigma + 1)  \text{Var}(g)  \bigg) \\
		&= \sigma  \text{Cov}(g, r^i) - \sigma  \text{Cov}(g, r^j) \\
		&= \sigma \Big( \text{Cov}(g, r^i) - \text{Cov}(g, r^j) \Big) \\
		&= \sigma  \text{Cov}(g, r^i - r^j) 
	\end{split}
	\end{equation*}
\\
We can consider asset $i$ as being stocks and asset $j$ being bonds. Empirically, we find that: 
\begin{enumerate}
	\item The equity premium $E\big( r^i \big) - E\big( r^j \big)$ is $6\%$.
	
	\item Std$(g)=0.036$ and Std$(r^i - r^j)=0.167$, and Corr$(r^i - r^j, g)=0.4$. Therefore,
	\begin{equation*}
	\begin{split}
		\text{Cov}(r^i - r^j, g) &= \text{Corr}(r^i - r^j, g) \text{Std}(g) \text{Std}(r^i - r^j) \\
			&= 0.4 *0.036 * 0.167 \\
			&= 0.0024\\
	\end{split}
	\end{equation*}
\\
\end{enumerate}

We can use these facts to find the implied risk aversion parameter $\sigma$.
	\begin{equation*}
	\begin{split}
		E\big( r^i \big) - E\big( r^j \big) &= \sigma  \text{Cov}(g, r^i - r^j) \\
		 \sigma &= \frac{E\big( r^i \big) - E\big( r^j \big) }{\text{Cov}(g, r^i - r^j)} \\
		\implies \hat{\sigma} &= \frac{0.06}{0.0024} =25
	\end{split}
	\end{equation*}
\\
This is a very large risk aversion parameter, as most microeconomic evidence suggests that $1 < \sigma <  6$.













\chapter{Investment}
Investment is much smaller fraction of GDP than consumption. However, investments are much more volatile than consumption. Recall: std(C) = 0.008 and std(I) = 0.065 (8 times as volatile).\\ 
\\
In deviation from trend, investment is three times as volatile as GDP, and consumption is half as volatile as GDP. Investments account for $80\%$ of the fall in GDP during recessions.\\
\\
In the long run, investment determines the capital stock, which in turn determines our standard of living.\\

\begin{center}
	\includegraphics[scale=0.4]{invVol}
\end{center}
$ $ 
$ $
\section{Theories of Investment}
Samuelson (1939) investment follows:
\begin{equation*}
\begin{split}
	I_t = \alpha Y_t - \gamma r_t
\end{split}
\end{equation*}
This is how we see the 'investment function' in the IS-LM: I(Y,r), which yields a standard multiplier/accelerator effect. As usual when you postulate reasonable relations you will get a good fit from a statistical point of view. However the amount of endogeneity in the above equation is tremendous. Doesn't $Y_t$ depend on $I_t$? What about $r_t$, is that independent of $I_t$?\\
\\
Hall and Jorgenson (1967) propose:
\begin{equation*}
\begin{split}
	\max_{K,L} F(K,L)-wL-r^KK 
\end{split}
\end{equation*} 
Under standard assumptions ($F'_K > 0$ and $F''_{KK} < 0$), investment demand is decreasing in the interest-rate:
\begin{equation*}
\begin{split}
	 \frac{\partial F(K,L)-wL-r^KK }{\partial K} = F'_K - r^K = 0 \implies F'_K = r^K \\
	 \implies \frac{\partial F'_K }{\partial K} =  \frac{\partial r^K }{\partial K} \\
	 F''_{KK} =  \frac{\partial r^K }{\partial K} \\
	 \implies  \frac{1}{F''_{KK}} =  \frac{\partial K }{\partial r^K} <0 
\end{split}
\end{equation*} 
\\
By arbitrage: $r=r^K-\delta$: firm's demand for capital is smooth in exogenous parameters such as productivity, interest rate, depreciation or taxes. There are no worries about the future, since firms can adjust the capital stock to any change in the environment. Firms' problem is static. Capital can be adjusted in a single period up to the point where: $F_K = r^K$.\\
\\
Tobin's Q (1969):
\begin{equation*}
\begin{split}
	 Q = \frac{\text{Market value of installed capital}}{\text{Replacement cost of capital}}
\end{split}
\end{equation*} 
\\ 
This is the ratio of the cost of buying the company to the cost of buying all of the company's capital. Companies should follow the investment rule:
\[ 
\bigg \{
  \begin{tabular}{ccc}
   \text{if } $Q>1$ \text{ then } $i_t > 0$ \\
   \text{if } $Q>1$ \text{ then } $i_t > 0$ \\
   \end{tabular}
\]
In the first case the new capital will increase the market value more that what it costs, so there will be value creation. In the second case the sale of existing capital will lower market value by less that the cash raised, which would create value. \\
Firms which have a value greater than what it would cost to reproduce their capital should be growing, while firms which are not worth what it would cost to reproduce them should be shrinking.\\
\\

\section{A Neoclassical Model with Adjustment Costs}
A large number of firms are born in period 0 and live for perpetuity.\\
\\
In each period they receive revenue $A_t F(K_t)$ with $F'>0$ and $F'' <0$.\\
\\
A firm may invest in each period $I_t$.\\
\\
The price of output goods and investment goods are normalized to 1, so investing $I_t$ also costs $I_t$. \\ 
\\
There is also a cost associated with adjusting the capital stock: $C(I)$ where $C''(.) > 0$ and $C(0) = C'(0) = 0$. For simplicity let's assume
\begin{equation*}
\begin{split}
	C(I) = \chi \frac{I^2}{2}
\end{split}
\end{equation*} 
with $\chi >0$.
\\
There is no capital depreciation.\\
\\
\section{The Firm's Problem}

A firm investing $I_t$ derives momentary profits: 
\begin{equation*}
\begin{split}
	A_t F(K_t) - I_t - C(I_t)	
\end{split}
\end{equation*} 
\\
There is no depreciation, so 
\begin{equation*}
\begin{split}
	K_{t+1} = K_t + I_t
\end{split}
\end{equation*} 
\\
The firm discounts the future by the interest rate (opportunity cost): $r$.\\
\\
The firms' problem is then given by
\begin{equation*}
\begin{split}
	\pi_0 = \max_{ ( I_t, K_{t+1} )_{t = 0}^\infty} \sum_{t=0}^{\infty} \frac{1}{(1+r)^t} \bigg( A_t F(K_t) - I_t - \chi \frac{I_t^2}{2}   \bigg)
\end{split}
\end{equation*} 
\begin{equation*}
\begin{split}
	\text{s.t. } K_{t+1} = K_t + I_t 
\end{split}
\end{equation*} 
\\
The Lagrangian is given by: 
\begin{equation*}
\begin{split}
	\Lagr = \sum_{t=s}^{\infty} \Bigg( \frac{1}{(1+r)^{s-t} } \bigg( A_t F(K_t) - I_t - \chi \frac{I_t^2}{2} \bigg) + \lambda_t \Big( K_t + I_t -  K_{t+1} \Big) \Bigg)
\end{split}
\end{equation*} 
\\
What is the interpretation of $\lambda_t$ ? It's the time-zero value to the firm of marginally relaxing the $t=0$ budget constraint in period t. As a consequence $q_t = \lambda_t (1 + r)^t$ is the time-t value of marginally relaxing the budget constraint in period $t$.\\
\\
Let's rewrite the Lagrangian in terms of $q_t$ instead of $\lambda_t$:
\begin{equation*}
\begin{split}
	\Lagr &= \sum_{t=s}^{\infty} \Bigg( \frac{1}{(1+r)^{s-t} } \bigg( A_t F(K_t) - I_t - \chi \frac{I_t^2}{2} \bigg) + \frac{q_t}{(1 + r)^{t-s}} \Big( K_t + I_t -  K_{t+1} \Big) \Bigg) \\
		&= \sum_{t=s}^{\infty}  \frac{1}{(1+r)^{s-t} } \Bigg( A_t F(K_t) - I_t - \chi \frac{I_t^2}{2}  + q_t \Big( K_t + I_t -  K_{t+1} \Big) \Bigg)
\end{split}
\end{equation*} 
\\
We maximise with respect to $I_t$ and $K_{t+1}$ by taking first order conditions. For $I_t$:
\begin{equation*}
\begin{split}
	\frac{\partial \Lagr}{\partial I_t} = \frac{1}{(1+r)^t} (-1 - C'(I_t) + q_t) &= 0 \\
	-1 - C'(I_t) + q_t &= 0 \\
	q_t &= 1 + C'(I_t) \\
	\implies q_t &= 1 + \chi I_t
\end{split}
\end{equation*} 
\\
For $K_{t+1}$: 
\begin{equation*}
\begin{split}
	\frac{\partial \Lagr}{\partial K_{t+1}} &= \frac{1}{1+r} \Big( A_{t+1} F'(K_{t+1}) + q_{t+1} \Big) - q_t = 0 \\
	 q_t &= \frac{1}{1+r} \Big( A_{t+1} F'(K_{t+1}) + q_{t+1} \Big) \\
	\implies q_t &= \frac{A_{t+1} F'(K_{t+1})}{1+r} + \frac{q_{t+1}}{1+r} \\
\end{split}
\end{equation*} 
\\

\section{Transversality Condition}

We assume that $A_{t+1} F'(K_{t+1}) = \bar{\pi}$ and hence: 
\begin{equation*}
\begin{split}
	q_t &= \frac{\bar{\pi}}{1+r} + \frac{q_{t+1}}{1+r} \\
	(1+r) q_t &= \bar{\pi} + q_{t+1} \\ 
	\implies q_{t+1} &= (1+r) q_t - \bar{\pi}
\end{split}
\end{equation*} 
\\
We solve this first order difference equation.\\
\\
The complementary solution is: 
\begin{equation*}
\begin{split}
	q_{t}^c &= c(1+r)^t 
\end{split}
\end{equation*} 
as $q_t$ is growing exponentially at rate $1+r$. We now find a solution to the non-homogenous part $- \bar{\pi}$, which is a constant. Therefore, we plug in $A$ for $q_t$ and $q_{t+1}$ in the equation and solve for $A$: 
\begin{equation*}
\begin{split}
	A &= (1+r) A - \bar{\pi} \\
	 \bar{\pi} &= (1+r) A - A \\
	 \bar{\pi} &= rA \\
	 \implies A &= \frac{\bar{\pi}}{r}
\end{split}
\end{equation*} 
We plug this into the system: 
\begin{equation*}
\begin{split}
	q_{t} &= c(1+r)^t + \frac{\bar{\pi}}{r}
\end{split}
\end{equation*} 
To pin down the constant $c$, we use the initial condition at $t=0$: 
\begin{equation*}
\begin{split}
	q_{0} &= c(1+r)^0 + \frac{\bar{\pi}}{r} \\
	q_{0} &= c + \frac{\bar{\pi}}{r} \\
	\implies q_{0} - \frac{\bar{\pi}}{r} &= c
\end{split}
\end{equation*} 
We plug this in to find the general solution to the system: 
\begin{equation*}
\begin{split}
	q_{t} &= \big(q_{0} - \frac{\bar{\pi}}{r} \big)(1+r)^t + \frac{\bar{\pi}}{r}
\end{split}
\end{equation*} 
\\
We know that the particular solution uniquely determines a sequence of $q_t$ and we know that this satisfies the FOC. However we know that there are other sequences that satisfy them as long as it follows:
\begin{equation*}
\begin{split}
	q_{t+1} =(1+r)q_t - \bar{\pi}
\end{split}
\end{equation*} 
There are infinitely many sequences!\\
\\
We iterate forward the difference equation: 
\begin{equation*}
\begin{split}
	q_t &= \frac{\bar{\pi}}{1+r} + \frac{q_{t+1}}{1+r} \text{ and } q_{t+1} = \frac{\bar{\pi}}{1+r} + \frac{q_{t+2}}{1+r} \\
	\\
	\implies q_t &= \frac{\bar{\pi}}{1+r} + \frac{\frac{\bar{\pi}}{1+r} + \frac{q_{t+2}}{1+r}}{1+r} \\
	 &= \frac{\bar{\pi}}{1+r} + \frac{\bar{\pi}}{(1+r)^2} + \frac{q_{t+2}}{(1+r)^2} \\
	 &= \frac{\bar{\pi}}{1+r} + \frac{\bar{\pi}}{(1+r)^2} + \frac{ \frac{\bar{\pi}}{1+r} + \frac{q_{t+3}}{1+r}  }{(1+r)^2} \\
	 &= \frac{\bar{\pi}}{1+r} + \frac{\bar{\pi}}{(1+r)^2} +  \frac{\bar{\pi}}{(1+r)^3} + \frac{q_{t+3}}{(1+r)^3}  \\
	 \\
	 \implies q_t &= \sum_{s=1}^{n} \frac{\bar{\pi}}{(1+r)^s} + \frac{q_{t+n}}{(1+r)^n}
\end{split}
\end{equation*} 
As $n \to \infty$,
\begin{equation*}
\begin{split}
	q_t &= \sum_{s=1}^{\infty} \frac{\bar{\pi}}{(1+r)^s} + \lim_{n \to \infty} \frac{q_{t+n}}{(1+r)^n} \\
\end{split}
\end{equation*} 
\\
We obtain the following transversality condition: 
\begin{equation*}
\begin{split}
	\lim_{n \to \infty} \frac{q_{t+n}}{(1+r)^n} &= q_t -  \sum_{s=1}^{\infty} \frac{\bar{\pi}}{(1+r)^s} \\
	&= q_t -  \bar{\pi} \sum_{s=1}^{\infty} \Big(\frac{1}{1+r} \Big)^2 \\
	&= q_t -  \bar{\pi} \frac{\frac{1}{1+r}}{1 - \frac{1}{1+r}} \\
	&= q_t -  \bar{\pi} \frac{\frac{1}{1+r}}{\frac{1+r -1}{1+r}} \\
	&= q_t -  \bar{\pi} \frac{\frac{1}{1+r}}{\frac{r}{1+r}} \\
	&= q_t -  \bar{\pi} \frac{1}{r} \\
	\implies \lim_{n \to \infty} \frac{q_{t+n}}{(1+r)^n} &= q_t - \frac{\bar{\pi}}{r} 
\end{split}
\end{equation*} 
\\
This implies that if $q_t \neq \frac{\bar{\pi}}{r}$ then the sequence $(q_t)$ is explosive.\\
\\
Even if $A_{t+1} F(K_{t+1})$ is not a constant, if the NPV of the MPK is high today not because the current MPK is high, but because future will be and so on. But if the high MPK never materializes this is not optimal behaviour.\\
\\
The tranversality condition rules out sequences which satisfies the FOC, but they are suboptimal as they are explosive. Given $\bar{\pi}$ the transversality condition singles out a unique $q_t, q_{t+1}, ...$.\\
\\

\section{The System}
 
The system that we need to analyse is made up of the two FOCs:
\begin{equation*}
\begin{split}
	q_t &= 1 + \chi I_t \\
	q_t &= \frac{A_{t+1} F'(K_{t+1})}{1+r} + \frac{q_{t+1}}{1+r} \\
\end{split}
\end{equation*} 
\\
We recall that $I_t = K_{t+1} - K_t = \Delta K_{t+1}$, hence plugging in: 
\begin{equation*}
\begin{split}
	q_t &= 1 + \chi \Delta K_{t+1} \\
	q_t &= \frac{A_{t+1} F'(K_{t+1})}{1+r} + \frac{q_{t+1}}{1+r} \\
\end{split}
\end{equation*} 
\\
We can solve the two equations for $\Delta K_{t+1}$:
\begin{equation*}
\begin{split}
	\chi \Delta K_{t+1} &= q_t -1  \\
	\Delta K_{t+1} &= \frac{1}{\chi} \big( q_t -1 \big)  \\
\end{split}
\end{equation*} 
\\
and $\Delta q_{t+1}$: 
\begin{equation*}
\begin{split}
	q_t &= \frac{A_{t+1} F'(K_{t+1})}{1+r} + \frac{q_{t+1}}{1+r} \\
	 \frac{q_{t+1}}{1+r} &= q_t - \frac{A_{t+1} F'(K_{t+1})}{1+r} \\
	 q_{t+1} &= (1+r) \Big( q_t - \frac{A_{t+1} F'(K_{t+1})}{1+r} \Big) \\
	 q_{t+1} &= (1+r)q_t - A_{t+1} F'(K_{t+1}) \\
	 q_{t+1} &= q_t + r q_t - A_{t+1} F'(K_{t+1}) \\
	 q_{t+1} - q_t &= r q_t - A_{t+1} F'(K_{t+1}) \\
	 \Delta q_{t+1} &= r q_t - A_{t+1} F'(K_{t+1}) \\
\end{split}
\end{equation*} 
As $\Delta K_{t+1} = \frac{1}{\chi} \big( q_t -1 \big)$, $K_{t+1} = K_t + \frac{1}{\chi} \big( q_t -1 \big)$, which we can plug in:
\begin{equation*}
\begin{split}
	\Delta q_{t+1} &= r q_t - A_{t+1} F'\Big(K_t + \frac{1}{\chi} \big( q_t -1 \big)\Big) \\
\end{split}
\end{equation*} 
\\
\\
Let's assume $A_{t+1} = A$. We have two first order difference equations: 
\begin{equation*}
\begin{split}
	\Delta K_{t+1} &= \frac{1}{\chi} \big( q_t -1 \big)  \\
	\Delta q_{t+1} &= r q_t - A_{t+1} F'\Big(K_t + \frac{1}{\chi} \big( q_t -1 \big)\Big) \\
\end{split}
\end{equation*} 
\\
We have a steady state with $\Delta K_{t+1} = 0$ and $\Delta q_{t+1} = 0$ for: 
\begin{equation*}
\begin{split}
	\bar{q} = 1
\end{split}
\end{equation*} 
\begin{equation*}
\begin{split}
	r = A F'(\bar{K})
\end{split}
\end{equation*} 
\\
\\

\section{Towards the Phase Diagram}

We start with the first equation: 
\begin{equation*}
\begin{split}
	\Delta K_{t+1} &= \frac{1}{\chi} \big( q_t -1 \big)  \\
\end{split}
\end{equation*} 
\\
There are 3 possible cases: 
\[ 
\Bigg \{
  \begin{tabular}{ccc}
   \text{if } $\Delta K_{t+1} = 0$ \text{ then } $q_t = 1$ \\
   \text{if } $q_t > 1$ \text{ then } $\Delta K_{t+1} > 0$ \\
   \text{if } $q_t < 1$ \text{ then } $\Delta K_{t+1} < 0$ \\
   \end{tabular}
\]
We can represent this in a phase diagram as follows: 
\begin{center}
	\includegraphics[scale=0.4]{phaseInv}
\end{center}
For the second equation: 
\begin{equation*}
\begin{split}
	\Delta q_{t+1} &= r q_t - A_{t+1} F'\Big(K_t + \frac{1}{\chi} \big( q_t -1 \big)\Big) \\
\end{split}
\end{equation*} 
\\
We plug in $\Delta q_{t+1} = 0$ and solve for $q_t$: 
\begin{equation*}
\begin{split}
	0 &= r q_t - A F'\Big(K_t + \frac{1}{\chi} \big( q_t - 1 \big)\Big) \\ 
	- r q_t &= - A F'\Big(K_t + \frac{1}{\chi} \big( q_t - 1 \big)\Big) \\ 
	 r q_t &= A F'\Big(K_t + \frac{1}{\chi} \big( q_t -1 \big)\Big) \\ 
	 q_t &= \frac{ A F'\Big(K_t + \frac{1}{\chi} \big( q_t -1 \big)\Big) }{r}\\ 
\end{split}
\end{equation*} 
\\
Furthermore, 
\begin{equation*}
\begin{split}
	 \frac{\partial q_t}{\partial K_t} &= \frac{1}{r} A F''\Big(K_t + \frac{1}{\chi} \big( q_t -1 \big)\Big) \frac{\partial K_t + \frac{1}{\chi} \big( q_t -1 \big)}{K_t}\\ 
	 \frac{\partial q_t}{\partial K_t} &= \frac{1}{r} A F''\Big(K_t + \frac{1}{\chi} \big( q_t -1 \big)\Big) \bigg( 1 + \frac{1}{\chi} \frac{\partial q_t}{\partial K_t} \bigg) \\
	 \frac{\partial q_t}{\partial K_t} &= \frac{1}{r} A F''\Big(K_t + \frac{1}{\chi} \big( q_t -1 \big)\Big) + \frac{1}{\chi} \frac{\partial q_t}{\partial K_t} \frac{1}{r} A F''\Big(K_t + \frac{1}{\chi} \big( q_t -1 \big)\Big)\\  
	 \frac{\partial q_t}{\partial K_t} - &  \frac{1}{\chi} \frac{\partial q_t}{\partial K_t} \frac{1}{r} A F''\Big(K_t + \frac{1}{\chi} \big( q_t -1 \big)\Big) = \frac{1}{r} A F''\Big(K_t + \frac{1}{\chi} \big( q_t -1 \big)\Big) \\  
	  \frac{\partial q_t}{\partial K_t} & \bigg(1 -  \frac{A}{\chi r} F''\Big(K_t + \frac{1}{\chi} \big( q_t -1 \big)\Big) \bigg) = \frac{1}{r} A F''\Big(K_t + \frac{1}{\chi} \big( q_t -1 \big)\Big) \\  
	  \frac{\partial q_t}{\partial K_t} & = \frac{A F''\Big(K_t + \frac{1}{\chi} \big( q_t -1 \big)\Big)}{r \bigg(1 -  \frac{A}{\chi r} F''\Big(K_t + \frac{1}{\chi} \big( q_t -1 \big)\Big) \bigg)}  \\  
	 \implies \frac{\partial q_t}{\partial K_t} & = \frac{A F''\Big(K_t + \frac{1}{\chi} \big( q_t -1 \big)\Big)}{r -  \frac{A}{\chi} F''\Big(K_t + \frac{1}{\chi} \big( q_t -1 \big)\Big) } <0 \\  
\end{split}
\end{equation*} 
Therefore as $K_t$ increases, $q_t$ decreases.\\
\\
For a given $q_t$
\[ 
\Bigg \{
  \begin{tabular}{ccc}
   \text{if } $K_t \uparrow \implies F' \downarrow$ \text{ then } $\Delta q_{t+1} > 0$ \\
   \text{if } $K_t \downarrow \implies F' \uparrow$ \text{ then } $\Delta q_{t+1} < 0$ \\
   \end{tabular}
\]
We can represent $\Delta q_{t+1}$ as a function of $K_t$ in a phase diagram as follows: 
\begin{center}
	\includegraphics[scale=0.4]{phaseInv2}
\end{center}
We find that the steady state satisfies the two conditions $\Delta K_{t+1} = 0$ and $\Delta q_{t+1} = 0$ in $E$ at $\bar{q} = 1$ and $r = A F'(\bar{K})$: 
\begin{center}
	\includegraphics[scale=0.4]{phaseInv3}
\end{center}
$ $
$ $

\section{The Saddle Path}

We have identified areas in the $(q,K)$ plane at which the economy has a chance not to explode. Still, does that mean that for each K there are many possible q (as long as they belong to the ?stable?areas)? \\
\\
No, for the same reason as there only existed one q which rendered an non-explosive sequence $(q)$, there exist only one $q$ for each $K$ such that the sequences $(q,K)$ are non-explosive. \\
\\
This relationship between $K$ and $q$, is called a saddle-path:
\begin{center}
	\includegraphics[scale=0.4]{phaseS}
\end{center}
$ $
\\
 Another way of seeing this is that each firm observes a $(K_t)$ sequence. Given this, to be nonexplosive, $q_t$ must satisfy: 
 \begin{equation*}
\begin{split}
	q_t &= \sum_{s=1}^{\infty} \frac{ A_{t+s}F'(K_{t+s})}{(1+r)^s} + \lim_{n \to \infty} \frac{q_{t+n}}{(1+r)^s} \\
\end{split}
\end{equation*} 
for every $t$. This sequence of $q_t$ maps out investments through
1+??Kt+1 =qt
 \begin{equation*}
\begin{split}
	1+ \chi \Delta K_{t+1} = q_t
\end{split}
\end{equation*} 
Which determines the sequence $(Kt)$.\\
\\
The existence of a unique saddle-path shows that there exists only one such sequence of $K_t$. So what guarantees a unique saddle-path in this model? \\
\\
The fact that $F'(K)$ is decreasing in $K$. If, for instance, individuals would get the idea that $(Kt)$ is an increasing sequence. The $(q)$ sequence will be decreasing over time, so will investments, and therefore also capital.\\
\\
\\

\section{A Permanent Productivity Shock}
Suppose that in period $t$, the economy is suddenly hit by a positive and permanent productivity shock. We will think of this as $AF'(K) \uparrow$ for all levels of $K$.\\
\\ 
As a consequence, the $\Delta q_{t+1}$ line will shift up. An immediate jump in $q$, and therefore also in investments. As the capital stock increases, $q$ declines and investments becomes lower.\\
\\
Eventually the economy reaches the new steady state, with more capital and more output.
\begin{center}
	\includegraphics[scale=0.4]{phaseShiftInv}
\end{center}
$ $
\\

Both $q$ and investments jump immediately, but not as much as before. At time $T$ the economy must be back on the 'old' saddle-path. Notice too that the path $A$ to $B$ crosses the $\Delta K = 0$ line.\\
\\
Thus, there are disinvestments even before we?re back to old productivity. That is not very strange; when productivity is high, we wish to have a lot of capital. When it is low, we wish to have less. Due to adjustment costs, we plan ahead.\\
\\
\begin{center}
	\includegraphics[scale=0.4]{phaseShiftInv2}
\end{center}
$ $
\\

\section{Tax Credit}
Another interesting situation arises when we consider the effect of taxes. In particular, a common argument heard is that recessions may be fought by temporary tax credits to investments.\\
\\
Thus, suppose that investments now costs $(1 - \theta)I$ instead of just $I$. The FOC with respect to $I$ is now.
\begin{equation*}
\begin{split}
	q_t + \theta_t = 1 + C'(I_t)
\end{split}
\end{equation*} 
The equation for $\Delta q_t$ is unchanged. And let's consider the effect of a temporary versus a permanent increase in $\theta$.
\begin{center}
	\includegraphics[scale=0.4]{permTax}
\end{center}
$ $
\\
\begin{center}
	\includegraphics[scale=0.4]{tempTax}
\end{center}
$ $
\\
In both cases there is an immediate fall in $q$ and an increase in $I$. This is as the new path of $K_t$ will be higher, the NPV of MPK will fall.\\
\\
However, in the temporary case, $q$ will fall less, and investments will therefore be higher. Moreover, $q$ will actually be increasing close to the terminal date of the tax credit, inducing even more investments.\\
\\
In the short run, a temporary tax credit increases investments and output more than a permanent one.
\\










\chapter{Unemployment}


Key Facts:
\begin{enumerate}
	\item Unemployment rate fluctuates over time counter-cyclically. 
	
	\item Unemployment rates differ substantially across countries. 
	
	\item Unemployment tends to be persistent.\\
\end{enumerate}

Real wages can be above the market-clearing level, resulting in unemployment as the demand for workers is less than the supply. This can be because of: 
\begin{enumerate}
	\item An index-linked minimum wage set by the government. 
	
	\item Wage bargaining by unions (e.g. the insider-outsider model) or optimal contracting between firms and workers. 
	
	\item  Efficiency wages: higher wage increases productivity by improving health, reducing labour turnover, attracting better workers (less adverse selection) and preventing shirking (less moral hazard e.g. the Shapiro Stiglitz model). Therefore higher wage helps to recruit, retain and motivate workers.
	
\end{enumerate}

	
\section{Efficiency Wages and the Solow Condition}

Solow (1979). The representative firm maximises real profits $\pi$: 
	\begin{equation*}
	\begin{split}
		\pi = Y - wL 
	\end{split}
	\end{equation*}
\\
The price of the good $Y$ is normalised to $1$. Output is an increasing, concave function of labour $L$, augmented by the level of effort $e$: 
	\begin{equation*}
	\begin{split}
		Y = F(eL) \text{  with } F'(.) >0, F''(.) <0 
	\end{split}
	\end{equation*}
\\
We assume perfect competition in the goods market (firms are price takers: they treat prices are given). However, there is imperfect competition in the labor market: firms determine their labor input $L$ and real wage $w$. 
\\
\\
The representative worker has inelastic labor supply $\bar{L}$ and exerts effort $e$ as a function of his wage $w$: 
	\begin{equation*}
	\begin{split}
		w = w(e) \text{  with } w'(.) >0, w''(.) <0 
	\end{split}
	\end{equation*} 
\\
We can therefore write real profits as follows: 
	\begin{equation*}
	\begin{split}
		\pi(L, w) = F\big(e(w)L\big) - wL
	\end{split}
	\end{equation*}
\\
Firms set wages $w$ and labour input $L$ to maximise their profits. We take FOCs with respect to $w$ and $L$. 
	\begin{equation*}
	\begin{split}
		\frac{\partial \pi(L, w)}{\partial L} &= \frac{\partial  F\big(e(w)L\big) - wL}{\partial L} = 0 \\
	 		 & \frac{\partial F\big(e(w)L\big)}{\partial L} - \frac{\partial wL}{\partial L} = 0 \\
			 & \frac{\partial F\big(e(w)L\big)}{\partial e(w)L} \frac{\partial e(w)L}{\partial L} - w = 0\\
			 & F'\big( e(w)L \big) e(w) - w = 0 \\
			 & F'\big( e(w)L \big) e(w) = w \\
			 \implies & F'\big( e(w)L \big) = \frac{w}{e(w)}
	\end{split}
	\end{equation*}
\\
	\begin{equation*}
	\begin{split}
		\frac{\partial \pi(L, w)}{\partial w} &= \frac{\partial  F\big(e(w)L\big) - wL}{\partial w} = 0 \\
		 & \frac{\partial F\big(e(w)L\big)}{\partial e(w)L}\frac{\partial e(w)L}{\partial w}  - \frac{\partial wL}{\partial w} = 0 \\
		 & F'\big( e(w)L \big) e'(w)L - L = 0 \\
		 & F'\big( e(w)L \big) e'(w)L = L \\
		  \implies & F'\big( e(w)L \big) e'(w) = 1 \\
	\end{split}
	\end{equation*}
\\
Combining FOCs yields, we obtain the Solow condition: 
	\begin{equation*}
	\begin{split}
		\frac{F'\big( e(w)L \big)}{F'\big( e(w)L \big) e'(w)} = \frac{ \frac{w}{e(w)} }{1} \\
		\frac{1}{e'(w)} = \frac{w}{e(w)} \\
		\implies e'(w^*) = \frac{e(w^*)}{w^*}
	\end{split}
	\end{equation*}
\\
The optimal wage $w^*$ maximises the average effort per wage $\frac{e(w^*)}{w^*}$. Firms increase wages up to the point where increases in wages no longer increase $\frac{e(w^*)}{w^*}$. \\ \\This can be see graphically: 
\\
\begin{center}
	\includegraphics[scale=0.4]{efficiencyWage}
\end{center}
Optimal labour demand is given by $L^* = L(w^*)$, where $L'(w)<0$ (for $w$ close to $w^*$). The market-clearing wage $w$ solves $L(w) = \bar{L}$. We can represent this graphically: 
\begin{center}
	\includegraphics[scale=0.6]{labourDemand}
\end{center}
\begin{enumerate}
	\item If $w^* > w$, then supply exceeds demand $L^* < \bar{L}$, so we have unemployment.
	\item If $w^* < w$, demand exceeds supply so wages $w$ will rise until $L^* = \bar{L}$ and we have full employment.\\
\end{enumerate}

To make sure that we are maximising profit, we take second order conditions. 
	\begin{equation*}
	\begin{split}
		\frac{\partial \pi(L, w)}{\partial L} &= F'\big( e(w)L \big) e(w) - w \\
		\implies \frac{\partial^2 \pi(L, w)}{\partial L^2} &= \frac{\partial F'\big( e(w)L \big)}{\partial e(w)L } \frac{\partial  e(w)L }{\partial L} e(w) \\
			&= F''\big(e(w)L\big) e(w) e(w) \\
			&= F''\big(e(w)L\big) e(w)^2 
	\end{split}
	\end{equation*}
\\
	\begin{equation*}
	\begin{split}
		\frac{\partial \pi(L, w)}{\partial L} &= F'\big( e(w)L \big) e(w) - w \\
		\implies \frac{\partial^2 \pi(L, w)}{\partial L \partial w} &= \frac{\partial F'\big( e(w)L \big)}{\partial w} e(w) + F'\big( e(w)L \big) \frac{\partial e(w)}{\partial w} - 1 \\
		&= \frac{\partial F'\big( e(w)L \big)}{\partial  e(w)L } \frac{\partial e(w)L }{\partial w}  e(w) + F'\big( e(w)L \big) e'(w) - 1 \\
		&= F''\big( e(w)L \big) e'(w)L e(w) + F'\big( e(w)L \big) e'(w) - 1 \\
		&= e'(w) F'\big( e(w)L \big) + e(w)e'(w)L F''\big( e(w)L \big) - 1
	\end{split}
	\end{equation*}
\\
	\begin{equation*}
	\begin{split}
		\frac{\partial \pi(L, w)}{\partial w} &= F'\big( e(w)L \big) e'(w)L - L = 0 \\
		\implies \frac{\partial^2 \pi(L, w)}{\partial w^2} &= \frac{\partial F'\big( e(w)L \big)}{\partial w} e'(w)L + F'\big( e(w)L \big)\frac{\partial e'(w)L}{\partial w} \\
			&= \frac{\partial F'\big( e(w)L \big)}{\partial e(w)L} \frac{\partial e(w)L}{\partial w} e'(w)L + F'\big( e(w)L \big)e''(w)L \\
			&= F''\big( e(w)L \big) e'(w)L e'(w)L + F'\big( e(w)L \big)e''(w)L \\
			&= F''\big( e(w)L \big) \big(e'(w)L\big)^2 + F'\big( e(w)L \big)e''(w)L \\
	\end{split}
	\end{equation*}
\\

We write out the Hessian matrix of $\pi(L, w)$.  
\begin{equation*}
\begin{split}
	& D^2 \pi (L, w) = 
	\begin{bmatrix}
 	   \frac{\partial^2 \pi(L, w)}{\partial L^2}  &  \frac{\partial^2 \pi(L, w)}{\partial L \partial w} \\
 	   \frac{\partial^2 \pi(L, w)}{\partial w \partial L}  &  \frac{\partial^2 \pi(L, w)}{\partial w^2}
	\end{bmatrix} \\
	\\
	 = &
	\begin{bmatrix}
 	   F''\big(e(w)L\big) e(w)^2   &   e'(w) F'\big( e(w)L \big) + e(w)e'(w)L F''\big( e(w)L \big) - 1 \\
 	   e'(w) F'\big( e(w)L \big) + e(w)e'(w)L F''\big( e(w)L \big) - 1  &   F''\big( e(w)L \big) \big(e'(w)L\big)^2 + F'\big( e(w)L \big)e''(w)L 
	\end{bmatrix} \\
\end{split}
\end{equation*}
\\
We find it's determinant.
\begin{equation*}
\begin{split}
	\det{\bigg( D^2 \pi (L, w) \bigg)} = & \bigg(F''\big(e(w)L\big) e(w)^2 \bigg) \bigg(F''\big( e(w)L \big) \big(e'(w)L\big)^2 + F'\big( e(w)L \big)e''(w)L  \bigg) \\
		& - \bigg( e'(w) F'\big( e(w)L \big) + e(w)e'(w)L F''\big( e(w)L \big) - 1 \bigg)^2
\end{split}
\end{equation*}
\\
From the second FOC, we know that $e'(w)F'\big(e(w)L\big) = 1$. Hence,  
\begin{equation*}
\begin{split}
	\det{\bigg( D^2 \pi (L, w) \bigg)} = & \bigg(F''\big(e(w)L\big) e(w)^2 \bigg) \bigg(F''\big( e(w)L \big) \big(e'(w)L\big)^2 + F'\big( e(w)L \big)e''(w)L  \bigg) \\
		& - \bigg( 1 + e(w)e'(w)L F''\big( e(w)L \big) - 1 \bigg)^2 \\ 
		\\
		 = & F''\big(e(w)L\big) e(w)^2  F''\big( e(w)L \big) \big(e'(w)L\big)^2 + F''\big(e(w)L\big) e(w)^2  F'\big( e(w)L \big)e''(w)L  \\
		& -  e(w)^2 e'(w)^2 L^2 F''\big( e(w)L \big)^2  \\ 
		\\
		 = & F''\big(e(w)L\big)^2 e(w)^2  e'(w)^2 L^2 + F''\big(e(w)L\big) e(w)^2  F'\big( e(w)L \big)e''(w)L  \\
		& -  e(w)^2 e'(w)^2 L^2 F''\big( e(w)L \big)^2  \\ 
		\\
		= & F''\big(e(w)L\big) e(w)^2  F'\big( e(w)L \big)e''(w)L  \\
\end{split}
\end{equation*}
\\
\begin{equation*}
\begin{split}
	F''\big(e(w)L\big) < 0, e''(w) < 0 \implies F''\big(e(w)L\big) e''(w) >0 \\
	e(w)^2>0, F'\big( e(w)L \big)>0, L >0 \\ 
	\implies \det{\bigg( D^2 \pi (L, w) \bigg)} = F''\big(e(w)L\big) e(w)^2  F'\big( e(w)L \big)e''(w)L >0 
\end{split}
\end{equation*}
\\
The Hessian matrix $D^2 \pi$ is negative definite, so we have a global maximum.





\section{Efficiency Wages with Reference Wage}

Summers (1988). Consider same model as before, but now suppose: 
\begin{equation*}
\begin{split}
    e(w)=\left\{
                \begin{array}{ll}
                  \Big( \frac{w - \widetilde{w}}{\widetilde{w}} \Big)^\beta &\text{ if } w > \widetilde{w} \\
                  0 & \text{ otherwise}
                \end{array}
              \right.
\end{split}
\end{equation*}
\\
with $0<\beta<1$ and the reference wage (the expected labour income): 
\begin{equation*}
\begin{split}
    \widetilde{w} = (1-u) w_m + u \gamma w_m
\end{split}
\end{equation*}
\\
where $w_m$ is the market real wage, $u$ the unemployment rate and $\gamma$ the replacement ratio (the fraction of wages obtained from unemployment benefits), with $0<\gamma<1$. We assume $\beta + \gamma <1$. We can represent the effort function graphically as follows:  
\begin{center}
	\includegraphics[scale=0.4]{effortRef}
\end{center}
The optimal wage $w^*$ which maximises the average effort per wage $\frac{e(w^*)}{w^*}$, satisfies the Solow condition $e'(w^*) = \frac{e(w^*)}{w^*}$:
\begin{gather*}
   e'(w) = \frac{\partial \Big( \frac{w - \widetilde{w}}{\widetilde{w}} \Big)^\beta}{\partial w} = \frac{\partial \Big( \frac{w - \widetilde{w}}{\widetilde{w}} \Big)^\beta }{\partial  \frac{w - \widetilde{w}}{\widetilde{w}} } \frac{\partial  \frac{w - \widetilde{w}}{\widetilde{w}} }{\partial w} =  \beta \Big( \frac{w - \widetilde{w}}{\widetilde{w}} \Big)^{\beta -1} \frac{1}{\widetilde{w}} \\
\end{gather*}
\\
Therefore, 
\begin{gather*}
	e'(w^*) = \frac{e(w^*)}{w^*} \iff \beta \Big( \frac{w - \widetilde{w}}{\widetilde{w}} \Big)^{\beta -1} \frac{1}{\widetilde{w}} = \frac{\Big( \frac{w - \widetilde{w}}{\widetilde{w}} \Big)^\beta}{w} \\
	 \beta \Big( \frac{w - \widetilde{w}}{\widetilde{w}} \Big)^{\beta -1} \frac{w}{\widetilde{w}} = \Big( \frac{w - \widetilde{w}}{\widetilde{w}} \Big)^\beta \\
	 \beta \frac{w}{\widetilde{w}} = \frac{\Big( \frac{w - \widetilde{w}}{\widetilde{w}} \Big)^\beta}{\Big( \frac{w - \widetilde{w}}{\widetilde{w}} \Big)^{\beta -1}} \\
	  \beta \frac{w}{\widetilde{w}} = \frac{w - \widetilde{w}}{\widetilde{w}} \\
	   \beta w = w - \widetilde{w} \\
	   \beta w - w = - \widetilde{w} \\
	  w -  \beta w =  \widetilde{w} \\
	  w (1 -  \beta) =  \widetilde{w} \\
	  \\
	\implies w  = \frac{ \widetilde{w} }{1 -  \beta} > w \text{ since } 0<\beta<1 \\
\end{gather*}
\\
We assume that by symmetry, all firms set the same wage: $w = w_m$. 
\begin{equation*}
\begin{split}
	w  = \frac{ \widetilde{w} }{1 -  \beta}  \implies \widetilde{w} = w (1 -  \beta) \\
    \text{and }\widetilde{w} = (1-u) w_m + u \gamma w_m \\ 
\end{split}
\end{equation*}
\\
Combining expressions: 
\begin{equation*}
\begin{split}
    w (1 -  \beta) &= (1-u) w_m + u \gamma w_m \\ 
    w (1 -  \beta) &= (1-u) w + u \gamma w \text{ since } w = w_m \\ 
    w (1 -  \beta) &= w( 1-u + u \gamma ) \\
    1 -  \beta &= 1-u + u \gamma  \\
     -  \beta &= -u + u \gamma  \\
     \beta &= u - u \gamma  \\
      \beta &= u(1 - \gamma)  \\
      \implies u &= \frac{\beta}{1-\gamma}
\end{split}
\end{equation*}
\\
We find that the unemployment rate is determined by the level of unemployment benefits $\gamma$. We note the limiting case $\beta \rightarrow 0$  yields outcome under perfect competition in labor market: $e = 1$ and $u = 0$.




\section{Efficiency Wages and Shirking}

Shapiro-Stiglitz model (AER 1984) with imperfect worker monitoring. The representative firm maximises real profits: 
\begin{gather*}
	\pi_t = Y_t - w_t L_t
\end{gather*}
The price of output normalised to 1 and the only costs are wage costs. We assume the following production technology: 
	\begin{equation*}
	\begin{split}
		Y_t &= F(e_t L_t) \text{  with } F'(.) >0, F''(.) <0 \\
		\implies \pi_t &= F(e_t L_t) - w_t L_t
	\end{split}
	\end{equation*}
\\
There is perfect competition in the goods market: firms take the price of goods as given (and normalised to 1). There is imperfect competition in the labor market: firm chooses labor input $L$ and real wage $w$ in a way that maximises profits.
\\
\\
The representative worker maximises the expected value of lifetime utility: 
\begin{gather*}
	U =  \sum_{t=0}^{\infty}\frac{1}{(1+\rho)^t}v_t
\end{gather*}
where $\rho$ is the intertemporal discount rate ($\rho >0$) and instantaneous utility: 
\begin{equation*}
\begin{split}
    v(w)=\left\{
                \begin{array}{ll}
                  w_t - e_t &\text{ if employed}  \\
                  0 & \text{ otherwise}
                \end{array}
              \right.
\end{split}
\end{equation*}
Employee effort is given by: 
\begin{equation*}
\begin{split}
    e_t =\left\{
                \begin{array}{ll}
                  \bar{e} &\text{ if not shirking (working)}  \\
                  0 & \text{ otherwise}
                \end{array}
              \right.
\end{split}
\end{equation*}
We assume the following job market transition probabilities in any given period: 
\begin{enumerate}
	\item $b$ the probability of losing a job exogenously. 
	
	\item $q$ the probability of being fired due to shirking. 
	
	\item $a = a(u)$ the probability of finding a job, depending on the unemployment rate $u$. 

\end{enumerate}
We further assume that $b + q < 1$. 
\\
\\
We derive the expected value of lifetime utility of workers in different states (there is no closed form expression, just a system of equations).  
\begin{enumerate}

\item For a worker that is employed and not shirking, the expected value of lifetime utility $V_E$ is the sum of utility in the current period $w-\bar{e}$ and discounted expected utility in the next period (with discount rate $\frac{1}{1+\rho}$). In the next period, he will lose his job with probability $p$ and therefore be unemployed with utility $V_U$ and be employed with probability $1-p$ with utility $V_E$.
\begin{gather*}
	V_E = w - \bar{e} + \frac{1}{1+\rho}\Big(bV_U + (1 - b)V_E \Big)
\end{gather*}

\item For a worker that is employed and shirking, the expected value of lifetime utility $V_S$ is the sum of utility in the current period $w$ and discounted expected utility in the next period (with discount rate $\frac{1}{1+\rho}$). In the next period, he will lose his job with probability $p + q$ (as we need to add the probability of being fired due to shirking) and therefore be unemployed with utility $V_U$. He will continue to be employed and shirking with probability $1- p - q$ with utility $V_S$.
\begin{gather*}
	V_S = w + \frac{1}{1+\rho}\Big((b+q)V_U + (1 - b - q)V_S \Big)
\end{gather*}

To simplify our analysis, we assume that if the worker is not fired in the next period, he will be employed and not shirk, with utility $V_E$: 
\begin{gather*}
	V_S = w + \frac{1}{1+\rho}\Big((b+q)V_U + (1 - b - q)V_E \Big)
\end{gather*}
We will use this expression from now on.

\item A worker that is unemployed has a utility of $0$ in the current period and has discounted expected utility in the next period (with discount rate $\frac{1}{1+\rho}$). In the next period he will find a job with probability $a$ and therefore have utility $V_E$. With probability $1-a$ he will not find a job and still be unemployed with utility $V_U$.
\begin{gather*}
	V_U = \frac{1}{1+\rho}\Big(aV_E + (1 - a)V_U \Big)
\end{gather*}

\end{enumerate}
We assume that if $V_E = V_S$ then workers do not shirk (as they are indifferent between shirking and not shirking). Firms set the lowest wage $w$ that prevents shirking, and therefore satisfies $V_E = V_S$. We derive the "No Shirking Condition" (NOC): 

\begin{gather*}
	V_E = V_S \\
	\iff w - \bar{e} + \frac{1}{1+\rho}\Big(bV_U + (1 - b)V_E \Big) = w + \frac{1}{1+\rho}\Big((b+q)V_U + (1 - b - q)V_E \Big)\\ 
	 - \bar{e} + \frac{1}{1+\rho}\Big(bV_U + (1 - b)V_E \Big) =  \frac{1}{1+\rho}\Big((b+q)V_U + (1 - b - q)V_E \Big)\\
	 \frac{1}{1+\rho}\Big(bV_U + (1 - b)V_E \Big) - \frac{1}{1+\rho}\Big((b+q)V_U + (1 - b - q)V_E \Big) = \bar{e} \\
	 bV_U + (1 - b)V_E  - (b+q)V_U - (1 - b - q)V_E  = (1+\rho) \bar{e}   \\
	 bV_U + V_E  - bV_E - bV_U +qV_U - V_E + bV_E + qV_E  = (1+\rho) \bar{e}   \\
	 - qV_U + qV_E  = (1+\rho) \bar{e}   \\
	q(V_E - V_U)  = (1+\rho) \bar{e}   \\
	\implies V_E - V_U = (1+\rho) \frac{\bar{e}}{q}
\end{gather*}

We recall that $V_E = w - \bar{e} + \frac{1}{1+\rho}\Big(bV_U + (1 - b)V_E \Big)$ and $V_U = \frac{1}{1+\rho}\Big(aV_E + (1 - a)V_U \Big)$, hence: 
\begin{equation*}
\begin{split}
	V_E - V_U &= w - \bar{e} + \frac{1}{1+\rho}\Big(bV_U + (1 - b)V_E \Big) -  \frac{1}{1+\rho}\Big(aV_E + (1 - a)V_U \Big)   \\
		&= w - \bar{e} + \frac{1}{1+\rho}\Big(bV_U + (1 - b)V_E  -  aV_E - (1 - a)V_U \Big)  \\
		&= w - \bar{e} + \frac{1}{1+\rho}\Big((1 -a - b)V_E  - (1 - a - b)V_U \Big)  \\
		&= w - \bar{e} + \frac{1}{1+\rho}(1 -a - b)\big( V_E - V_U \big)   \\
\end{split}
\end{equation*}

Plugging in $V_E - V_U = (1+\rho) \frac{\bar{e}}{q}$: 
\begin{gather*}
	V_E - V_U = w - \bar{e} + \frac{1}{1+\rho}(1 -a - b) (1+\rho) \frac{\bar{e}}{q}   \\
	\implies V_E - V_U = w - \bar{e} + (1 -a - b) \frac{\bar{e}}{q}
\end{gather*}
\\
We have found two expressions for $V_E - V_U$, which we can set equal to each other and simplify: 
\begin{gather*}
	V_E - V_S = (1+\rho) \frac{\bar{e}}{q} = w - \bar{e} + (1 -a - b) \frac{\bar{e}}{q} \\
	(1+\rho) \frac{\bar{e}}{q} - (1 -a - b) \frac{\bar{e}}{q}  = w - \bar{e} \\
	\frac{\bar{e}}{q} (1+\rho - 1 +a +b)  = w - \bar{e} \\
	\frac{\bar{e}}{q} (\rho +a +b)  = w - \bar{e} \\
	\implies w = \bar{e} + \frac{\bar{e}}{q} (\rho +a +b) 
\end{gather*}
\\
In equilibrium, there are no shirkers as firms set wages $w$ such that $V_E = V_S$. Therefore, there are no firings due to shirking and the number of unemployed workers who find work is equal to the number of employed workers who are fired exogenously: 
\begin{gather*}
	a\big( \bar{L} - L\big) = bL \\
	\implies a = \frac{bL}{\bar{L} - L}
\end{gather*}
\\
Furthermore, 
\begin{equation*}
\begin{split}
	a + b &= \frac{bL}{\bar{L} - L} + b \\
		&= \frac{bL}{\bar{L} - L} + b \frac{\bar{L} - L}{\bar{L} - L} \\
		&= \frac{bL + b\bar{L} - bL}{\bar{L} - L}  \\
		&= \frac{b\bar{L}}{\bar{L} - L}  \\
		&= \frac{b}{ \frac{\bar{L} - L}{\bar{L}} } \\
	\implies a+b &= \frac{b}{ u } \text{ using } u = \frac{\bar{L} - L}{\bar{L}} \\
\end{split}
\end{equation*}
\\
We plug this into our expression for $w$ to obtain the No Shirking Condition (NSC): 
\begin{gather*}
	w = \bar{e} + \frac{\bar{e}}{q} \Big(\rho +\frac{b}{ u }\Big) \\
	\implies w = \bar{e}\Bigg(1 + \frac{1}{q} \Big(\rho +\frac{b}{ u }\Big) \Bigg)
\end{gather*}
\\
\\
We now derive the labour demand curve. We take FOCs of the firm's profit $\pi$ with respect to the labour input $L$: 
	\begin{equation*}
	\begin{split}
		\pi &= F(e L) - w L \\
		\frac{\partial F(e L) - w L}{\partial L} &= \frac{\partial F(e L) - w L}{\partial L} = F'(eL)e - w = 0
	\end{split}
	\end{equation*}
\\
Using the implicit function theorem: 
	\begin{equation*}
	\begin{split}
		\frac{\partial L}{\partial w} = \frac{1}{e^2 F''(eL)} <0 \text{ (difficult to derive)}
	\end{split}
	\end{equation*}
\\
We combine the NSC with the labour demand curve to obtain the labour market equilibrium. 
\begin{center}
	\includegraphics[scale=0.4]{NSC}
\end{center}
We can see that the No Shirking Condition is decreasing with unemployment. We can examine the effect of a rise in the probability of being fired due to shirking $q$: 
\begin{center}
	\includegraphics[scale=0.4]{riseQ}
\end{center}
The NSC shifts to the right, which leads to a decrease in equilibrium wage $w^*$ and an increase in employment $L^*$.
\\


\section{Hysteresis: Insider-Outsider Effects}

Hysteresis means persistence, path-dependence. We present the insider-outsider model of Blanchard and Summers (NBER MA 1986). The representative firm maximises real profits: 
\begin{gather*}
	\pi_t = Y_t - w_t L_t
\end{gather*}
The production function is given by: 
\begin{gather*}
	Y_t = A_t L_t^\alpha \\
\end{gather*}
where $0<\alpha<1$ and technology $A_t = \bar{A}\xi_t$ where $\xi_t$ is i.i.d with $E\big( \ln{\xi_t} \big) = 0$. Therefore, 
	\begin{equation*}
	\begin{split}
	E_{t-1}\big( \ln{\xi_t} \big) &= 0 \\
 	\ln{E_{t-1}\big( \xi_t \big)} &=0 \\
	E_{t-1}\big( \xi_t \big) &= e^0 = 1\\
	\implies E_{t-1}\big( \bar{A} \xi_t \big) &= E_{t-1}\big( \bar{A} \big) E_{t-1}\big( \xi_t \big) \\
		&=  E_{t-1}\big( \bar{A} \big) \\
		&= \bar{A}
	\end{split}
	\end{equation*}
and plugging in $Y_t = A_t L_t^\alpha$ and $A_t = \bar{A}\xi_t$: 
	\begin{equation*}
	\begin{split}
	\pi_t &= A_t L_t^\alpha - w_t L_t \\ 
		&= \bar{A}\xi_t L_t^\alpha - w_t L_t \\ 
	\end{split}
	\end{equation*}
\\
There is perfect competition in the goods market: firms take the price of goods as given (and normalised to 1). There is imperfect competition in the labor market: insiders set the real wage $w$ through collective wage bargaining. Workers are insiders if employed in previous period or outsiders if unemployed in previous period. 
\\
\\
$L_t$ is the number of employed workers in period $t$, whilst $N_t$ is the number of insiders in period $t$. Therefore: 
\begin{equation*}
\begin{split}
	N_t = L_{t-1}
\end{split}
\end{equation*}
Insiders set the wage rate$w_t^*$ to achieve the full employment of insiders in expectations: $E\big( L_t \big) = N_t$. Then, the technology shock $\xi_t$ is realised and finally, firms choose how much labour to utilise $L_t$. The hire $\min(L_t, N_t)$ insiders and $\max(L_t - N_t, 0)$ outsiders. If $L_t < N_t$, they hire only $L_t$ of the $N_t$ insiders. If $L_t > N_t$, they hire all $N_t$ insiders and $L_t - N_t$ outsiders.
\\
\\
Insiders consider the labour demand function of firms as a function of wages $w$, to set $w_t^*$ such that $E\big( L_t \big) = N_t$. Firms maximise profits with respect to labour inputs $L_t$, taking wages as given: 
\begin{equation*}
\begin{split}
	\frac{\partial \pi_t}{\partial L_t} &= \frac{\partial \bar{A}\xi_t L_t^\alpha - w_t L_t }{\partial L_t} = 0 \\
		&  \alpha \bar{A}\xi_t L_t^{\alpha-1} - w_t = 0 \\ 
		 &  \alpha \bar{A}\xi_t L_t^{\alpha-1} = w_t \\
		 & L_t^{\alpha-1} = w_t \big( \alpha \bar{A}\xi_t \big)^{-1}  \\
		 & \Big( L_t^{\alpha-1} \Big)^{\frac{1}{\alpha-1}} = \Big( w_t \big( \alpha \bar{A}\xi_t \big)^{-1} \Big)^{\frac{1}{\alpha-1}}  \\
		 & L_t =  w_t^{\frac{1}{\alpha-1}} \big( \alpha \bar{A}\xi_t \big)^{\frac{-1}{\alpha-1}} \\
		\implies & L_t =  \big( \alpha \bar{A}\xi_t \big)^{\frac{1}{1-\alpha}}  w_t^{\frac{-1}{1-\alpha}} 
\end{split}
\end{equation*}
Taking logs, 
\begin{equation*}
\begin{split}
	\ln{L_t} &=  \ln{\Big( \big( \alpha \bar{A}\xi_t \big)^{\frac{1}{1-\alpha}}  w_t^{\frac{-1}{1-\alpha}} \Big)} \\
		&=  \ln{ \big( \alpha \bar{A}\xi_t \big)^{\frac{1}{1-\alpha}} } + \ln{ w_t^{\frac{-1}{1-\alpha}} } \\
		&=  \ln{ \Big( \big( \alpha \bar{A}\big)^{\frac{1}{1-\alpha}}   \xi_t ^{\frac{1}{1-\alpha}} \Big)}  + \ln{ w_t^{\frac{-1}{1-\alpha}} } \\
		&=  \ln{ \big( \alpha \bar{A}\big)^{\frac{1}{1-\alpha}} } + \ln{  \xi_t ^{\frac{1}{1-\alpha}} }  + \ln{ w_t^{\frac{-1}{1-\alpha}} } \\
		\implies \ln{L_t} &=  \ln{L_0} + \frac{1}{1-\alpha} \ln{  \xi_t  }  - \frac{1}{1-\alpha} \ln{ w_t }
\end{split}
\end{equation*}
where $L_0 = \big( \alpha \bar{A}\big)^{\frac{1}{1-\alpha}}$. Insiders set $w_t^*$ such that:
\begin{equation*}
\begin{split}
 	E\big( L_t \big) = N_t \iff E\big( \ln{L_t} \big) = \ln{N_t}
\end{split}
\end{equation*}
We plug in our expression for $\ln{L_t}$: 
\begin{gather*}
 	E\bigg( \ln{L_0} + \frac{1}{1-\alpha} \ln{  \xi_t  }  - \frac{1}{1-\alpha} \ln{ w_t } \bigg) = \ln{N_t} \\
	\ln{L_0} +  \frac{1}{1-\alpha} E\big(\ln{  \xi_t  } \big)  - \frac{1}{1-\alpha} E\big( \ln{ w_t } \big) = \ln{N_t} \\
	\ln{L_0}  - \frac{1}{1-\alpha} \ln{ w_t^* }  = \ln{L_{t-1}}
\end{gather*}
using $E\big(\ln{ \xi_t } \big) = 0$, $N_t = L_{t-1}$ and the fact that workers choose $w_t$ in period $t$. Therefore, 
\begin{gather*}
	 \frac{1}{1-\alpha} \ln{ w_t^* }  = \ln{L_0}  -  \ln{L_{t-1}} \\
	 \ln{ w_t^* }  = (1-\alpha) \big( \ln{L_0}  -  \ln{L_{t-1}} \big) \\
\end{gather*}
We plug our expression for the optimal wage rate rate set by insiders $w_t^*$ into the labour demand function to determine how much labour firms employ in response to the wage rate: 
\begin{gather*}
		 \ln{L_t} =  \ln{L_0} - \frac{1}{1-\alpha} \ln{ w_t } + \frac{1}{1-\alpha} \ln{  \xi_t  }  \\
		 \ln{L_t} =  \ln{L_0} - \frac{1}{1-\alpha} \bigg( (1-\alpha) \big( \ln{L_0}  -  \ln{L_{t-1}} \big) \bigg) + \frac{1}{1-\alpha} \ln{  \xi_t  }  \\
		 \ln{L_t} =  \ln{L_0} -  \ln{L_0}  + \ln{L_{t-1}}+ \frac{1}{1-\alpha} \ln{  \xi_t  }  \\
		 \implies \ln{L_t} = \ln{L_{t-1}}+ \frac{1}{1-\alpha} \ln{  \xi_t  }  \\
\end{gather*}
As log employment $\ln{L_t}$ follows a random walk, we have hysteresis in unemployment.

We note that if insiders equal entire labor force, i.e. $N_t = \bar{L}$, then $\ln{L_t} = \ln{\bar{L}} + \frac{1}{1-\alpha} \ln{  \xi_t  }$, so no hysteresis.







\chapter{Nominal Rigidities}

Nominal rigidities include sticky prices and (downwardly) rigid nominal wages. These are important to explain Keynesian economics: with sticky prices output is determined by aggregate demand in the short run. Sticky prices require imperfect competition in the goods market (i.e. firms as price setters), as with perfect competition $P=MC$.\\
\\
Mankiw (QJE 1985) argues that nominal rigidities can be explained by menu costs: (typically small) costs of changing prices. \\
\\
In both cases, sticky price $\bar{P}_i$ has only a second-order effect on firm's profits.\\
\\
For the profit function $\pi_i(P_i)$ with optimal price $P_i^*$ satisfying $\pi_i'(P_i^*) = 0$ and $\pi_i''(P_i^*) <0$, we can write the second-order Taylor approximation around optimal price  $P_i^*$: 
\begin{equation*}
\begin{split}
	\pi_i(P_i) & \approx \pi_i(P_i^*) + \pi_i'(P_i^*)\big(P_i - P_i^* \big) + \frac{1}{2} \pi_i''(P_i^*) \big(P_i - P_i^* \big)^2 \\
		& \approx \pi_i(P_i^*)  + \frac{1}{2} \pi_i''(P_i^*) \big(P_i - P_i^* \big)^2 \\
\end{split}
\end{equation*}
\\
We can express the loss of profits to sticky prices (which are stuck at $\bar{P}_i$ instead of $P_i^*$): 
\begin{equation*}
\begin{split}
	\pi_i(P_i) - \pi_i(P_i^*)  & \approx  \frac{1}{2} \pi_i''(P_i^*) \big(P_i - P_i^* \big)^2  < 0 \\
\end{split}
\end{equation*}
\\
We can see this graphically: 
\begin{center}
	\includegraphics[scale=0.4]{sticky}
\end{center}
We can see that sticky prices have a second order effect on profits $\pi$: it is only do to the concavity of the profit function. Around $P_i^*$ profits are very close to the optimum. As we will see, on aggregate there are large-scale consequences to sticky prices.
\\

\section{Imperfect Competition with Flexible Prices}

The representative firm $i \in [0,1]$ maximises it's real profits $\pi_i$: 
\begin{equation*}
\begin{split}
	\pi_i = \frac{P_i}{P} Q_i - \frac{W}{P} L_i 
\end{split}
\end{equation*}
where $P_i$ is the price set by firm $i$, $P$ is the aggregate price level, $Q_i$ is the output of firm $i$, $W$ is the economy-wide nominal wage rate and $L_i$ is the labour used by firm $i$.  \\
\\
The production technology for firm $i$ is : 
\begin{equation*}
\begin{split}
	Q_i = L_i 
\end{split}
\end{equation*}
Output is equal to the number of workers employed. Therefore : 
\begin{equation*}
\begin{split}
	\pi_i &= \frac{P_i}{P} Q_i - \frac{W}{P} Q_i  \\
		&= Q_i \bigg( \frac{P_i}{P} - \frac{W}{P} \bigg) \\
		&= Q_i \frac{P_i - W}{P} 
\end{split}
\end{equation*}
\\ 
We assume monopolistic competition in the goods market: firm have the market power to set their price $P_i$.\\
There is perfect competition in the labor market: agents take nominal wage $W$ as given.\\
\\
The representative consumer $i$ maximises utility: 
\begin{equation*}
\begin{split}
	U_i = C_i - \frac{1}{\gamma} L_i^\gamma \text{ with } \gamma > 1
\end{split}
\end{equation*}
the agent experiences utility from consumption $C_i$ and disutility from labour $L_i$. We introduce the budget constraint: 
\begin{equation*}
\begin{split}
	C_i = \pi_i + \frac{W}{P}L_i 
\end{split}
\end{equation*}
As consumers own the firms, they earn profits which they can consume in addition to labour earnings (at the real wage rate $\frac{W}{P}$). We introduce the demand function for good $i$: 
\begin{equation*}
\begin{split}
	Q_i &= Y \bigg( \frac{P_i}{P}  \bigg)^{-\eta} \\
		&= Y \bigg( \frac{P}{P_i}  \bigg)^{\eta}
\end{split}
\end{equation*}
where $\eta$ is the elasticity of demand. Aggregate demand is: 
\begin{equation*}
\begin{split}
	Y = \frac{M}{P}
\end{split}
\end{equation*}
where $M$ is aggregate demand (or money supply). There is a negative relationship between aggregate output and the price level. \\
\\
We derive producer behaviour with flexible prices. We substitute the demand function into the profit function: 
\begin{equation*}
\begin{split}
	\pi_i &= Q_i \frac{P_i - W}{P} \\
		&= Y \bigg( \frac{P_i}{P}  \bigg)^{-\eta} \frac{P_i - W}{P} \\
\end{split}
\end{equation*}
\\ 
As there is imperfect competition in the goods market can chooses prices $P_i$ to maximise profits. Taking FOCs with respect to $P_i$: 
\begin{equation*}
\begin{split}
	\frac{\partial \pi_i}{\partial P_i} &= \frac{\partial Y \Big( \frac{P_i}{P}  \Big)^{-\eta} \frac{P_i - W}{P}}{\partial P_i} =0  \\
		& \frac{\partial Y \Big( \frac{P_i}{P}  \Big)^{-\eta}}{\partial P_i} \frac{P_i - W}{P} + Y \Big( \frac{P_i}{P}  \Big)^{-\eta} \frac{\partial \frac{P_i - W}{P}}{\partial P_i} =0 \\
		& Y \frac{\partial \Big( \frac{P_i}{P}  \Big)^{-\eta}}{\partial  \frac{P_i}{P} } \frac{\partial  \frac{P_i}{P} }{P_i} \frac{P_i - W}{P} + Y \Big( \frac{P_i}{P}  \Big)^{-\eta} \frac{1}{P} =0 \\
		& Y (-\eta)\Big( \frac{P_i}{P}  \Big)^{-\eta-1} \frac{1}{P} \frac{P_i - W}{P} + Y \Big( \frac{P_i}{P}  \Big)^{-\eta} \frac{1}{P} =0 \\
		& Y \Big( \frac{P_i}{P}  \Big)^{-\eta} \frac{1}{P} - \eta \frac{P_i - W}{P} \frac{Y}{P} \Big( \frac{P_i}{P}  \Big)^{-\eta-1} =0 \\
		& \Big( \frac{P_i}{P}  \Big)^{-\eta} \frac{Y}{P} = \eta \frac{P_i - W}{P} \frac{Y}{P} \Big( \frac{P_i}{P}  \Big)^{-\eta-1}  \\
		& \frac{ \Big( \frac{P_i}{P}  \Big)^{-\eta}}{\Big( \frac{P_i}{P}  \Big)^{-\eta-1} }  = \eta \frac{P_i - W}{P}   \\
		&  \Big( \frac{P_i}{P}  \Big)^{-\eta - (\eta - 1)}  = \eta \frac{P_i - W}{P}   \\
		&  \frac{P_i}{P}  = \eta \Big( \frac{P_i}{P} - \frac{W}{P} \Big)   \\
		&  \frac{P_i}{P}  = \eta  \frac{P_i}{P} - \eta \frac{W}{P}  \\
		&  \frac{P_i}{P} -\eta  \frac{P_i}{P} = - \eta \frac{W}{P}  \\
		&  \eta  \frac{P_i}{P} - \frac{P_i}{P}  = \eta \frac{W}{P}  \\
		&  (\eta  - 1) \frac{P_i}{P} = \eta \frac{W}{P}  \\
		\\
		\implies & \frac{P_i}{P} = \frac{\eta}{\eta  - 1} \frac{W}{P} > \frac{W}{P} \\
\end{split}
\end{equation*}
\\ 
We find that firms engage in markup pricing due to imperfect competition. All firms face the same conditions so they all set the same price $P_i$. Therefore for all $i$, $P_i = P$ where $P$ is the average price. In the symmetric equilibrium: 
\begin{equation*}
\begin{split}
	\frac{P_i}{P} = 1 &= \frac{\eta}{\eta  - 1} \frac{W}{P}  \\
	\implies \frac{\eta  - 1}{\eta} &=  \frac{W}{P} < 1
\end{split}
\end{equation*}
\\ 
We note that prices above marginal cost and real wage below the marginal product of labour are due to imperfect competition.\\
\\
\\
We now turn to deriving consumer behaviour. We plug the budget constraint $C_i = \pi_i + \frac{W}{P}L_i$ into the utility function $U_i = C_i - \frac{1}{\gamma} L_i^\gamma$: 
\begin{equation*}
\begin{split}
	U_i = \pi_i + \frac{W}{P}L_i  - \frac{1}{\gamma} L_i^\gamma
\end{split}
\end{equation*}
\\ 
Consumers do not have control over firms so take $\pi_i$ as given, but they choose their labour supply $L_i$ to maximise their utility. Taking FOCs w.r.t. $L_i$: 
\begin{equation*}
\begin{split}
	\frac{\partial U_i}{\partial L_i} &= \frac{\partial \pi_i + \frac{W}{P}L_i  - \frac{1}{\gamma} L_i^\gamma}{\partial L_i} = 0 \\
		& \frac{W}{P} - \frac{1}{\gamma}\gamma L_i^{\gamma-1} = 0 \\
		& L_i^{\gamma-1} = \frac{W}{P}  \\ 
		\\
		\implies & L_i^S = \bigg( \frac{W}{P} \bigg)^{\frac{1}{\gamma-1}}
\end{split}
\end{equation*}
\\ 
As $\frac{\partial L}{\partial W} >0$ the income effect is larger than the substitution effect. Since in equilibrium, $\frac{W}{P} = \frac{\eta  - 1}{\eta}$ we can plug this into the labour supply: 
\begin{equation*}
\begin{split}
	L_i^S = \bigg(  \frac{\eta  - 1}{\eta} \bigg)^{\frac{1}{\gamma-1}}
\end{split}
\end{equation*}
where the production technology $Q_i = L_i$ implies that $Y = L$, hence we can write aggregate supply as: 
\begin{equation*}
\begin{split}
	Y^S_{FLEX} = \bigg(  \frac{\eta  - 1}{\eta} \bigg)^{\frac{1}{\gamma-1}} < 1
\end{split}
\end{equation*}
With flexible prices, output is determined by aggregate supply and money $M$ is neutral. The market equilibrium $Y$ under imperfect competition is below the socially optimal level ($Y = 1$ under perfect competition) due to the aggregate demand externality. The reduction in prices $P_i$ (and therefore $P$) increases aggregate demand $Y = \frac{M}{P}$, leading to additional rise in demand $Q_i$ (Blanchard and Kiyotaki, AER 1987).\\
\\



\section{Menu Costs}

We consider the same imperfect competition model as before but now assume that there is a menu cost: a fixed cost $Z$ of changing goods prices $P_i$. \\
\\
Starting from a flexible price equilibrium (Z = 0), we consider and (unanticipated) change in aggregate demand from $M_0$ to $M_1$.\\
\\
Sticky prices $P_i = P_0$ are a Nash equilibrium $\iff$
\begin{equation*}
\begin{split}
	\pi_{ADJ} - \pi_{FIX} < Z
\end{split}
\end{equation*}
where $\pi_{FIX}$ is profits if all firms keep prices fixed at $P_0$ and $\pi_{ADJ}$ is profits for firm that deviates and adjusts price to optimal level. Therefore $\pi_{ADJ} - \pi_{FIX}$ is the increase in profits which results from adjusting prices. Firms will only adjust if the increase in profits is larger that the menu cost $Z$. We can see this graphically: 
\begin{center}
	\includegraphics[scale=0.45]{drop}
\end{center}
With sticky prices $P = \bar{P}$, output is determined by aggregate demand: 
\begin{equation*}
\begin{split}
	Y = \frac{M}{\bar{P}}
\end{split}
\end{equation*}
and real wages are also increasing in aggregate demand: 
\begin{equation*}
\begin{split}
	\frac{W}{P} = \bigg( \frac{M}{P} \bigg)^{\gamma - 1}
\end{split}
\end{equation*}
\\
\\
We derive the labour market equilibrium with sticky prices. The labour supply is still given by the consumer's FOC w.r.t. $L_i$: $L^S = \Big( \frac{W}{P} \Big)^{\frac{1}{\gamma-1}}$. However, with sticky prices the firm's FOC w.r.t. $P_i$ no longer holds. Instead, $Y = \frac{M}{\bar{P}}$ and $Y=L$ gives rise to effective labour demand by firms $L^d = \frac{M}{\bar{P}}$. We obtain the labour market equilibrium by equating supply and demand: 
\begin{equation*}
\begin{split}
	L^d = \frac{M}{\bar{P}} = \bigg( \frac{W}{P} \bigg)^{\frac{1}{\gamma-1}} = L^S \\
\end{split}
\end{equation*}
\\
\begin{center}
	\includegraphics[scale=0.35]{graphTable}
\end{center}

Rearranging, 
\begin{equation*}
\begin{split}
	\bigg(\frac{W}{P}\bigg)^{\gamma -1} = \bigg( \frac{M}{P} \bigg)^{\frac{\gamma -1}{\gamma -1}}  \\
	\implies \frac{M}{P} = \bigg(\frac{W}{P}\bigg)^{\gamma -1}
\end{split}
\end{equation*}
We find that the real wage is increasing in aggregate demand. \\
\\ 
We now derive the profit function $\pi_{ADJ}$ and $\pi_{FIX}$. We recall the profit function: 
\begin{equation*}
\begin{split}
	\pi_i &= Y \bigg( \frac{P_i}{P}  \bigg)^{-\eta} \frac{P_i - W}{P} \\
\end{split}
\end{equation*}
and equilibrium real wage with sticky prices:
\begin{equation*}
\begin{split}
	\frac{M}{P} = \bigg(\frac{W}{P}\bigg)^{\gamma -1}
\end{split}
\end{equation*}
\\
\begin{enumerate}
\item For firms that keep their price $P_i$ fixed at $P_0$, markup pricing $\frac{P_i}{P} = \frac{\eta}{\eta  - 1}\frac{W}{P}$ fails. We substitute $P_i = P_0 = P$, $Y = \frac{M}{P}$ and $\frac{W}{P} = ( \frac{M}{P} )^{\gamma - 1}$ into the profit function: 
\begin{equation*}
\begin{split}
	\pi_i &= Y \bigg( \frac{P_i}{P}  \bigg)^{-\eta} \bigg( \frac{P_i}{P} - \frac{W}{P}  \bigg) \\
		&= Y \bigg( \frac{P}{P}  \bigg)^{-\eta} \bigg( \frac{P}{P} - \frac{W}{P}  \bigg) \\
		&= Y \bigg(1 - \frac{W}{P}  \bigg) \\
		&= Y  -  Y \frac{W}{P} \\
		&= Y  -  Y \bigg( \frac{M}{P} \bigg)^{\gamma - 1} \\
		&= \frac{M}{P} -  \frac{M}{P}  \bigg( \frac{M}{P} \bigg)^{\gamma - 1} \\
	\implies \pi_{FIX} &= \frac{M}{P} -  \bigg( \frac{M}{P} \bigg)^\gamma \\
\end{split}
\end{equation*}
\\
\item For firm that adjusts price, we plug $\frac{P_i}{P} = \frac{\eta}{\eta  - 1}\frac{W}{P}$,  $\frac{W}{P} = ( \frac{M}{P} )^{\gamma - 1}$, $Y = \frac{M}{P}$ and $P = P_0$ into the profit function: 
\begin{equation*}
\begin{split}
	\pi_i &= Y \bigg( \frac{P_i}{P}  \bigg)^{-\eta} \bigg( \frac{P_i}{P} - \frac{W}{P}  \bigg) \\
		&= Y \bigg( \frac{\eta}{\eta  - 1}\frac{W}{P}  \bigg)^{-\eta} \bigg( \frac{\eta}{\eta  - 1}\frac{W}{P} - \frac{W}{P}  \bigg) \\
		&= Y \bigg( \frac{\eta}{\eta  - 1} \bigg)^{-\eta} \bigg( \frac{W}{P}  \bigg)^{-\eta} \frac{W}{P} \bigg( \frac{\eta}{\eta  - 1} - 1  \bigg) \\
		&= Y \bigg( \frac{\eta}{\eta  - 1} \bigg)^{-\eta} \bigg( \frac{W}{P}  \bigg)^{-\eta} \frac{W}{P}  \frac{1}{\eta  - 1}  \\
		&=  \frac{1}{\eta  - 1} \bigg( \frac{\eta}{\eta  - 1} \bigg)^{-\eta} Y \bigg( \Big( \frac{M}{P} \Big)^{\gamma - 1}  \bigg)^{1-\eta}   \\
		&=  \frac{1}{\eta  - 1} \bigg( \frac{\eta}{\eta  - 1} \bigg)^{-\eta} \frac{M}{P} \bigg( \frac{M}{P} \bigg)^{\gamma - 1 - \eta \gamma - \eta}   \\
	\implies \pi_{ADJ} &=  \frac{1}{\eta  - 1} \bigg( \frac{\eta}{\eta  - 1} \bigg)^{-\eta} \bigg( \frac{M}{P_0} \bigg)^{\gamma - \eta (\gamma - 1)}   \\
\end{split}
\end{equation*}

\end{enumerate}
$ $
$ $
\\
We plug in plausible parameter values, and we find that implausibly high menu costs are required to sustain a Nash equilibrium of sticky prices.
\\
We take $\gamma = 11$ (so inelastic labour supply $L^S = \big( \frac{W}{P} \big)^{\frac{1}{\gamma-1}}$ with $\frac{1}{\gamma - 1} = 0.1$) and $\eta = 5$ (so markup facto $\frac{\eta}{\eta -1} = 1.25$.\\
\\
We start from the flexible-price outcome: $Y_{flex} = \Big(  \frac{\eta  - 1}{\eta} \Big)^{\frac{1}{\gamma-1}} = 0.978$.\\
\\
We consider a $3\%$ drop in M: $\frac{M}{P_0} = 0.97 \frac{M_0}{P_0} = 0.97 Y_{flex}$ using $Y = \frac{M}{P}$. \\
\\
Substituting into $\pi_{FIX} = \frac{M}{P} -  \bigg( \frac{M}{P} \bigg)^\gamma$ and $\pi_{ADJ} =  \frac{1}{\eta  - 1} \bigg( \frac{\eta}{\eta  - 1} \bigg)^{-\eta} \bigg( \frac{M}{P_0} \bigg)^{\gamma - \eta (\gamma - 1)}$: 
\begin{equation*}
\begin{split}
	\frac{\pi_{ADJ} - \pi_{FIX}}{Y_{flex}} = 0.258
\end{split}
\end{equation*}
\\
So, menu costs of more than $25\%$ of real revenue (output) are needed for fixed prices to be a Nash equilibrium. The low elasticity of labor supply $\frac{1}{\gamma - 1}$ amounts to high real wage flexibility ( $\frac{W}{P} = Y^{\gamma-1}$ ). Small menu costs lead to nominal rigidity (i.e. small $\pi_{ADJ} - \pi_{FIX}$) if real rigidity is high or the price sensitivity of profits is low.\\




\section{Menu Costs and Real Rigidities}

We add real rigidity to the model (Ball and Romer 1990). Instead of perfect competition in labor market with $\frac{M}{P} = \bigg(\frac{W}{P}\bigg)^{\gamma -1}$, we assume a real wage function (for example, due to efficiency wages or wage bargaining):
\begin{equation*}
\begin{split}
	\frac{W}{P} = BY^\beta \text{ with } B, \beta > 0
\end{split}
\end{equation*}
\\
We derive the profit functions $\pi_{FIX}$ and $\pi_{ADJ}$. We recall the profit function:
\begin{equation*}
\begin{split}
\pi_i =  \bigg( \frac{P_i}{P} - \frac{W}{P} \bigg) Y \bigg( \frac{P_i}{P}  \bigg)^{-\eta}
\end{split}
\end{equation*}
the real wage function: 
\begin{equation*}
\begin{split}
\frac{W}{P} = BY^\beta 
\end{split}
\end{equation*}
and optimal prices are given by: 
\begin{equation*}
\begin{split}
\frac{P_i}{P} = \frac{\eta}{\eta  - 1} \frac{W}{P} 
\end{split}
\end{equation*}
\\
\\
For firms that keep price $P_i$ fixed  markup pricing $\frac{P_i}{P} = \frac{\eta}{\eta  - 1} \frac{W}{P}$  fails. We substitute $P_i = P_0 = P$, $Y = \frac{M}{P}$ and $\frac{W}{P} = BY^\beta$ into the profit function: 
\begin{equation*}
\begin{split}
\pi_{FIX} &=  \bigg( \frac{P_i}{P} - \frac{W}{P} \bigg) Y \bigg( \frac{P_i}{P}  \bigg)^{-\eta} \\
	&=  \bigg( \frac{P}{P} - \frac{W}{P} \bigg) Y \Big( \frac{P}{P}  \bigg)^{-\eta} \\
	&=  \bigg( 1 - \frac{W}{P} \bigg) Y 1^{-\eta} \\
	&=  Y -  Y \frac{W}{P} \\
	&=  \frac{M}{P} -  Y BY^\beta \\
	&=  \frac{M}{P_0} -  BY^{\beta+1} \\
	\implies \pi_{FIX} &=  \frac{M}{P_0} -  B\bigg(\frac{M}{P_0}\bigg)^{\beta+1} \\
\end{split}
\end{equation*}
\\
For firms that adjusts price $P_i$, we substitute $\frac{P_i}{P} = \frac{\eta}{\eta  - 1} \frac{W}{P}$, $\frac{W}{P} = BY^\beta$, $Y = \frac{M}{P}$ and $P = P_0$ into $\pi_i$: 
\begin{equation*}
\begin{split}
	\pi_i &=  \bigg( \frac{P_i}{P} - \frac{W}{P} \Big) Y \bigg( \frac{P_i}{P}  \bigg)^{-\eta} \\
		&=  \bigg( \frac{\eta}{\eta  - 1} \frac{W}{P} - \frac{W}{P} \bigg) Y \bigg( \frac{\eta}{\eta  - 1} \frac{W}{P}  \bigg)^{-\eta} \\
		&=  \frac{W}{P}  \bigg( \frac{\eta}{\eta  - 1} - 1 \bigg) Y \bigg( \frac{\eta}{\eta  - 1} \bigg)^{-\eta} \bigg( \frac{W}{P}  \bigg)^{-\eta} \\
		&=   \frac{\eta - (\eta  - 1)}{\eta  - 1}  Y \bigg( \frac{\eta}{\eta  - 1} \bigg)^{-\eta} \bigg( \frac{W}{P}  \bigg)^{1-\eta} \\
		&=   \frac{1}{\eta  - 1}  \bigg( \frac{\eta}{\eta  - 1} \bigg)^{-\eta} Y \bigg( BY^\beta  \bigg)^{1-\eta} \\
		&=   \frac{1}{\eta  - 1}  \bigg( \frac{\eta}{\eta  - 1} \bigg)^{-\eta} Y B^{1-\eta}Y^{\beta - \beta \eta} \\
		&=  B^{1-\eta} \frac{1}{\eta  - 1}  \bigg( \frac{\eta}{\eta  - 1} \bigg)^{-\eta} Y^{1 + \beta - \beta \eta} \\
	\implies \pi_{ADJ} &= B^{1-\eta} \frac{1}{\eta  - 1} \bigg( \frac{\eta}{\eta  - 1} \bigg)^{-\eta} \bigg( \frac{M}{P_0} \bigg)^{1 + \beta - \beta \eta} \\
\end{split}
\end{equation*}
\\
We derive the flexible price outcome with real rigidities:
\begin{equation*}
\begin{split}
	\frac{W}{P} &= BY^\beta \\
	Y^\beta &= \frac{1}{B}\frac{W}{P} \\
	Y &= \bigg( \frac{1}{B}\frac{W}{P} \bigg)^\frac{1}{\beta} \\
\end{split}
\end{equation*}
As $\frac{P_i}{P} = \frac{\eta}{\eta  - 1} \frac{W}{P}$ and $P_i = P$, $\frac{P}{P} = \frac{\eta}{\eta  - 1} \frac{W}{P} = 1$ and therefore, $ \frac{W}{P} = \frac{\eta- 1}{\eta}$. Plugging this in: 
\begin{equation*}
\begin{split}
	Y_{flex} &= \bigg( \frac{1}{B}\frac{\eta- 1}{\eta} \bigg)^\frac{1}{\beta} \\
\end{split}
\end{equation*}
\\
We derive plausible parameter values which yield a Nash equilibrium of sticky prices, as real rigidity amplify the effect of nominal rigidity.\\
\\
We take $\beta = 0.1$ (so high real wage rigidity, wages don't react a lot to changes in output), $B = 0.806$ (so output gap $\approx  5\%$) and $\eta = 5$ (so markup factor $\frac{\eta}{\eta-1} = 1.25$). \\
\\
We start from the flexible-price outcome: $Y_{flex} = \bigg( \frac{1}{B}\frac{\eta- 1}{\eta} \bigg)^\frac{1}{\beta} = 0.928$ (previously 0.978).
We consider a $3\%$ drop in M: $\frac{M}{P} = 0.97 \frac{M_0}{P_0} = 0.97 Y_{flex}$. Substituting into $\pi_{FIX} = \frac{M}{P_0} -  B\bigg(\frac{M}{P_0}\bigg)^{\beta+1}$ and $\pi_{ADJ} = B^{1-\eta} \frac{1}{\eta  - 1} \bigg( \frac{\eta}{\eta  - 1} \bigg)^{-\eta} \bigg( \frac{M}{P_0} \bigg)^{1 + \beta - \beta \eta}$: 
\begin{equation*}
\begin{split}
	\frac{\pi_{ADJ} - \pi_{FIX}}{Y_{flex}} = 0.0000181
\end{split}
\end{equation*}
\\
So, menu costs only need to be $0.002\%$ of real revenue (output) for sticky prices to be a Nash equilibrium. As a result, small microeconomic nominal friction with some real rigidity could generate substantial aggregate nominal rigidity. In conclusion, a Nash equilibrium of sticky prices can be obtained for small menu costs with real rigidity.

\section{Empirical Evidence}

Ball, Mankiw and Romer (1988) test the implication of the menu cost model. When average inflation $\bar{\pi}$ is higher, aggregate demand shocks $\Delta x$ have a smaller effect on output $y$ because prices are adjusted more often. In th etime series 
\begin{equation*}
\begin{split}
	y_t = c + \gamma_t + \tau \Delta x_t + \delta y_{t-1}
\end{split}
\end{equation*}
and in the cross section:
\begin{equation*}
\begin{split}
	\tau_t = \underset{(0.079)}{0.600} - \underset{(1.074)}{4.835} \bar{\pi}_i +  \underset{(2.088)}{7.118} \bar{\pi}_i^2
\end{split}
\end{equation*}
\\
where $\Delta x$ log nominal GDP ($\bar{R}^2 = 0.388$, s.e.e. = 0.215, standard errors in parentheses). There is a significantly negative relation between $\bar{\pi}$ and $\tau$ (for $\bar{\pi} < 34\%$).\\
\\
Microeconomic evidence on price adjustments:
\begin{enumerate}
\item  Prices adjust about once per year (excluding sales), though there is heterogeneity in frequency, size and timing.
\item The costs of price adjustment can be significant (e.g. $1\%$ for supermarkets; Levy, Bergen, Dutta and Venable, QJE 1997).
\end{enumerate}










\chapter{Monetary Policy}

There are two key objectives for monetary policy: 
\begin{enumerate}
	\item Price stability or inflation stabilisation around an inflation target $\pi^*$. 
	
	\item Output stabilisation.
\end{enumerate}
These objectives depend on the mandate given to the Central Bank.
\\
\section{Monetary Policy Objectives}
We assume the following aggregate supply equation with white noise shock $s_t$ (with $E_t\big(s_t\big)$):
\begin{gather*}
		 y_t = \bar{y} + b\big( \pi_t - \pi_t^e \big) + s_t 
\end{gather*}
where $\bar{y}$ is natural rate of output, $\pi_t^e$ is expected inflation and $s_t$ is a supply shock. This is equivalent to the expectations augmented Phillips curve: 
\begin{gather*}
		 y_t -\bar{y} - s_t =  b\big( \pi_t - \pi_t^e \big) \\
		  \pi_t - \pi_t^e  = \frac{1}{b} \big( y_t -\bar{y} - s_t \big) \\
		  \pi_t = \pi_t^e + \frac{1}{b} \big( y_t -\bar{y} \big) - \frac{s_t}{b} \\
		  \implies  \pi_t = \pi_t^e + \theta \big( y_t -\bar{y} \big) + \epsilon_t \\
\end{gather*}
where $\theta= \frac{1}{b}$ and $\epsilon_t = - \frac{s_t}{b}$. This is the Keynesian way of looking at aggregate supply. 
\\
\\ 
To reach the inflation target $E\big( \pi_t \big) = \pi^*$, if inflation expectations are the same as the target $\pi_t^e = \pi^*$, then we set output at the natural rate: $y_t = \bar{y}$. If inflation expectations are below the target $\pi_t^e < \pi^*$, then we set output above the natural rate: $y_t > \bar{y}$. If inflation expectations are above the target $\pi_t^e > \pi^*$, then we set output below the natural rate: $y_t < \bar{y}$.
\\
\\ 
The stabilisation of shocks $s_t$ and $\epsilon_t$ is feasible, unless there is perfect foresight (i.e. $\pi_t^e = \pi_t$), and desirable, unless there are technology and preference shocks (recall RBC). 
\\
\\
Note: Systematic expansionary policy ($E\big(y_t\big) > \bar{y}$) requires $\pi_t > \pi_t^e$, so with 
\begin{enumerate}
	\item Adaptive expectations (i.e. $\pi_t^e = \pi_{t?1}$), increasing inflation ($\pi_t > \pi_{t?1}$).
	\item Rational expectations (i.e. $\pi_t^e = E\big(\pi_t \big)$), it is impossible (as $E\big(y_t\big) = \bar{y}$).

\end{enumerate}
$ $ 
$ $
\\
The costs of inflation:
\begin{enumerate}
	\item Shoe-leather costs due to the reduction of real money holdings caused by the higher opportunity cost of money ($i = r + \pi^e$). 
	\item Menu costs associated with adjusting nominal prices and wages.
	\item Distortion of relative prices due to the infrequent price adjustments.
	\item Inflation distortions induced by the tax system.
	\item Inconvenience of changing value unit of account and errors in financial planning.
\end{enumerate}
$ $ 
$ $
\\
In addition, higher inflation tends to be more variable and unpredictable. For unanticipated inflation:
\begin{enumerate}
	\item Greater uncertainty (which is welfare reducing under risk aversion). 
	\item The unintended redistribution of wealth in nominal assets (between borrowers and savers).
\end{enumerate}
$ $ 
$ $
\\
The benefits of inflation:
\begin{enumerate}
	\item There is a reduction of real wage rigidity when nominal wages are downwardly rigid (?inflation greases the wheels of the labor market?).
	\item Monetary policy is less constrained by an (effective) lower bound on nominal interest rate. 
	\item The government receives revenues from printing money (seignorage) which allows for a reduction of distortionary taxes.
\end{enumerate}
$ $ 
$ $ 
\\
In conclusion, the optimal rate of inflation $\pi^*$ is likely to be positive, but small.


\section{Inflation Bias}

There is no inflation-output trade-off in long run (the AS curve vertical). We cannot increase output in the long run by increasing inflation. However empirically, inflation is often higher than the rate that is optimal or socially desirable: there is inflation bias.
\\ 
\\
Kydland and Prescott (JPE 1977): Discretionary low-inflation monetary policy is dynamically inconsistent.
\\
\\ 
Dynamic inconsistency: a multi-period decision $(x_t)_{t=0}^T$ which is optimal at time $0$ but no longer so at some future time $0 < t' < T$ (in game theory, equilibrium not subgame-perfect).
\\
\\
We present the monetary policy game of Barro and Gordon (JPE 1983).
\\
\\
The monetary policymaker minimises the following social welfare loss function:
\begin{gather*}
		 L = \frac{1}{2} a \big( \pi - \pi^* \big)^2 + \frac{1}{2} \big( y - y^*  \big)^2
\end{gather*}
where $\pi^*$ is socially optimal inflation, $a$ is the relative importance given to inflation relative to output and $y^*$ is socially optimal output, with $y^* > \bar{y}$ (due to distortions or market imperfections).
\\
\\
The economy is described by the aggregate supply equation: 
\begin{gather*}
	y = \bar{y} + b \big(\pi - \pi^e \big) + s \text{ with } b > 0
\end{gather*}
where $\bar{y}$ is the natural rate of output, and $s$ aggregate supply shock, with s $\sim$ i.i.d. $(0,\,\sigma_s^{2})$.
\\
\\
Timing: 
\begin{enumerate}
	\item The public rationally forms inflation expectations: $\pi^e = E\big( \pi \big)$.
	\item The aggregate supply shock $s$ is realised and monetary policymaker sets $\pi$.
\end{enumerate}
$ $ 
$ $ 
\\
We plug the aggregate supply equation into the welfare loss function:
\begin{gather*}
	L = \frac{1}{2} a \big( \pi - \pi^* \big)^2 + \frac{1}{2} \big( \bar{y} + b \big(\pi - \pi^e \big) + s - y^*  \big)^2
\end{gather*}
\\
We proceed by backwards induction: we solve the monetary authority's problem, which sets inflation $\pi$ after inflation expectations $\pi^e$  are formed and the aggregate supply shock $s$ is realised. The monetary authority's reaction function minimises the loss $L$ with respect to $\pi$, taking inflation expectations $\pi^e$ and the aggregate supply shock $s$ as given. Taking FOCs with respect to $\pi$: 
\begin{equation*}
\begin{split}
	\frac{\partial L}{\pi} &= \frac{1}{2} a \frac{\partial \big(\pi - \pi^* \big)^2}{\partial \pi - \pi^*} \frac{\partial \pi - \pi^*}{\partial \pi }  + \frac{1}{2} \frac{\partial \big( \bar{y} + b \big(\pi - \pi^e \big) + s - y^*  \big)^2}{\partial  \bar{y} + b \big(\pi - \pi^e \big) + s - y^* }\frac{\partial  \bar{y} + b \big(\pi - \pi^e \big) + s - y^* }{\partial \pi} = 0 \\
		&  \frac{1}{2} a 2 \big(\pi - \pi^* \big) 1  + \frac{1}{2} 2\big( \bar{y} + b \big(\pi - \pi^e \big) + s - y^*  \big) b = 0 \\
		&  a \big(\pi - \pi^* \big)  + \big( \bar{y} + b \big(\pi - \pi^e \big) + s - y^*  \big) b  = 0 \\
		&  a \pi - a\pi^*  +  b\bar{y} + b^2 \pi - b^2\pi^e + b s - by^*  = 0 \\
		& \big( a + b^2 \big) \pi  = a\pi^* - b\bar{y} + b^2\pi^e - b s + by^* \\
		\implies & \pi = \frac{a}{ a + b^2}\pi^* + \frac{b}{ a + b^2}\big(y^* - \bar{y}\big) + \frac{b^2}{ a + b^2}\pi^e - \frac{b}{ a + b^2} s 
\end{split}
\end{equation*}
\\
As the public has rational expectations, expectations about future inflation are correct: $\pi^e = E\big(\pi\big)$. We can plug in our expression for $\pi$ to find $\pi^e$. 
\begin{equation*}
\begin{split}
	\pi^e &= E\Big(\frac{a}{ a + b^2}\pi^* + \frac{b}{ a + b^2}\big(y^* - \bar{y}\big) + \frac{b^2}{ a + b^2}\pi^e - \frac{b}{ a + b^2} s \Big) \\
	\pi^e &= \frac{a}{ a + b^2}\pi^* + \frac{b}{ a + b^2}\big(y^* - \bar{y}\big) + \frac{b^2}{ a + b^2}\pi^e \\ 
	\pi^e & - \frac{b^2}{ a + b^2}\pi^e = \frac{a}{ a + b^2}\pi^* + \frac{b}{ a + b^2}\big(y^* - \bar{y}\big) \\
	\pi^e & \Big(\frac{a + b^2 - b^2}{a + b^2} \Big) = \frac{a}{ a + b^2}\pi^* + \frac{b}{ a + b^2}\big(y^* - \bar{y}\big) \\
	a\pi^e &= a\pi^* + b\big(y^* - \bar{y}\big) \\
	\implies \pi^e &= \pi^* + \frac{b}{a}\big(y^* - \bar{y}\big) > \pi^*\\
\end{split}
\end{equation*}
\\
We find that the public expects inflation to be above the socially optimal level if $y^* > \bar{y}$. The public knows that the monetary authority has inflation bias. We now find the level of inflation which is set by the monetary authority by plugging in $\pi^e$ into it's reaction function. 
\begin{equation*}
\begin{split}
	\pi &= \frac{a}{ a + b^2}\pi^* + \frac{b}{ a + b^2}\big(y^* - \bar{y}\big) + \frac{b^2}{ a + b^2}\pi^e - \frac{b}{ a + b^2} s \\
		&= \frac{a}{ a + b^2}\pi^* + \frac{b}{ a + b^2}\big(y^* - \bar{y}\big) + \frac{b^2}{ a + b^2} \Big( \pi^* + \frac{b}{a}\big(y^* - \bar{y}\big) \Big) - \frac{b}{ a + b^2} s  \\
		&= \frac{a}{ a + b^2}\pi^* + \frac{b}{ a + b^2}\big(y^* - \bar{y}\big) + \frac{b^2}{ a + b^2}  \pi^* +  \frac{b^2}{ a + b^2}  \frac{b}{a}\big(y^* - \bar{y}\big)  - \frac{b}{ a + b^2} s  \\
		&= \frac{a + b^2}{ a + b^2}\pi^* + \frac{b}{ a + b^2}\Big(1 + \frac{b^2}{a}\Big) \big(y^* - \bar{y}\big) - \frac{b}{ a + b^2} s  \\
\end{split}
\end{equation*}
\\
As $\frac{b}{ a + b^2}\Big(1 + \frac{b^2}{a}\Big) = \frac{b}{ a + b^2}\Big(\frac{a + b^2}{a}\Big) = \frac{b}{ a + b^2}\Big(\frac{a + b^2}{a}\Big) = \frac{b}{a}$, 
\begin{equation*}
\begin{split}
	\pi = \pi^* + \frac{b}{a} \big(y^* - \bar{y}\big) - \frac{b}{ a + b^2} s  \\
\end{split}
\end{equation*}
\\
There is no inflation surprises as the inflation expectations of the public were correct as $E\big( \pi \big) = \pi^e$. We now plug in  $\pi$ and $\pi^e$ into the aggregate supply equation to determine output $y$. 
\begin{equation*}
\begin{split}
	y &= \bar{y} + b \bigg(\pi^* + \frac{b}{a} \big(y^* - \bar{y}\big) - \frac{b}{ a + b^2} s - \Big( \pi^* + \frac{b}{a}\big(y^* - \bar{y}\big) \Big) \bigg) + s \\
		&= \bar{y} + b \bigg( - \frac{b}{ a + b^2} s  \bigg) + s \\
		&= \bar{y} - \frac{b^2}{ a + b^2} s  + \frac{ a + b^2}{ a + b^2} s \\
		\implies y &=  \bar{y} + \frac{ a }{ a + b^2} s
\end{split}
\end{equation*}
On average, output is equal to it's natural rate: $E(y) = \bar{y}$. We now calculate the expected loss of the policymaker by plugging in $\pi$ and $y$ into the loss function: 
\begin{equation*}
\begin{split}
	L_D &= \frac{1}{2} a \big( \pi - \pi^* \big)^2 + \frac{1}{2} \big( y - y^*  \big)^2 \\
		&= \frac{1}{2} a \Bigg(  \pi^* + \frac{b}{a} \big(y^* - \bar{y}\big) - \frac{b}{ a + b^2} s  - \pi^* \Bigg)^2 + \frac{1}{2} \Bigg( \bar{y} + \frac{ a }{ a + b^2} s - y^*  \Bigg)^2 \\
		&= \frac{1}{2} a \Bigg( \frac{b}{a} \big(y^* - \bar{y}\Big) - \frac{b}{ a + b^2} s  \Bigg)^2 + \frac{1}{2} \Bigg( \big(\bar{y}- y^*\big) + \frac{ a }{ a + b^2} s \Bigg)^2 \\
		&= \frac{1}{2} a \Bigg( \Big( \frac{b}{a} \big(y^* - \bar{y}\big)\Big)^2 - 2  \frac{b}{a} \big(y^* - \bar{y}\big) \frac{b}{ a + b^2} s + \Big( \frac{b}{ a + b^2} s \Big)^2 \Bigg) \\
		&   + \frac{1}{2} \Bigg( \big(\bar{y}- y^*\big)^2 + 2 \big(\bar{y}- y^*\big) \frac{ a }{ a + b^2} s + \Big(  \frac{ a }{ a + b^2} s \Big)^2 \Bigg) \\
		&= a  \frac{1}{2} \Big( \frac{b}{a} \big(y^* - \bar{y}\big)\Big)^2 -  \frac{b^2}{ a + b^2}\big(y^* - \bar{y}\big) s + \frac{a}{2}  \Big( \frac{b}{ a + b^2} s \Big)^2  \\
		&   + \frac{1}{2}  \big(\bar{y}- y^*\big)^2 +  \big(\bar{y}- y^*\big) \frac{ a }{ a + b^2} s +\frac{1}{2}  \Big(  \frac{ a }{ a + b^2} s \Big)^2  \\
\end{split}
\end{equation*}
Taking expectations: 
\begin{equation*}
\begin{split}
	E\big(L_D\big) &= E\bigg(  a  \frac{1}{2} \Big( \frac{b}{a} \big(y^* - \bar{y}\big)\Big)^2 -  \frac{b^2}{ a + b^2}\big(y^* - \bar{y}\big) s + \frac{a}{2}  \Big( \frac{b}{ a + b^2} s \Big)^2   \\
		&   + \frac{1}{2}  \big(\bar{y}- y^*\big)^2 +  \big(\bar{y}- y^*\big) \frac{ a }{ a + b^2} s +\frac{1}{2}  \Big(  \frac{ a }{ a + b^2} s \Big)^2 \bigg)\\
		&= E\bigg( a  \frac{1}{2} \Big( \frac{b}{a} \big(y^* - \bar{y}\big)\Big)^2 + \frac{a}{2}  \Big( \frac{b}{ a + b^2} s \Big)^2 + \frac{1}{2}  \big(\bar{y}- y^*\big)^2 +\frac{1}{2}  \Big(  \frac{ a }{ a + b^2} s \Big)^2   \bigg) \\
		&= E\bigg( a  \frac{1}{2} \frac{b^2}{a^2} \big(y^* - \bar{y}\big)^2 + \frac{a}{2} \frac{b^2}{ \big(a + b^2\big)^2} s^2 + \frac{1}{2}  \big(\bar{y}- y^*\big)^2 +\frac{1}{2} \frac{ a^2 }{ \big(a + b^2\big)^2} s^2   \bigg) \\
		&= \frac{1}{2} \frac{b^2}{a} \big(y^* - \bar{y}\big)^2 + \frac{a}{2} \frac{b^2}{ \big(a + b^2\big)^2} \sigma^2_s + \frac{1}{2}  \big(\bar{y}- y^*\big)^2 +\frac{1}{2} \frac{ a^2 }{ \big(a + b^2\big)^2} \sigma^2_s \\
		&= \Big(  \frac{1}{2} \frac{b^2}{a} + \frac{1}{2}\Big) \big(y^* - \bar{y}\big)^2 + \Bigg(\frac{a}{2} \frac{b^2}{ \big(a + b^2\big)^2} +\frac{1}{2} \frac{ a^2 }{ \big(a + b^2\big)^2}\Bigg) \sigma^2_s   \\
		&= \Big(  \frac{1}{2} \frac{b^2}{a} + \frac{1}{2}\Big) \big(y^* - \bar{y}\big)^2 + \Bigg(\frac{a}{2} \frac{b^2}{ \big(a + b^2\big)^2} +\frac{1}{2} \frac{ a^2 }{ \big(a + b^2\big)^2}\Bigg) \sigma^2_s  \\
		&=  \frac{1}{2} \Big( \frac{b^2}{a} + 1\Big) \big(y^* - \bar{y}\big)^2 + \frac{1}{2} \frac{ a }{ a + b^2} \Bigg(\frac{b^2}{ a + b^2} + \frac{ a^2 }{ a + b^2}\Bigg) \sigma^2_s  \\
		&=  \frac{1}{2} \Big( \frac{b^2}{a} + \frac{a}{a}\Big) \big(y^* - \bar{y}\big)^2 + \frac{1}{2} \frac{ a }{ a + b^2} \Bigg(\frac{b^2}{ a + b^2} + \frac{ a^2 }{ a + b^2}\Bigg) \sigma^2_s  \\
		\implies E\big(L_D\big) &=  \frac{1}{2} \frac{b^2 +a}{a} \big(y^* - \bar{y}\big)^2 + \frac{1}{2} \frac{ a }{ a + b^2}  \sigma^2_s  \\
\end{split}
\end{equation*}
\\

\section{Solutions to Inflation Bias}
There are multiple possible solutions for inflation bias: 
\begin{enumerate}
\item  Commitment: Abandon discretion and set monetary policy before inflation expectations $\pi_e$ are formed.\\
\\
We suppose that the policymaker (monetary authority) commits to $\pi = \pi_C$. Then, if the commitment is credible, for rational expectations $\pi^e = \pi_C$. We plug this into the aggregate supply equation to determine output: 
\begin{equation*}
\begin{split}
	y &= \bar{y} + b \big(\pi - \pi^e \big) + s  \\
	\implies y &= \bar{y} + b \big(\pi_C - \pi_C \big) + s \\
	\implies y &= \bar{y} + s
\end{split}
\end{equation*}
Under a commitment, the optimal policy yields $\pi = \pi^*$ and $y = \bar{y} + s$. We find the expected loss of the monetary authority: 
\begin{equation*}
\begin{split}
	L_C &= \frac{1}{2} a \big( \pi - \pi^* \big)^2 + \frac{1}{2} \big( y - y^*  \big)^2 \\
	\implies L &= \frac{1}{2} a \big( \pi_C - \pi^* \big)^2 + \frac{1}{2} \big( \bar{y} + s - y^*  \big)^2 \\
\end{split}
\end{equation*}
\\
Under the assumption that the level of inflation is set to be the socially optimal level: $\pi_C = \pi^*$, 
\begin{equation*}
\begin{split}
	 L_C &= \frac{1}{2} \big( \bar{y} + s - y^*  \big)^2 \\
		 &= \frac{1}{2} \Big( \big( \bar{y} - y^*  \big)^2 + 2 s \big( \bar{y} - y^*  \big) + s^2 \Big) \\
		 &= \frac{1}{2}  \big( \bar{y} - y^*  \big)^2 +  s \big( \bar{y} - y^*  \big) + \frac{1}{2} s^2  \\
\end{split}
\end{equation*}
\\
Taking expectations, 
\begin{equation*}
\begin{split}
	 E\big(L_C\big) &= E\Big( \frac{1}{2}  \big( \bar{y} - y^*  \big)^2 +  s \big( \bar{y} - y^*  \big) + \frac{1}{2} s^2 \Big) \\
	 \implies E\big(L_C\big) &=  \frac{1}{2}  \big( \bar{y} - y^*  \big)^2 + \frac{1}{2}\sigma_s^2
\end{split}
\end{equation*}
\\
We now derive the condition for which the expected welfare loss is lower under the commitment:
\begin{gather*}
	 E\big(L_C\big) < E\big(L_D\big) \\
	 \iff \frac{1}{2}  \big( \bar{y} - y^*  \big)^2 + \frac{1}{2}\sigma_s^2 < \frac{1}{2} \frac{b^2 +a}{a} \big(y^* - \bar{y}\big)^2 + \frac{1}{2} \frac{ a }{ a + b^2}  \sigma^2_s \\
	 \iff \big( \bar{y} - y^*  \big)^2 + \sigma_s^2 < \frac{b^2 +a}{a} \big(y^* - \bar{y}\big)^2 + \frac{ a }{ a + b^2}  \sigma^2_s \\
	 \Big( 1 -  \frac{ a }{ a + b^2} \Big)\sigma_s^2 < \Big( \frac{b^2 +a}{a} - 1 \Big) \big(y^* - \bar{y}\big)^2 \\
	 \frac{a + b^2 - a }{ a + b^2} \sigma_s^2 <  \frac{b^2 +a- a}{a}  \big(y^* - \bar{y}\big)^2  \\
	\implies  \frac{ b^2 }{ a + b^2} \sigma_s^2 <  \frac{b^2}{a}  \big(y^* - \bar{y}\big)^2  \\
\end{gather*}

If shocks are very important flexibility is more important. Compared to discretion, there is lower inflation loss, but higher output volatility loss. There is therefore a credibility-flexibility trade-off. The problem is that commitments may not be feasible or credible. Furthermore, rules limit flexibility and cannot incorporate all contingencies.

\item Delegation: Delegate monetary policy to central banker (CB) with different preferences.
\\
\\
We recall that for the following policy objective (with the aggregate supply equation plugged in): 
\begin{gather*}
	L = \frac{1}{2} a \big( \pi - \pi^* \big)^2 + \frac{1}{2} \big( \bar{y} + b \big(\pi - \pi^e \big) + s - y^*  \big)^2
\end{gather*}
\\
We obtained the following equilibrium inflation and output: 
\begin{gather*}
	\pi = \pi^* + \frac{b}{a} \big(y^* - \bar{y}\big) - \frac{b}{ a + b^2} s  \\
	y =  \bar{y} + \frac{ a }{ a + b^2} s
\end{gather*}
\\
Therefore, Rogoff (QJE 1985) argues that introducing a 'conservative', inflation-averse central banker with $a_{CB} > a$ reduces inflation bias, but also the stabilisation of supply shocks.
\\
\\
Svensson (AER 1997) argues that a central banker with a 'conservative' inflation target $\pi^* = \pi^* - b (y^* - \bar{y})$ eliminates inflation bias, without affecting output stabilization.
\\
\\
Blinder (JEP 1997) argues that 'responsible' central banker with output target $y_{CB}^* = \bar{y}$ eliminates inflation bias, without affecting output stabilization.
\\
\\
The problem is that having a conservative central banker leads to credibility-flexibility trade-off. Furthermore, a central banker?s preferences cannot be directly verified.
\\
\item  Incentive contracts: Give the central banker a personal incentives to prevent inflation bias (Walsh, AER 1995; Persson and Tabellini, CRCPP 1993).
\\
\\
We set an inflation penalty to eliminate inflation bias: 
\begin{gather*}
	L_I = L + b(y^* - \bar{y})\pi
\end{gather*}
\\
We plug in $L$:
\begin{gather*}
	L_I =  \frac{1}{2} a \big( \pi - \pi^* \big)^2 + \frac{1}{2} \big( \bar{y} + b \big(\pi - \pi^e \big) + s - y^*  \big)^2 + b(y^* - \bar{y})\pi
\end{gather*}
\\
Taking the FOC with respect to $\pi$ to obtain the reaction function of the monetary authority: 
\begin{equation*}
\begin{split}
	\frac{\partial L_I}{\partial \pi} &= \frac{1}{2} a 2 \big( \pi - \pi^* \big) + \frac{1}{2} 2  \big( \bar{y} + b \big(\pi - \pi^e \big) + s - y^*  \big) b + b(y^* - \bar{y}) = 0 \\
		&  a \big( \pi - \pi^* \big) + b \big( \bar{y} + b \big(\pi - \pi^e \big) + s - y^*  \big)  + b(y^* - \bar{y}) = 0 \\
		&  a \pi - a \pi^* + b  \bar{y} + b^2 \pi - b^2 \pi^e + b s - b y^*  +  b y^* - b \bar{y} = 0 \\
		&  \big(a + b^2) \pi = a\pi^* + b^2 \pi^e - b s  \\
		\implies &  \pi = \frac{a}{a + b^2}\pi^* + \frac{b^2}{a + b^2} \pi^e - \frac{b}{a + b^2} s  \\
\end{split}
\end{equation*}
\\
As the public has rational expectations, expectations about future inflation are correct: $\pi^e = E\big(\pi\big)$. We can plug in our expression for $\pi$ to find $\pi^e$. 
\begin{equation*}
\begin{split}
	\pi^e =  E\big( \pi \big) &= E\bigg( \frac{a}{a + b^2}\pi^* + \frac{b^2}{a + b^2} \pi^e - \frac{b}{a + b^2} s \bigg)  \\
	\implies \pi^e &=  \frac{a}{a + b^2}\pi^* + \frac{b^2}{a + b^2} \pi^e  \\
	\pi^e\Big(1 - \frac{b^2}{a + b^2}\Big) &=  \frac{a}{a + b^2}\pi^*  \\
	\pi^e\Big( \frac{a + b^2 - b^2}{a + b^2}\Big) &=  \frac{a}{a + b^2}\pi^*  \\
	a \pi^e &=  a \pi^*  \\
	\implies \pi^e &=  \pi^*  \\
\end{split}
\end{equation*}
\\
We can plug this into the reaction function of the monetary authority to obtain the level of inflation: 
\begin{equation*}
\begin{split}
	\pi &= \frac{a}{a + b^2}\pi^* + \frac{b^2}{a + b^2} \pi^* - \frac{b}{a + b^2} s  \\
		&= \frac{a + b^2}{a + b^2}\pi^* - \frac{b}{a + b^2} s  \\
	\implies \pi &= \pi^* - \frac{b}{a + b^2} s  \\
\end{split}
\end{equation*}
We find that we have eliminated the inflation bias as $E(\pi) = \pi^*$. The problem is that it is hard to implement financial penalties or firing requirements.
\\

\item Reputation: Repeated interaction makes policymaker behave better.\\
Trigger strategies with higher $\pi^e$ as punishment after inflation bias (Barro and Gordon, JME 1983). Problem: Multiple equilibria; trigger strategy arbitrary and hard to coordinate. Uncertainty about policymaker?s preferences and rational updating of $\pi^e$ makes inflation- prone policymakers mimic low-inflation types (Backus and Driffill, AER 1985; Barro, JME 1986).


\end{enumerate}


\section{Two-period monetary policy game with reputation}

The policymaker minimizes a social welfare losses function over two periods: 
\begin{gather*}
	\Lagr = l_1 + \delta l_2 \text{ with } 0 < \delta < 1
\end{gather*}
where 
\begin{gather*}
	l_t = \frac{1}{2} a \big( \pi_t - \pi^* \big)^2 - \big( y_t - \bar{y})	\text{ with } a>0
\end{gather*}
\\
The economy described by aggregate supply equation: 
\begin{gather*}
	y_t = \bar{y} + b \big( \pi_t - \pi_t^e \big) + s_t 
\end{gather*}
\\
Therefore, 
\begin{equation*}
\begin{split}
	\Lagr &= \frac{1}{2} a \big( \pi_1 - \pi^* \big)^2 - \big( y_1 - \bar{y} \big) + \delta \bigg( \frac{1}{2} a \big( \pi_2 - \pi^* \big)^2 - \big( y_2- \bar{y} \big)  \bigg) \\
		&= \frac{1}{2} a \big( \pi_1 - \pi^* \big)^2 - \Big( \bar{y} + b \big( \pi_1 - \pi_1^e \big) + s_t  - \bar{y} \Big) + \delta \bigg( \frac{1}{2} a \big( \pi_2 - \pi^* \big)^2 - \Big( \bar{y} + b \big( \pi_2 - \pi_2^e \big) + s_t  - \bar{y} \Big)  \bigg) \\
		&= \frac{1}{2} a \big( \pi_1 - \pi^* \big)^2 -  b \big( \pi_1 - \pi_1^e \big) - s_t   + \delta \bigg( \frac{1}{2} a \big( \pi_2 - \pi^* \big)^2 -  b \big( \pi_2 - \pi_2^e \big) - s_t     \bigg) \\
\end{split}
\end{equation*}
\\
The public does not observe $\pi^*$ but has rational expectations with prior $E\big( \pi^* \big) = \pi^*$.
\\
\\
In every period, the timing is as follows: 
\begin{enumerate}
	\item The public forms $\pi_t^e$. 
	\item $s_t$ is observed. 
	\item The policymaker set $\pi_t$.
\end{enumerate}
$ $ 
$ $ 
\\
People rationallyy use $\pi_1$ to form $\pi_2^e$, through postulate updating: $\pi_2^e = u_0 + u_1 \pi_1$. We plug this into $\Lagr$: 
\begin{equation*}
\begin{split}
	\Lagr &= \frac{1}{2} a \big( \pi_1 - \pi^* \big)^2 -  b \big( \pi_1 - \pi_1^e \big) - s_t   + \delta \bigg( \frac{1}{2} a \big( \pi_2 - \pi^* \big)^2 -  b \big( \pi_2 - (u_0 + u_1 \pi_1) \big) - s_t   \bigg) \\
	&= \frac{1}{2} a \big( \pi_1 - \pi^* \big)^2 -  b \big( \pi_1 - \pi_1^e \big) - s_t   + \delta \bigg( \frac{1}{2} a \big( \pi_2 - \pi^* \big)^2 -  b \big( \pi_2 - u_0 - u_1 \pi_1 \big) - s_t   \bigg) \\
\end{split}
\end{equation*}
\\
The policymaker minimises $\Lagr$ with respect to $\pi_1$ and $\pi_2$. We take FOCs: 
\begin{equation*}
\begin{split}
	\frac{\partial \Lagr}{\partial \pi_2} &= \delta \frac{1}{2} a 2 \big( \pi_2 - \pi^* \big) + \delta (-b) = 0 \\
	& \delta a \big( \pi_2 - \pi^* \big) = \delta b   \\
	& a \pi_2 - a \pi^* = b   \\
	& a \pi_2  = a \pi^* + b   \\
	& \pi_2  = \frac{a \pi^* + b}{a}   \\
	& \pi_2  = \frac{a \pi^* + b}{a}   \\
	\implies & \pi_2  = \pi^* + \frac{b}{a} > \pi^* \\
\end{split}
\end{equation*}
\\
\begin{equation*}
\begin{split}
	\frac{\partial \Lagr}{\partial \pi_1} &= \frac{1}{2} a 2 \big( \pi_1 - \pi^* \big) - b + \delta (-b) (-u_1) = 0 \\
		&  a  \pi_1 - a \pi^*  = b - \delta b u_1  \\
		&  a  \pi_1  =  a \pi^* +  b - \delta b u_1  \\
		&  \pi_1  =  \pi^* +  \frac{b}{a} - \delta \frac{b}{a} u_1  \\
		\implies &  \pi_1  =  \pi^* +  \frac{b}{a} \Big(1  - \delta u_1 \Big)  \\
\end{split}
\end{equation*}
\\
Using rational expectations, $\pi_2^e = E\big( \pi_2 \vert \pi_1 \big)$, hence: 
\begin{equation*}
\begin{split}
	\pi_2^e &= E\big( \pi_2 \vert \pi_1 \big) \\
		&= E\big(\pi^* + \frac{b}{a}  \vert \pi_1 \big) \\
		&= E\big(\pi^* \vert \pi_1 \big) + \frac{b}{a}  \\
\end{split}
\end{equation*}
\\
However, we know that $\pi_1  =  \pi^* +  \frac{b}{a} \Big(1  - \delta u_1 \Big)$, hence, $\pi^* = \pi_1 -  \frac{b}{a} \Big(1  - \delta u_1 \Big)$ and thereby: 
\begin{equation*}
\begin{split}
	\pi_2^e &= E\bigg(\pi_1 -  \frac{b}{a} \Big(1  - \delta u_1 \Big) \vert \pi_1 \bigg) + \frac{b}{a}  \\
		&= \pi_1 -  \frac{b}{a} \Big(1  - \delta u_1 \Big) + \frac{b}{a}  \\
		&= \pi_1 -  \frac{b}{a}  + \frac{b}{a}  \delta u_1 + \frac{b}{a}  \\
	\implies \pi_2^e &= \pi_1 + \frac{b}{a}  \delta u_1  \\
\end{split}
\end{equation*}
\\
We recall that $\pi_2^e = u_0 + u_1 \pi_1$. Therefore, by matching coefficients $u_1 = 1$ and $u_0 = \frac{b}{a}\delta u_1 =  \frac{b}{a}\delta$. Hence, plugging this in: 
\begin{equation*}
\begin{split}
	\pi_1  =  \pi^* +  \frac{b}{a} \Big(1  - \delta \Big) < \pi^* + \frac{b}{a} = \pi_2  \\
\end{split}
\end{equation*}
\\
There is a reputation effect which reduces the inflation bias in the first period. \\
\\
We note that due to rational expectations $\pi_t = E\big(\pi_t\big)$, therefore in every period $t$: 
\begin{equation*}
\begin{split}
	y_t &= \bar{y} + b \big( \pi_t - \pi_t^e \big) + s_t \\
		 &= \bar{y} + b \big(\pi_t^e - \pi_t^e \big) + s_t \\
		 &= \bar{y} + s_t \\
	\implies E\big(y_t\big) &= E\big( \bar{y} + s_t \big) = \bar{y}
\end{split}
\end{equation*}
\\
\\
\\
Transparency: Information disclosure improves central bank's incentives. Greater information disclosure by central bank allows private sector to infer central bank's intentions and adjust private sector inflation expectations accordingly, which imposes discipline on central bank (Geraats, EJ 2002).
\\
\\

\section{Monetary Policy Rules}

Prominent examples of monetary rules:

\begin{enumerate}
\item Friedman (1960, AER 1968): proposes that we set a constant money growth rate: $\hat{M}_t = k$. He argues that activist monetary policy is undesirable because of uncertainty about the state of the economy and the propensity to overreact due to policy lags.
	
\item Taylor (CRCSPP 1993) proposes that central banks abide by the following nominal interest rule: 
\begin{gather*}
	i_t = \bar{r} + \pi^* + 1.5\big( \pi_t - \pi^* \big) + 0.5\big( y_t - \bar{y} \big)
\end{gather*}
\\
	This rule satisfies the 'Taylor principle' $\frac{\partial i_t}{\partial \pi_t} > 1$ and corresponds to (approximately) optimal monetary policy for several models.
\end{enumerate}
$ $ 
$ $
\\
Problem with instrument rules (Svensson, JEL 2003):
\begin{enumerate}
\item The design of optimal monetary policy rule is complicated by uncertainty about the structure of economy.
\item The mechanical adoption of instrument rules is not optimal for all contingencies.
\end{enumerate}
$ $ 
$ $
\\
Svensson (EER 1997) describes optimal monetary policy in a simple dynamic backward-looking model. The monetary policymaker minimizes the expected value social welfare loss: 
\begin{gather*}
	\Lagr = E_t \bigg( \sum_{s=t}^{\infty} \delta^{s-t} L_t  \bigg) \text{ with } 0 < \delta < 1
\end{gather*}
where
\begin{gather*}
	L_t = \frac{1}{2} \big( \pi_t - \pi^* \big)^2
\end{gather*}
\\
The economy is described by the Phillips curve: 
\begin{gather*}
	\pi_{t+2} = \pi_{t+1} + \alpha \big( y_{t+1} - \bar{y} \big) + \epsilon_{t+2} \text{ with } \alpha > 0 
\end{gather*}
and the IS equation: 
\begin{gather*}
	y_{t+1} - \bar{y} = \rho\big(y_t - \bar{y} \big) - \beta \big( r_t - \bar{r} \big) + v_{t+1} \text{ with } 0 < \rho < 1, \beta >0 
\end{gather*}
where $\bar{r}$ is the natural real interest rate, $\epsilon_t \sim i.i.d. \big( 0, \sigma_v^2 \big)$. \\
\\
In every period $t$, the policymaker sets the interest rate $r_t$ after observing output $y_t$ and inflation $\pi_t$. \\
\\
We can rewrite $\Lagr$ by plugging in the Phillips curve and then the IS equation: 
\begin{equation*}
\begin{split}
	\Lagr &= E_t \Bigg( \sum_{s=t}^{\infty} \delta^{s-t} \frac{1}{2} \big( \pi_s - \pi^* \big)^2  \Bigg) \\
		&= E_t \Bigg( \sum_{s=t}^{\infty} \delta^{s-t} \frac{1}{2} \Big( \pi_{s-1} + \alpha \big( y_{s-1} - \bar{y} \big) + \epsilon_{s} - \pi^* \Big)^2  \Bigg) \\
		&= E_t \Bigg( \sum_{s=t}^{\infty} \delta^{s-t} \frac{1}{2} \bigg( \pi_{s-1} + \alpha \Big( \rho\big(y_{s-2} - \bar{y} \big) - \beta \big( r_{s-2} - \bar{r} \big) + v_{s-1} \Big) + \epsilon_{s} - \pi^* \bigg)^2  \Bigg) \\
		\\
		&=  \frac{1}{2} \bigg( \pi_{t-1} + \alpha \Big( \rho\big(y_{t-2} - \bar{y} \big) - \beta \big( r_{t-2} - \bar{r} \big) + v_{t-1} \Big) + \epsilon_{t} - \pi^* \bigg)^2 \\
		& +  \delta \frac{1}{2}  E_t \Bigg( \bigg( \pi_{t} + \alpha \Big( \rho\big(y_{t-1} - \bar{y} \big) - \beta \big( r_{t-1} - \bar{r} \big) + v_{t} \Big) + \epsilon_{t+1} - \pi^* \bigg)^2  \Bigg) \\
		& +  \delta^2 \frac{1}{2}  E_t \Bigg( \bigg( \pi_{t+1} + \alpha \Big( \rho\big(y_{t} - \bar{y} \big) - \beta \big( r_{t} - \bar{r} \big) + v_{t+1} \Big) + \epsilon_{t+2} - \pi^* \bigg)^2  \Bigg) \\
		& +  \delta^3 \frac{1}{2}  E_t \Bigg( \bigg( \pi_{t+2} + \alpha \Big( \rho\big(y_{t+1} - \bar{y} \big) - \beta \big( r_{t+1} - \bar{r} \big) + v_{t+2} \Big) + \epsilon_{t+3} - \pi^* \bigg)^2  \Bigg) \\
		& + ... 
\end{split}
\end{equation*}
\\
The policymaker maximises $\Lagr$ with respect to $r_t$. Taking FOCs: 
\begin{equation*}
\begin{split}
	\frac{\partial \Lagr }{\partial r_t} &= \delta^2 \frac{1}{2} \frac{\partial  E_t \Bigg( \bigg( \pi_{t+1} + \alpha \Big( \rho\big(y_{t} - \bar{y} \big) - \beta \big( r_{t} - \bar{r} \big) + v_{t+1} \Big) + \epsilon_{t+2} - \pi^* \bigg)^2  \Bigg)}{\partial  \bigg( \pi_{t+1} + \alpha \Big( \rho\big(y_{t} - \bar{y} \big) - \beta \big( r_{t} - \bar{r} \big) + v_{t+1} \Big) + \epsilon_{t+2} - \pi^* \bigg) }  \\
	& + \frac{\partial  \bigg( \pi_{t+1} + \alpha \Big( \rho\big(y_{t} - \bar{y} \big) - \beta \big( r_{t} - \bar{r} \big) + v_{t+1} \Big) + \epsilon_{t+2} - \pi^* \bigg) }{\partial r_t} = 0\\
	 \implies & \delta^2 \frac{1}{2} 2 E_t \bigg( \pi_{t+1} + \alpha \Big( \rho\big(y_{t} - \bar{y} \big) - \beta \big( r_{t} - \bar{r} \big) + v_{t+1} \Big) + \epsilon_{t+2} - \pi^* \bigg) \alpha(-\beta) = 0 \\
	 \implies & - \alpha \beta \delta^2 \bigg( E_t \big( \pi_{t+1} \big) + \alpha \Big( \rho\big(y_{t} - \bar{y} \big) - \beta \big( r_{t} - \bar{r} \big)  \Big) - \pi^* \bigg) = 0 \\
	 \implies &  E_t \big( \pi_{t+1} \big) + \alpha \Big( \rho\big(y_{t} - \bar{y} \big) - \beta \big( r_{t} - \bar{r} \big)  \Big) - \pi^*  = 0 \\
	 \\
	 \implies &  E_t \big( \pi_{t+1} \big) + \alpha \rho\big(y_{t} - \bar{y} \big) - \alpha \beta \big( r_{t} - \bar{r} \big)   = \pi^*   \\
\end{split}
\end{equation*}
\\
We recall that $\pi_{t+2} = \pi_{t+1} + \alpha \big( y_{t+1} - \bar{y} \big) + \epsilon_{t+2}$, hence 
\begin{equation*}
\begin{split}
	\pi_{t+1} &= \pi_{t} + \alpha \big( y_{t} - \bar{y} \big) + \epsilon_{t+1} \\
	\implies E_t\big(\pi_{t+1}\big) &= E_t\Big( \pi_{t} + \alpha \big( y_{t} - \bar{y} \big) + \epsilon_{t+1} \Big) \\
		&=\pi_{t} + \alpha \big( y_{t} - \bar{y} \big) 
\end{split}
\end{equation*}
We plug this into the FOC: 
\begin{equation*}
\begin{split}
	  \pi_{t} + \alpha \big( y_{t} - \bar{y} \big)  + \alpha \rho\big(y_{t} - \bar{y} \big) - \alpha \beta \big( r_{t} - \bar{r} \big)   = \pi^*   \\
\end{split}
\end{equation*}
\\
Rearranging gives us the Taylor rule: 
\begin{gather*}
	  \pi_{t} + \alpha y_{t} - \alpha \bar{y}   + \alpha \rho y_{t} -  \alpha \rho \bar{y}  - \alpha \beta  r_{t} + \alpha \beta \bar{r}   = \pi^*   \\
	   \alpha y_{t} - \alpha \bar{y} + \alpha \rho y_{t} - \alpha \rho \bar{y} - \alpha \beta  r_{t} + \alpha \beta \bar{r} = \pi^* - \pi_{t} \\
	   \alpha \Big( y_{t} -  \bar{y} +  \rho y_{t} -  \rho \bar{y} -  \beta  r_{t} +  \beta \bar{r} \Big) = \pi^* - \pi_{t} \\
	   \big( 1 + \rho ) y_{t}  - \big(1+ \rho \big) \bar{y} -  \beta \big( r_{t} -  \bar{r} \big)  = \frac{1}{\alpha} \big( \pi^* - \pi_{t} \big) \\
	  \big( 1 + \rho ) \Big(y_{t}  - \bar{y} \Big) -  \frac{1}{\alpha} \big( \pi^* - \pi_{t} \big) =  \beta \big( r_{t} -  \bar{r} \big) \\
	  \beta \big( r_{t} -  \bar{r} \big) = \big( 1 + \rho ) \Big(y_{t}  - \bar{y} \Big) +  \frac{1}{\alpha} \big( \pi_{t} - \pi^* \big) \\
	    r_{t} -  \bar{r}  =  \frac{1}{\alpha \beta} \big( \pi_{t} - \pi^* \big) + \frac{1}{\beta} \big( 1 + \rho ) \Big(y_{t}  - \bar{y} \Big) \\
	    \\
	   \implies r_{t}  =  \bar{r} + \frac{1}{\alpha \beta} \big( \pi_{t} - \pi^* \big) + \frac{1}{\beta} \big( 1 + \rho ) \Big(y_{t}  - \bar{y} \Big) \\
\end{gather*}
\\


\section{Monetary Policy in Practice}
Price stability requires a nominal anchor, depending on the monetary policy regime. Common monetary policy frameworks (see Mishkin, JME 1999): 
\begin{enumerate}
\item Monetary targeting: money growth $\hat{M}$ is a nominal anchor and policy instrument.
\item Inflation targeting: explicit inflation target $\pi^*$ is a nominal anchor; nominal interest rate is the policy instrument set by an independent, transparent and accountable central bank.
\item Exchange rate targeting: the exchange rate target $e^*$ is a nominal anchor; the interest rate and/or exchange rate interventions are used as a policy instrument. To remain in the target zone, there is either a fixed exchange rate or a currency board.
\end{enumerate}
The prevalent monetary policy instrument is the short-term nominal interest rate, except for exchange rate targeters.
\\
Number of countries for each monetary regime: 
\begin{center}
	\includegraphics[scale=0.4]{MPframe}
\end{center}
In practice, the solution to inflation bias is to delegate monetary policy to independent, transparent central bank and appoint responsible central bankers with long overlapping tenures and explicit inflation target. Taylor rules are a reasonable description of actual monetary policy, but there is no adoption of mechanical rules (besides exchange rate pegs).\\
\\
Monetary policy needs to be forward-looking as:
\begin{enumerate}
\item There are lags in monetary policy transmission.\\
Svensson (EER 1997) optimal policy under inflation targeting implies inflation-forecast targeting:
\begin{gather*}
	E\big(\pi_{t+2}\big) = \pi^*
\end{gather*}

\item The monetary policy instrument is short-term interest rate, but the economy is affected by longer-term interest rates.\\
Assuming the expectations theory of the term structure, the long-term interest rate is the average of expected future short-term interest rates:
\begin{gather*}
	i_{n,t} = \frac{1}{n} \Big( i_{1, t} + E\big(  i_{1, t+1} \big) + .. + E\big(  i_{1, t+n-1} \big)  \Big) 
\end{gather*}
where $i_n$ is the nominal interest rate (yield to maturity) for an n-period bond. Therefore, the central bank communications provide an additional policy tool to affect private sector expectations of future policy rates through the 'management of expectations' (Woodford, 2005).

\end{enumerate}













\chapter{Fiscal Policy}

There are three main sources of government financing: 

\begin{enumerate}
	\item Taxation (lump-sum or distortionary)
	
	\item Seignorage (printing money)
	
	\item Government debt (running budget deficit)
\end{enumerate}
$ $
$ $
We introduce the government budget constraint: 
\begin{equation*}
\begin{split}
	G_t + r B_t = T_t + S_t + D_t
\end{split}
\end{equation*}
where $G_t$ is government purchases, $T_t$ is tax revenue, $B_t$ is real government bonds at start of period $t$, $r$ is the real interest rate, $S_t = \frac{M_t - M_{t-1}}{P_t}$ is seignorage in period $t$, $D_t = B_{t+1} - B_t$ is the government budget deficit (and debt issue). We note that $\widetilde{D}_t = D_t - r B_t$ is the primary deficit.\\
\\
We can plug $D_t = B_{t+1} - B_t$ into the budget constraint: 
\begin{equation*}
\begin{split}
	G_t + r B_t = T_t + S_t + B_{t+1} - B_t
\end{split}
\end{equation*}
\\ 
We assume that governments exist forever. We derive the intertemporal budget constraint: 
\begin{equation*}
\begin{split}
	 r B_t + B_t &= T_t + S_t - G_t + B_{t+1} \\
	 (1+r) B_t &= T_t + S_t - G_t + B_{t+1} \\
	 B_t &= \frac{1}{1+r} \Big( T_t + S_t - G_t + B_{t+1}  \Big) \\
\end{split}
\end{equation*}
Plugging in $B_{t+1} = \frac{1}{1+r} \Big( T_{t+1} + S_{t+1} - G_{t+1} + B_{t+2}  \Big)$\\
\begin{equation*}
\begin{split}
	 B_t &= \frac{1}{1+r} \bigg( T_t + S_t - G_t +  \frac{1}{1+r} \Big( T_{t+1} + S_{t+1} - G_{t+1} + B_{t+2}  \Big)  \bigg) \\
	 	&= \frac{1}{1+r} \sum_{s=t}^{T} \frac{1}{(1+r)^{s-t}} \Big(T_s + S_s - G_s \Big) + \frac{1}{(1+r)^{T+1-t}}B_{T+1} \\
\end{split}
\end{equation*}
As $T \xrightarrow{} \infty$, $ \frac{1}{(1+r)^{T+1-t}}B_{T+1} \xrightarrow{} 0$ (the transversality condition) and we have: 
\begin{equation*}
\begin{split}
	 B_t = \frac{1}{1+r} \sum_{s=t}^{\infty} \frac{T_s}{(1+r)^{s-t}} + \frac{1}{1+r} \sum_{s=t}^{\infty} \frac{S_s}{(1+r)^{s-t}}  -  \frac{1}{1+r} \sum_{s=t}^{\infty} \frac{G_s}{(1+r)^{s-t}} \\
	 B_t  + \frac{1}{1+r} \sum_{s=t}^{\infty} \frac{G_s}{(1+r)^{s-t}}  = \frac{1}{1+r} \sum_{s=t}^{\infty} \frac{T_s}{(1+r)^{s-t}} + \frac{1}{1+r} \sum_{s=t}^{\infty} \frac{S_s}{(1+r)^{s-t}}  \\
	 B_t  + \frac{1}{1+r} \sum_{s=t}^{\infty} \frac{G_s}{(1+r)^{s-t}}  = \frac{1}{1+r} \bigg( \sum_{s=t}^{\infty} \frac{T_s}{(1+r)^{s-t}} + \sum_{s=t}^{\infty} \frac{S_s}{(1+r)^{s-t}} \bigg)  \\
\end{split}
\end{equation*}
Hence the intertemporal budget constraint:
\begin{equation*}
\begin{split}
	 (1+r)B_t + \sum_{s=t}^{\infty} \frac{G_s}{(1+r)^{s-t}}  = \sum_{s=t}^{\infty} \frac{T_s}{(1+r)^{s-t}} + \sum_{s=t}^{\infty} \frac{S_s}{(1+r)^{s-t}}  \\
\end{split}
\end{equation*}
\\ 
\\
\section{Seignorage}
Seignorage is real government revenues from printing money $M$. 
\begin{equation*}
\begin{split}
	S_t = \frac{M_t - M_{t-1}}{P_t} = \frac{M_t - M_{t-1}}{M_t}\frac{M_t}{P_t}
\end{split}
\end{equation*}
\\ 
An inflation tax is capital loss on real money balances due to inflation: 
\begin{equation*}
\begin{split}
	\Xi = \frac{M_{t-1}}{P_{t-1}} - \frac{M_{t-1}}{P_t} &= \frac{M_{t-1}}{P_{t-1}}\frac{P_t}{P_t} - \frac{M_{t-1}}{P_t}\frac{P_{t-1}}{P_{t-1}} \\
		&= \frac{P_t - P_{t-1}}{P_t}\frac{M_{t-1}}{P_{t-1}}
\end{split}
\end{equation*}
\\ 
In continuous time: 
\begin{equation*}
\begin{split}
	S = \frac{\dot{M}}{P} = \frac{\dot{M}}{M} \frac{M}{P}
\end{split}
\end{equation*}
and 
\begin{equation*}
\begin{split}
	\Xi = \pi \frac{M}{P} 
\end{split}
\end{equation*}
where $\dot{x} = \frac{d x}{dt}$ and $\pi = \frac{\dot{P}}{P}$. For constant $\frac{M}{P}$, $\frac{\dot{M}}{P} = \frac{\dot{P}}{P}$ so $S = \Xi$.
\\

\section{The Cagan Model}

We introduce the money market equilibrium of Cagan (1956). For $b>0$: 
\begin{equation*}
\begin{split}
	\ln{\frac{M}{P}} = a - b i + \ln{Y}
\end{split}
\end{equation*}
\\ 
The Fisher equation: 
\begin{equation*}
\begin{split}
	i = r + \pi^e 
\end{split}
\end{equation*}
where $i$ is the nominal interest rate, $r$ the real interest rate and $\pi_e$ expected inflation.\\
\\ 
We further assume superneutrality (i.e. change in money growth has no effect on real resource allocation):
\begin{equation*}
\begin{split}
	Y = \bar{Y}	\text{  and  } r = \bar{r}
\end{split}
\end{equation*}
as well as perfect foresight: 
\begin{equation*}
\begin{split}
	\pi^e = \pi
\end{split}
\end{equation*}
\\
To derive this result, we write the money market equilibrium condition in levels: 
\begin{equation*}
\begin{split}
	\ln{\frac{M}{P}} &= a - b i + \ln{Y} \\
	\frac{M}{P} &= e^{a - b i + \ln{Y}} \\
	 &= e^{a - b i} e^{\ln{Y}} \\
	 &= e^{a - b i} Y \\
	 &= e^{a - b (r + \pi^e)} \bar{Y} \\
	 &= e^{a - b \bar{r} - b \pi^e} \bar{Y} \\
	  &=  \bar{Y} e^{a - b \bar{r}} e^{ - b \pi^e}  \\
	  \implies \frac{M}{P} &= Ce^{ - b \pi^e} \text{  where  } C =   \bar{Y} e^{a - b \bar{r}}   \\
\end{split}
\end{equation*}
\\ 
In the steady-state, $\frac{M}{P}$ and $g_M = \frac{\dot{M}}{M}$ are constant, so $g_M = \pi = \pi^e$. We recall that $S = \frac{\dot{M}}{M} \frac{M}{P}$, hence in the steady-state:
\begin{equation*}
\begin{split}
	S_{ss} = \frac{\dot{M}}{M} \frac{M}{P} &= g_M \frac{M}{P} \\
		&= g_M Ce^{ - b \pi^e} \\
		&= g_M Ce^{ - b g_M} \\
\end{split}
\end{equation*}
\\
Seignorage is constant in the steady-state: $\dot{S}_{ss} = 0$. Hence, the government maximises seignorage revenue by taking FOCs of $S_{ss}$ with respect to money growth $g_M$: 
\begin{equation*}
\begin{split}
	\frac{\partial S_{ss}}{\partial g_M} &= \frac{\partial  g_M Ce^{ - b g_M}}{\partial g_M} = 0 \\
		 & 1Ce^{ - b g_M} + g_MC(-b)e^{ - b g_M} = 0 \\
		 & (1-b g_M)Ce^{ - b g_M} = 0 \\
		  & Ce^{ - b g_M} -  b g_M Ce^{ - b g_M} = 0 \\
		  & 1  =  b g_M \\ 
		  \implies & g_M = \frac{1}{b}
\end{split}
\end{equation*}
\\
We take the second order condition: 
\begin{equation*}
\begin{split}
	\frac{\partial^2 S_{ss}}{\partial g_M^2} &= \frac{\partial (1-b g_M)Ce^{ - b g_M} }{\partial g_M} \\
		&= -bCe^{ - b g_M} + (1-b g_M)(-b)Ce^{ - b g_M} \\
		&= -bCe^{ - b g_M} - b(1-b g_M)Ce^{ - b g_M}  < 0 \text{ for } g_M < \frac{1}{b}
\end{split}
\end{equation*}
hence we have a maximised steady-state seignorage at $g_M = \frac{1}{b}$, which yields: 
\begin{equation*}
\begin{split}
	S_{ss}^* &= g_M Ce^{ - b g_M} \\
		&=  \frac{1}{b}Ce^{ - b \frac{1}{b}} \\
		&=  \frac{1}{b}Ce^{0} \\
		&=  \frac{C}{b}
\end{split}
\end{equation*}
\\
According to Cagan?s money demand estimations, $\frac{1}{3} < b < \frac{1}{2}$. So, maximum seignorage for (continuous) growth rate $\pi = g_M = \frac{1}{b}$ with $b$ of 2 to 3.\\
\\
Note: Using $M(t) = e^g M^t M(0)$, which implies a growth factor of $\frac{M(1)}{M(0)} = e^{g_M}$, this gives us an annual price growth factor $\frac{P(1)}{P(0)}$ of $e^2 = 7.4$ to $e^3 = 20.1$, so $640\%$ to $1910\%$ inflation. However, empirically we observe a maximum seignorage is about $10\%$ of GDP.\\
\\
In conclusion, there is an 'Inflation tax Laffer curve' which puts a limit on seignorage $S = \frac{\dot{M}}{M} \frac{M}{P}$ in the steady state, as increases in money growth $\frac{\dot{M}}{M}$ raises the 'tax rate' but erodes the 'tax base' $\frac{M}{P}$.\\

\begin{center}
	\includegraphics[scale=0.35]{laffercurve}
\end{center}
\begin{center}
	\includegraphics[scale=0.35]{lafferemp}
\end{center}
$ $
$ $
\\

\section{Hyperinflation}

Hyperinflation is defined as inflation over $50\%$ per month (or $12, 875\%$ per year).\\
\\
The desired real money holdings is:
\begin{equation*}
\begin{split}
	m^*(t) = Ce^{-b\pi(t)}
\end{split}
\end{equation*}
\\
There is a gradual log-linear adjustment of real money holdings:
\begin{equation*}
\begin{split}
	\frac{d \ln{m(t)}}{d t} = \beta \Big(  \ln{m^*(t)} - \ln{m(t)}  \Big) \\
\end{split}
\end{equation*}
where $m(t) = \frac{M(t)}{P(t)}$ is real money holdings, and $0 < \beta < \frac{1}{b}$.\\
\\
We substitute $m^*(t) = Ce^{-b\pi(t)}$ into $\frac{d \ln{m(t)}}{d t} = \beta \Big(  \ln{m^*(t)} - \ln{m(t)}  \Big)$. We note that $\frac{d \ln{m}}{dt} = \frac{1}{m} \frac{dm}{dt} = \frac{\dot{m}}{m}$, hence: 
\begin{equation*}
\begin{split}
	\frac{d \ln{m}}{dt} = \beta \Big(  \ln{\big( Ce^{-b\pi(t)} \big)} - \ln{m(t)}  \Big) &=  \frac{\dot{m}}{m}\\
		\beta \Big(  \ln{C} -b\pi(t) - \ln{m(t)}  \Big) &=  \frac{\dot{m}}{m} \\
		\implies \dot{m} = \beta \Big(  \ln{C} -b\pi(t) - \ln{m(t)} &  \Big)m
\end{split}
\end{equation*}
In the steady-state, $\dot{m} = 0$ and therefore $m = \frac{M}{P}$ is constant. Therefore money growth equals inflation $g_M = \pi_{ss}$, hence: 
\begin{equation*}
\begin{split}
	0 &= \beta \Big(  \ln{C} - b\pi(t) - \ln{m(t)} \Big)m \\
	0 &=   \ln{C} - b\pi(t) - \ln{m(t)} \\
	 b\pi(t) &=  \ln{C} - \ln{m(t)} \\
	 \implies \pi(t) = \pi_{ss} &= \frac{1}{b} \big( \ln{C} - \ln{m(t)} \big) \\
\end{split}
\end{equation*}
\\
As in the Cagan model $S_{ss} = g_M m$, therefore $S_{ss} = \pi_{ss}m$, and hence, 
\begin{equation*}
\begin{split}
	S_{ss}(m) = \frac{1}{b}\big( \ln{C} - \ln{m} \big) m
\end{split}
\end{equation*}
\\
Furthermore, 
\begin{equation*}
\begin{split}
	b S_{ss}(m) &=\big( \ln{C} - \ln{m} \big) m \\
	b S_{ss}(m) - b\pi m &=\big( \ln{C} - \ln{m} \big) m  - b\pi m \\
	b S_{ss}(m) - b\pi m &=\big( \ln{C} - b\pi - \ln{m} \big) m  \\
	\beta \big(b S_{ss}(m) - b\pi m \big) &= \beta \big( \ln{C} - b\pi - \ln{m} \big) m  \\
	\implies \beta \big(b S_{ss}(m) - b\pi m \big) &= \dot{m}
\end{split}
\end{equation*}
\\
Then,
\begin{equation*}
\begin{split}
	\dot{m} &= \beta \big(  b S_{ss}(m) - b \pi m \big) \\
		&=  \beta b \big( S_{ss}(m) - \pi m \big) \\
		&= \beta b  \big( S_{ss}(m) -  (S - \dot{m})  \big)
\end{split}
\end{equation*}
using $m = \frac{M}{P}$, so $\frac{\dot{m}}{m} = g_M - \pi \implies \pi  = g_M - \frac{\dot{m}}{m}$, hence,
\begin{equation*}
\begin{split}
	\pi m &= \Big(g_M - \frac{\dot{m}}{m} \Big) m \\
		&= g_M m - \dot{m} \\
		\implies \pi m &= S - \dot{m}
\end{split}
\end{equation*}
\\
Rearranging, 
\begin{equation*}
\begin{split}
	\dot{m} &= \beta b  \big( S_{ss}(m) -  (S - \dot{m})  \big) \\
	\dot{m} &= \beta b  S_{ss}(m) -  \beta b S + \beta b \dot{m} \\
	\dot{m} - \beta b \dot{m}  &= \beta b  S_{ss}(m) -  \beta b S \\
	\big( 1- \beta b \big) \dot{m}  &= \beta b  \big(S_{ss}(m) -  S \big) \\ 
	\implies \dot{m}  &=  \frac{\beta b}{1- \beta b} \big(S_{ss}(m) -  S \big) \\ 
\end{split}
\end{equation*}
Let $S^* = \max_m S_{ss}(m)$ the maximum steady state seignorage. If seignorage needs $S>S^*$, then $S>S_{ss}(m)$ so $\dot{m} <0$. \\
While $m \xrightarrow{} 0$, $g_M = \frac{S}{m} \xrightarrow{} \infty$ and $\pi = g_M - \frac{\dot{m}}{m} \xrightarrow{} \infty$ (hyperinflation!). \\
\\
In conclusion, hyperinflation arises if seignorage needs exceed maximum steady state seignorage $S^*$.\\
\\
Empirically, we find a positive relationship between money growth and inflation: 
\begin{center}
	\includegraphics[scale=0.35]{hyEmp}
\end{center}
However, there is a slight negative relationship between fiscal deficits and seignorage: 
\begin{center}
	\includegraphics[scale=0.35]{hyper}
\end{center}
$ $ 
$ $
\\

\section{Ricardian Equivalence}

Ricardian equivalence: the financing of government expenditure by raising taxes or issuing government bonds has the same effect on the economy.
\begin{equation*}
\begin{split}
	\text{tax financing  } \iff \text{  debt financing} 
\end{split}
\end{equation*}
under the following assumption: 
\begin{enumerate}
\item Constant time path government purchases 
\item Representative agent with infinite horizon 
\item Perfect capital markets (ability to shift consumption between periods)
\item Lump-sum taxes (no distorsions)
\item Superneutrality
\end{enumerate}
This is because debt financing is equivalent to higher taxes in the future, so there is no change to lifetime utility.
$ $
$ $
\\
We recall the government budget constraint: 
\begin{equation*}
\begin{split}
	G_t + r B_t = T_t + S_t + D_t
\end{split}
\end{equation*}
\\
We consider a Ricardian experiment: the government introduces a tax cut $\Delta T < 0$ financed by increase in debt, so $\Delta D = -\Delta T > 0$. \\
\\
It is paid for by an increase in government debt: $\Delta B = - \Delta T > 0$, which requires raising taxes by 
\begin{equation*}
\begin{split}
	(1+r)^n \Delta B = -(1+r)^n \Delta T >0
\end{split}
\end{equation*}
in $n$ periods in the future. \\
\\
The present value of the tax change is the sum of the tax change today and the discounted tax change in the future: 
\begin{equation*}
\begin{split}
	\Delta T -\frac{(1+r)^n\Delta T}{(1+r)^n} = \Delta T - \Delta T = 0
\end{split}
\end{equation*}
so the consumer?s intertemporal budget constraint is not affected and and consumption is unchanged. The consumer saves the current tax cut to pay for higher future taxes. National saving $S = Y - C - G$ is not affected as the decrease in government saving is offset by an increase in private saving.\\ 
\\
We introduce the budget constraint of the representative consumer: 
\begin{equation*}
\begin{split}
	C_t + T_t + B_{t+1} + \frac{M_t}{P_t} = (1+r) B_t + \frac{M_{t-1}}{P_t} + Y_t  
\end{split}
\end{equation*}
where $B_{t+1}$ is bond holdings in period $t+1$ and $S_t = \frac{M_t - M_{t-1}}{P_t}$ is seignorage in period $t$. Rearranging, we obtain the intertemporal budget constraint of the representative consumer: 
\begin{equation*}
\begin{split}
	C_t + T_t + B_{t+1} + \frac{M_t}{P_t} &= (1+r) B_t + \frac{M_{t-1}}{P_t} + Y_t  \\
	C_t + T_t  + \frac{M_t- M_{t-1}}{P_t}  - Y_t  + B_{t+1} &= (1+r) B_t   \\
	C_t + T_t  + S_t  - Y_t  + B_{t+1} &= (1+r) B_t   \\
	B_t &= \frac{1}{1+r} \Big( C_t + T_t  + S_t  - Y_t  + B_{t+1} \Big) \\
\end{split}
\end{equation*}
Plugging in $B_{t+1} = \frac{1}{1+r} \Big( C_{t+1} + T_{t+1}  + S_{t+1}  - Y_{t+1}  + B_{t+2} \big)$: 
\begin{equation*}
\begin{split}
	B_t &= \frac{1}{1+r} \bigg( C_t + T_t  + S_t  - Y_t  + \frac{1}{1+r} \Big( C_{t+1} + T_{t+1}  + S_{t+1}  - Y_{t+1}  + B_{t+2} \Big) \bigg) \\
		&= \frac{1}{1+r} \sum_{s=T}^{T} \frac{1}{(1+r)^{s-t}} \Big( C_t + T_t  + S_t  - Y_t  \Big) + \frac{1}{(1+r)^{T+1-t}} B_{T+1}  \\
\end{split}
\end{equation*}
Letting $T \xrightarrow{} \infty$, and under the transversality condition $\lim_{T \to \infty} \frac{1}{(1+r)^{T+1-t}} B_{T+1} = 0$, we have:
\begin{equation*}
\begin{split}
	B_t &= \frac{1}{1+r} \sum_{s=T}^{\infty} \frac{1}{(1+r)^{s-t}} \Big( C_t + T_t  + S_t  - Y_t  \Big)   \\
	(1+r)B_t &= \sum_{s=T}^{\infty} \frac{1}{(1+r)^{s-t}} \Big( C_t + T_t  + S_t  - Y_t  \Big)   \\
	(1+r)B_t &= \sum_{s=T}^{\infty} \frac{C_t}{(1+r)^{s-t}} + \sum_{s=T}^{\infty} \frac{T_t}{(1+r)^{s-t}} + \sum_{s=T}^{\infty} \frac{S_t}{(1+r)^{s-t}} - \sum_{s=T}^{\infty} \frac{Y_t}{(1+r)^{s-t}}   \\
\end{split}
\end{equation*}
Therefore the intertemporal constraint of the representative consumer is: 
\begin{equation*}
\begin{split}
	(1+r)B_t +  \sum_{s=T}^{\infty} \frac{Y_t}{(1+r)^{s-t}} &= \sum_{s=T}^{\infty} \frac{C_t}{(1+r)^{s-t}} + \sum_{s=T}^{\infty} \frac{T_t}{(1+r)^{s-t}} + \sum_{s=T}^{\infty} \frac{S_t}{(1+r)^{s-t}}   \\
\end{split}
\end{equation*}
\\
\\
We prove Ricardian equivalence as follows. We recall the intertemporal government budget constraint: 
\begin{equation*}
\begin{split}
	 (1+r)B_t + \sum_{s=t}^{\infty} \frac{G_s}{(1+r)^{s-t}}  = \sum_{s=t}^{\infty} \frac{T_s}{(1+r)^{s-t}} + \sum_{s=t}^{\infty} \frac{S_s}{(1+r)^{s-t}}  \\
	  (1+r)B_t  = \sum_{s=t}^{\infty} \frac{T_s}{(1+r)^{s-t}} + \sum_{s=t}^{\infty} \frac{S_s}{(1+r)^{s-t}} - \sum_{s=t}^{\infty} \frac{G_s}{(1+r)^{s-t}} \\
\end{split}
\end{equation*}
which we substitute into the intertemporal constraint of the representative consumer: 
\begin{equation*}
\begin{split}
	 \sum_{s=t}^{\infty} \frac{T_s}{(1+r)^{s-t}} + \sum_{s=t}^{\infty} \frac{S_s}{(1+r)^{s-t}} - \sum_{s=t}^{\infty} \frac{G_s}{(1+r)^{s-t}}  +  \sum_{s=T}^{\infty} \frac{Y_t}{(1+r)^{s-t}} \\ 
	 = \sum_{s=T}^{\infty} \frac{C_t}{(1+r)^{s-t}} + \sum_{s=T}^{\infty} \frac{T_t}{(1+r)^{s-t}} + \sum_{s=T}^{\infty} \frac{S_t}{(1+r)^{s-t}} 
\end{split}
\end{equation*}
\begin{equation*}
\begin{split}
	  - \sum_{s=t}^{\infty} \frac{G_s}{(1+r)^{s-t}}  +  \sum_{s=T}^{\infty} \frac{Y_t}{(1+r)^{s-t}} &= \sum_{s=T}^{\infty} \frac{C_t}{(1+r)^{s-t}} \\
	  \\
	  \sum_{s=T}^{\infty} \frac{Y_t}{(1+r)^{s-t}} &= \sum_{s=T}^{\infty} \frac{C_t}{(1+r)^{s-t}} +  \sum_{s=t}^{\infty} \frac{G_s}{(1+r)^{s-t}} \\
\end{split}
\end{equation*}
\\
We find that government bonds $B_t$ are not a net wealth to the consumer!  \\
\\
We note that a fiscal contraction could be expansionary if consumers anticipate lower future government purchases that increase consumers lifetime resources and therefore consumption!\\
\\
\\

\section{Tax Smoothing}

We introduce the tax smoothing model of Barro (JPE 1979). \\
\\
The government minimises the expected value of the deadweight losses due to a distortionary tax $\tau$: 
\begin{equation*}
\begin{split}
	  L_t = \sum_{s=t}^{\infty} \frac{1}{(1+r)^{s-t}} \phi(\tau_s)Y_s
\end{split}
\end{equation*}
where $\phi(0) = 0$, $\phi'(.)>0$ and $\phi''(.)>0$. \\
\\
We introduce the intertemporal government budget constraint with income tax rate $\tau_s$ and no seignorage $S$:
\begin{equation*}
\begin{split}
	 (1+r)B_t + \sum_{s=t}^{\infty} \frac{G_s}{(1+r)^{s-t}}  = \sum_{s=t}^{\infty} \frac{\tau_s Y_s}{(1+r)^{s-t}}   \\
\end{split}
\end{equation*}
We assume that income $Y_s$ is exogenous and government expenditure $G_s$ is stochastic.\\
\\
We write the Lagrangian: 
\begin{equation*}
\begin{split}
	  \Lagr  = E_t \bigg( \sum_{s=t}^{\infty} \frac{1}{(1+r)^{s-t}} \phi(\tau_s)Y_s \bigg) + \lambda \bigg( (1+r)B_t + \sum_{s=t}^{\infty} \frac{G_s}{(1+r)^{s-t}} -\sum_{s=t}^{\infty} \frac{\tau_s Y_s}{(1+r)^{s-t}}   \bigg)
\end{split}
\end{equation*}
Taking FOCs with respect to the current tax rate $\tau_t$: 
\begin{equation*}
\begin{split}
	\phi'(\tau_t) Y_t - \lambda Y_t = 0 \implies  \phi'(\tau_t) Y_t = \lambda 
\end{split}
\end{equation*}
\\
Taking FOCs with respect to the tax rate in any future period $\tau_{t+n}$: 
\begin{equation*}
\begin{split}
	\frac{1}{(1+r)^n} E_t \Big( \phi'(\tau_{t+n}) \Big) Y_{t+n} - \lambda \frac{Y_{t+n}}{(1+r)^n} = 0\\
	\frac{1}{(1+r)^n} E_t \Big( \phi'(\tau_{t+n}) \Big) Y_{t+n} = \lambda \frac{Y_{t+n}}{(1+r)^n}\\
	\implies E_t \Big( \phi'(\tau_{t+n}) \Big) = \lambda
\end{split}
\end{equation*}
and finally the FOC with respect to $\lambda$, which is the budget constraint: 
\begin{equation*}
\begin{split}
	 (1+r)B_t + \sum_{s=t}^{\infty} \frac{G_s}{(1+r)^{s-t}}  = \sum_{s=t}^{\infty} \frac{\tau_s Y_s}{(1+r)^{s-t}}   \\
\end{split}
\end{equation*}
Combining the first two FOCs: 
\begin{equation*}
\begin{split}
	 \phi'(\tau_t) Y_t = E_t \Big( \phi'(\tau_{t+n}) \Big) 
\end{split}
\end{equation*}
\\
We find that the government engages in tax-rate smoothing. \\
\\
If in addition, we remove uncertainty about future income $Y_t$, hence: 
\begin{equation*}
\begin{split}
	 \phi'(\tau_t) Y_t =  \phi'(\tau_{t+n})
\end{split}
\end{equation*}
\\
We introduce the quadratic loss function $\phi(\tau) = \frac{1}{2} \kappa \tau^2$, with first derivative  $\phi'(\tau) =\kappa \tau$. Hence: 
\begin{equation*}
\begin{split}
	 \phi'(\tau_t) Y_t &= E_t \Big( \phi'(\tau_{t+n}) \Big) \\
	 \implies \kappa \tau_t &= E_t \Big( \kappa \tau_{t+n} \Big) \\
	 \implies  \tau_t &= E_t \Big(  \tau_{t+n} \Big) \\
\end{split}
\end{equation*}
We find that the current tax rate equals the expected future rate.\\
\\
To solve for the current tax rate $\tau_t$, we take expectations in the budget constraint: 
\begin{equation*}
\begin{split}
	 E_t\bigg( (1+r)B_t + \sum_{s=t}^{\infty} \frac{G_s}{(1+r)^{s-t}} \bigg) =  E_t\bigg(  \sum_{s=t}^{\infty} \frac{\tau_s Y_s}{(1+r)^{s-t}} \bigg)  \\
	 (1+r)B_t + \sum_{s=t}^{\infty} \frac{E_t\big(G_s\big)}{(1+r)^{s-t}}  = \sum_{s=t}^{\infty} \frac{E_t\big( \tau_s \big) Y_s}{(1+r)^{s-t}}   \\
\end{split}
\end{equation*}
Using $ \tau_t = E_t \Big(  \tau_{t+n} \Big)$ for every $s = t, t+1, ...$ : 
\begin{equation*}
\begin{split}
	 (1+r)B_t + \sum_{s=t}^{\infty} \frac{E_t\big(G_s\big)}{(1+r)^{s-t}}  = \sum_{s=t}^{\infty} \frac{\tau_s Y_s}{(1+r)^{s-t}}   \\
\end{split}
\end{equation*}
Since the tax rate $\tau_t$ is the same in every period, we can take it out of the summation and solve for $\tau_t$: 
\begin{equation*}
\begin{split}
	 (1+r)B_t + & \sum_{s=t}^{\infty} \frac{E_t\big(G_s\big)}{(1+r)^{s-t}}  = \tau_s \sum_{s=t}^{\infty} \frac{Y_s}{(1+r)^{s-t}}   \\
	 \implies \tau_s &= \frac{ (1+r)B_t + \sum_{s=t}^{\infty} \frac{E_t\big(G_s\big)}{(1+r)^{s-t}} }{ \tau_s \sum_{s=t}^{\infty} \frac{Y_s}{(1+r)^{s-t}} }  \\
\end{split}
\end{equation*}
\\
We find that the tax rate $\tau_t$ depends on the present value of expected government purchases and output. 
\\
\section{Fiscal Policy in Practice}

\begin{enumerate}
\item For some developing countries, seignorage important (and inflation high), but in advanced economies, seignorage limited by independent central bank with low inflation target.
\item Ricardian equivalence typically fails (due to credit constraints and lack of foresight). However, consumers may act Ricardian for short-term tax cuts
aimed at stimulating economy, making them ineffective!
\item Tax smoothing contributes to countercyclical fiscal policy (through 'automatic stabilizer', reducing tax revenues in recession). However, fiscal policy typically driven by political influences.
\end{enumerate}






\chapter{Exchange Rates}

\section{Overview}


The spot exchange rate $S_t$ is domestic price of foreign currency at time $t$ A home nominal appreciation: $S_t \downarrow$ (depreciation $S_t \uparrow$).\\
\\
The forward exchange rate $F_t$ is the domestic price at time $t$ for foreign currency at time $t + 1$.\\
\\
Assuming international capital mobility, risk neutrality and rational expectations, arbitrage ensures the following:\\
\begin{enumerate}
\item Covered interest parity
\begin{equation*}
\begin{split}
	1 + i_t = \big( 1 + i_t^* \big) \frac{F_t}{S_t}
\end{split}
\end{equation*}
where $i_t$ is the domestic interest rate, $i_t^*$ is the foreign interest rate, and $\frac{F_t}{S_t}$ is the expected proportional change in FX rate.

\item Uncovered interest parity 
\begin{equation*}
\begin{split}
	1 + i_t = \big( 1 + i_t^* \big) E_t \bigg( \frac{S_{t+1}}{S_t} \bigg)
\end{split}
\end{equation*}
where $E_t \bigg( \frac{S_{t+1}}{S_t} \bigg)$ is the expected proportional chance in FX rate. 
	
\end{enumerate}
$ $ 
\\
These equations hold because if they don't then there are arbitrage opportunities. \\
\\
\\
We define the real exchange rate: 
\begin{equation*}
\begin{split}
	\xi = \frac{P}{S P^*}
\end{split}
\end{equation*}
where $P$ is the domestic price level and $S P*$ is the currency adjusted foreign price level. $\xi$ is the relative price of goods at home vs abroad. A home real appreciation is $\xi \uparrow$ and a depreciation is $\xi \downarrow$. \\
\\
The law of one price: for any good i, 
\begin{equation*}
\begin{split}
	P_i = S P_i^* 
\end{split}
\end{equation*}
There are no currency adjusted price differences. For example arbitrage in the market for iPads: price differences lead to arbitrage and convergence in prices. \\
\\
Purchasing power parity (PPP): \\
$    $-absolute PPP: $\xi = 1$, 
\begin{equation*}
\begin{split}
	P = S P^*
\end{split}
\end{equation*}
\\
$    $-relative PPP: $\xi = constant$, 
\begin{equation*}
\begin{split}
	\dot{P} = \dot{S} + \dot{P}^*
\end{split}
\end{equation*}
\\
Empirical evidence: \\
\begin{enumerate}
\item CIP holds for advanced economies without capital controls, but UIP fails.
\item Nominal exchange rate very volatile and hard to predict.
Meese and Rogoff (JIE 1983):\\
 For short forecast horizons (1-12 months), random walk outperforms macro models, even if structural forecasts are based on actual realization of explanatory variables.\\
  For longer forecast horizons (2-3 years), macro models do outperform random walk.
\item Large and persistent deviations from law of one price (potential explanations: transport costs, tariffs and other trade barriers, imperfect competition with ?pricing to market?)
\item Real exchange rate close to random walk. However, for long sample periods and/or large deviations from PPP, mean reversion occurs.
PPP puzzle: Assuming deviations from PPP caused by nominal rigidities, convergence to PPP should be fast. However, half-life of PPP deviations appears 3-5 years, suggesting convergence extremely slow.
\item Real exchange rate increasing in per capita income (Balassa 1964).
\end{enumerate}
$ $ 
$ $

\section{Flexible Price Monetary Model}

We define the money market equilibrium:
\begin{equation*}
\begin{split}
	m_t - p_t = \kappa y_t - \theta i_t  \text{   with } \kappa, \theta >0 
\end{split}
\end{equation*}
where $m_t - p_t$ is the real money supply, $y_t$ is output and $i_t$ is the interest rate. \\
\\
Purchasing power parity: 
\begin{equation*}
\begin{split}
	p_t = s_t + p_t^*
\end{split}
\end{equation*}
\\
Uncovered interest parity: 

\begin{equation*}
\begin{split}
	1 + i_t &= \big( 1 + i_t^* \big) E_t \bigg( \frac{S_{t+1}}{S_t} \bigg) \\
	\implies \ln{\big(1 + i_t\big)} &= \ln{\bigg(  \big( 1 + i_t^* \big) E_t \bigg( \frac{S_{t+1}}{S_t} \bigg)  \bigg)} \\
		&= \ln{\big( 1 + i_t^* \big)} + \ln{ E_t \bigg( \frac{S_{t+1}}{S_t} \bigg)  } \\
		&= \ln{\big( 1 + i_t^* \big)} + \ln{ E_t \big( S_{t+1} \big)  - \ln{S_{t}}  } \\
\end{split}
\end{equation*}
Taking first order Taylor series approximations $\ln{\big(1 + i_t\big)} \approx i_t$, hence:
\begin{equation*}
\begin{split}
	i_t & \approx  i_t^* + \ln{ E_t \big( S_{t+1} \big)  - \ln{S_{t}}  } \\
	i_t & \approx  i_t^* + E_t \big( s_{t+1} \big)  - s_t  \\
\end{split}
\end{equation*}
where $x = \ln{X}$ with $X \in \{ M, P, Y, S \}$. Therefore $s_t = \ln{S_t}$. \\
\\
\begin{equation*}
\begin{split}
	p_t &= s_t + p_t^* \\
	\implies s_t &= p_t - p_t^* \\
\end{split}
\end{equation*}
\\
As $m_t - p_t = \kappa y_t - \theta i_t$, then $p_t  = m_t - \kappa y_t + \theta i_t $, which we plug in: 
\begin{equation*}
\begin{split}
	s_t &= m_t - \kappa y_t + \theta i_t - \big(  m_t^* - \kappa y_t^* + \theta i_t^* \big) \\
		&= \big( m_t - m_t^* \big) - \kappa \big( y_t -  y_t^* \big) + \theta \big( i_t - i_t^* \big)  \\
		&= k_t  + \theta \Big( E_t \big( s_{t+1} \big)  - s_t \Big)  \text{ using }  i_t  \approx  i_t^* + E_t \big( s_{t+1} \big)  - s_t \\
\end{split}
\end{equation*}
\\
where $k_t = \big( m_t - m_t^* \big) - \kappa \big( y_t -  y_t^* \big)$ measures inverse macroeconomic fundamentals. Rearranging:
\begin{equation*}
\begin{split}
	s_t &= k_t  + \theta  E_t \big( s_{t+1} \big) - \theta s_t  \\
	s_t + \theta s_t &= k_t  + \theta  E_t \big( s_{t+1} \big) \\
	s_t \big( 1+ \theta \big) &= k_t  + \theta  E_t \big( s_{t+1} \big) \\
	s_t  &= \frac{ k_t  + \theta  E_t \big( s_{t+1} \big) }{ 1+ \theta} \\
	\implies s_t  &= \frac{1}{1+\theta} k_t + \frac{\theta}{1+\theta} E_t \big( s_{t+1} \big) \\
\end{split}
\end{equation*}
\\
Forward substitution: 
\begin{equation*}
\begin{split}
	 s_{t+1}  &= \frac{1}{1+\theta} k_{t+1} + \frac{\theta}{1+\theta} E_{t+1} \big( s_{t+2} \big) \\
\end{split}
\end{equation*}
hence, 
\begin{equation*}
\begin{split}
	s_t  &= \frac{1}{1+\theta} k_t + \frac{\theta}{1+\theta} E_t \bigg( \frac{1}{1+\theta} k_{t+1} + \frac{\theta}{1+\theta} E_{t+1} \big( s_{t+2} \big) \bigg) \\
	&= \frac{1}{1+\theta} k_t + \frac{\theta}{(1+\theta)^2} E_t \big( k_{t+1} \big) + \frac{\theta^2}{(1+\theta)^2} E_t  \big( s_{t+2} \big) \\
	&= \frac{1}{1+\theta} k_t + \frac{\theta}{(1+\theta)^2} E_t \big( k_{t+1} \big) + \frac{\theta^2}{(1+\theta)^2} E_t \bigg( \frac{1}{1+\theta} k_{t+2} + \frac{\theta}{1+\theta} E_{t+2} \big( s_{t+3} \big) \bigg) \\
	&= \frac{1}{1+\theta} k_t + \frac{\theta}{(1+\theta)^2} E_t \big( k_{t+1} \big) + \frac{\theta^2}{(1+\theta)^3} E_t \big( k_{t+2} \big) + \frac{\theta^3}{(1+\theta)^3} E_t \big( s_{t+3} \big) \\
\end{split}
\end{equation*}
We have found the following recursion: 
\begin{equation*}
\begin{split}
	s_t  &= \sum_{\tau=t}^{T} \bigg( \frac{\theta}{1+\theta} \bigg)^{\tau - t} \frac{1}{1+\theta} E_t \big( k_\tau  \big) + \bigg( \frac{\theta}{1+\theta} \bigg)^{T+ 1 -\tau} E_t \big( s_{T+1}  \big)
\end{split}
\end{equation*}
Assuming the no-bubble condition $\lim_{T \to \infty} \bigg( \frac{\theta}{1+\theta} \bigg)^{T+ 1 -\tau} E_t \big( s_{T+1}  \big) = 0$, as $t \to \infty$, we obtain: 
\begin{equation*}
\begin{split}
	s_t  &= \sum_{\tau=t}^{\infty} \bigg( \frac{\theta}{1+\theta} \bigg)^{\tau - t} \frac{1}{1+\theta} E_t \big( k_\tau  \big)
\end{split}
\end{equation*}
\\
\\
In the steady-state, $k_\tau = \bar{k}$, hence
\begin{equation*}
\begin{split}
	\bar{s}  &= \sum_{\tau=t}^{\infty} \bigg( \frac{\theta}{1+\theta} \bigg)^{\tau - t} \frac{1}{1+\theta} \bar{k} \\
		&=  \frac{1}{1+\theta} \bar{k} \sum_{\tau=t}^{\infty} \bigg( \frac{\theta}{1+\theta} \bigg)^{\tau - t} \\
		&=  \frac{1}{1+\theta} \bar{k} \frac{1}{1-  \frac{\theta}{1+\theta}} \\
		&=  \frac{1}{1+\theta} \frac{1}{\frac{1+\theta - \theta}{1+\theta}} \bar{k} \\
		&=  \frac{1}{1+\theta} \frac{1}{\frac{1}{1+\theta}} \bar{k} \\
		&=  \frac{1}{1+\theta} (1+\theta) \bar{k} \\
		\implies \bar{s} &=  \bar{k} \\
\end{split}
\end{equation*}
\\
\\
Suppose there is an unexpected announcement at $t = 0$ of a permanent change in fundamentals from $\bar{k}$ to $\bar{k}'$ at time $t = T$ in the future. \\
\\
Hence, using $s_t  = \sum_{\tau=t}^{\infty} \bigg( \frac{\theta}{1+\theta} \bigg)^{\tau - t} \frac{1}{1+\theta} E_t \big( k_\tau  \big)$, we obtain:
\begin{equation*}
\begin{split}
	s_t  &= \sum_{\tau=t}^{T-1} \bigg( \frac{\theta}{1+\theta} \bigg)^{\tau - t} \frac{1}{1+\theta} \bar{k} + \sum_{\tau=T}^{\infty} \bigg( \frac{\theta}{1+\theta} \bigg)^{\tau - t} \frac{1}{1+\theta} \bar{k}' \\
	&=  \frac{1}{1+\theta} \bar{k}  \frac{1 - \Big( \frac{\theta}{1+\theta} \Big)^{T - 1 - t + 1}}{1 - \frac{\theta}{1+\theta}}  + \frac{1}{1+\theta}  \sum_{\tau=T}^{\infty} \bigg( \frac{\theta}{1+\theta} \bigg)^{\tau - t} \bar{k}' \\
\end{split}
\end{equation*}
\\
using the fact that $\sum_{\tau = 0}^{T} a^\tau = \frac{1 - a^{T+1} }{1-a} $\\
\begin{equation*}
\begin{split}
	s_t  &=  \frac{1}{1+\theta} \frac{1 - \Big( \frac{\theta}{1+\theta} \Big)^{T - t}}{1 - \frac{\theta}{1+\theta}} \bar{k} + \frac{1}{1+\theta} \bigg( \sum_{\tau=t}^{\infty} \bigg( \frac{\theta}{1+\theta} \bigg)^{\tau - t} -  \sum_{\tau=t}^{T-1} \bigg( \frac{\theta}{1+\theta} \bigg)^{\tau - t} \bigg)  \bar{k}' \\
	 &=  \frac{1}{1+\theta}  \frac{1 - \Big( \frac{\theta}{1+\theta} \Big)^{T - t}}{\frac{1+\theta - \theta}{1+\theta}} \bar{k} + \frac{1}{1+\theta} \bigg( \frac{1}{1-  \frac{\theta}{1+\theta}} -  \frac{1 - \Big( \frac{\theta}{1+\theta}  \Big)^{T-t} }{1 - \frac{\theta}{1+\theta} } \bigg)  \bar{k}' \\
	 &=  \frac{1}{1+\theta}  \frac{1 - \Big( \frac{\theta}{1+\theta} \Big)^{T - t}}{\frac{1}{1+\theta}} \bar{k} + \frac{1}{1+\theta} \bigg( \frac{1}{ \frac{1+\theta - \theta}{1+\theta}} -  \frac{1- \Big( \frac{\theta}{1+\theta}  \Big)^{T-t} }{\frac{1+\theta - \theta}{1+\theta} } \bigg)  \bar{k}' \\
	 &=  \frac{1}{1+\theta} \big(1+\theta\big) \bigg(1 - \Big( \frac{\theta}{1+\theta} \Big)^{T - t} \bigg) \bar{k} + \frac{1}{1+\theta} \bigg( \frac{1}{\frac{1}{1+\theta}} -  \frac{1- \Big( \frac{\theta}{1+\theta}  \Big)^{T-t} }{\frac{1}{1+\theta} } \bigg)  \bar{k}' \\
	 &=  \bigg(1 - \Big( \frac{\theta}{1+\theta} \Big)^{T - t} \bigg) \bar{k} + \frac{1}{1+\theta} \bigg( \big(1+\theta \big) -  \big(1+\theta \big) \bigg( 1 - \Big( \frac{\theta }{1+\theta}  \Big)^{T-t} \bigg)  \bigg)  \bar{k}' \\
	 &=  \bigg(1 - \Big( \frac{\theta}{1+\theta} \Big)^{T - t} \bigg) \bar{k} + \bigg( 1 -  \bigg( 1 - \Big( \frac{\theta }{1+\theta}  \Big)^{T-t} \bigg)  \bigg)  \bar{k}' \\
	 &=  \bigg(1 - \Big( \frac{\theta}{1+\theta} \Big)^{T - t} \bigg) \bar{k} + \bigg( 1  -   1 + \Big( \frac{\theta }{1+\theta}  \Big)^{T-t}   \bigg)  \bar{k}' \\
	 &=  \bar{k} - \Big( \frac{\theta}{1+\theta} \Big)^{T - t}\bar{k}  +  \Big( \frac{\theta }{1+\theta}  \Big)^{T-t}  \bar{k}' \\
	 &=  \bar{k} + \big( \bar{k}' -  \bar{k} \big) \Big( \frac{\theta}{1+\theta} \Big)^{T - t}  \\
\end{split}
\end{equation*}
\\
\\
Hence, an anticipated change in fundamentals $k$ has immediate effect on the nominal exchange rate $s$.
\\
\\
\\
For fixed exchange rates $\forall \tau, s_\tau = \bar{s}$. The country restricts monetary policy $m_\tau$ such that $k_\tau = \bar{s}$. It effectively imports monetary policy from abroad: $i = i^*$ and $\pi = \pi^*$. \\
\\
Note: International capital mobility requires a macroeconomic policy choice of either fixed exchange rate or independent domestic monetary policy. There is an international macroeconomic policy trilemma.
 
\section{Harrod-Balassa-Samuelson Effect}

There are two sectors in the economy: tradable goods (T) and nontradables (NT). The production function is for each sector is : 
\begin{equation*}
\begin{split}
	Y_i &= A_i L_i^{\theta_i} K_i^{1-\theta_i} \text{  for  }  i = T, N \\
\end{split}
\end{equation*}
where $0 < \theta_T \leq \theta_N <0$, as nontradable services are more labour intensive than physical tradable goods.\\
\\
There are two factors of production:
\begin{enumerate}
	\item Labor $L$ which is mobile between sectors: 
		\begin{equation*}
		\begin{split}
			L = L_T + L_N
		\end{split}
		\end{equation*}
		Therefore, by arbitrage wages are equal in both sectors:
		\begin{equation*}
		\begin{split}
			W = W_T + W_N
		\end{split}
		\end{equation*}
		where $W$ is the real wage.
	\item Capital $K$ which is mobile between sectors and countries, hence the real interest R (in terms of tradables) is the same in both sectors
		\begin{equation*}
		\begin{split}
			R = R_T + R_N
		\end{split}
		\end{equation*}
		
\end{enumerate}
$ $
$ $
There is perfect competition in goods and factor markets. \\
\\
We denote the aggregate price index:
\begin{equation*}
\begin{split}
	P = P_T^\gamma P_N^{1-\gamma}
\end{split}
\end{equation*}
where $0<\gamma<1$ is the share of the traded sector. We let tradables be the numeraire: $P_T = 1$. We express the price of nontraded in terms of $P_t$.\\
\\
Firms maximize the present value of lifetime real profits (in terms of tradables):
\begin{equation*}
\begin{split}
	\pi_{i, t} = \sum_{\tau=t}^\infty \bigg(\frac{1}{1+R}\bigg)^{\tau - t} \Big( P_{i, \tau} Y_{i, \tau} - W_\tau L_{i, \tau} - I_{i, \tau}  \Big)
\end{split}
\end{equation*}
\\
Profit in each period is the difference between revenue and the sum of costs and investment.\\
Capital accumulation (from tradables) is reversible (capital can be sold - i.e. negative investment) and there is no depreciation: 
\begin{equation*}
\begin{split}
	K_{i, t+1} = K_{i, t} + I_{i, t}	
\end{split}
\end{equation*}
where $K_{i, t}>0$ and investment $I_{i, t}$ \\
\\
The law of one price for tradables, by a simple arbitrage argument: 
\begin{equation*}
\begin{split}
	P_T = S P_T^*
\end{split}
\end{equation*}
\\
We assume a small, open economy: 
\begin{equation*}
\begin{split}
	R = \bar{R} \text{  and  } P_T^* = \bar{P_T^*}
\end{split}
\end{equation*}
where $\bar{R}$ is the average return across countries and $\bar{P_T^*}$ is the average price. \\
\\
We can plug the production function and law of motion of capital into the profit equation: 
\begin{equation*}
\begin{split}
	\pi_{i, t} = \sum_{\tau=t}^\infty \bigg(\frac{1}{1+R}\bigg)^{\tau - t} \Big( P_{i, \tau} A_{i, \tau} L_{i, \tau}^{\theta_i} K_{i, \tau}^{1-\theta_i} - W_\tau L_{i, \tau} - \big(K_{i, t+1} - K_{i, t} \big)  \Big)
\end{split}
\end{equation*}
\\
The firm maximises profits with respect to labour used $L_{i,t}$ and the amount of investment in capital $K_{i, t+1}$. The FOC w.r.t. $L_{i,t}$ is: 
\begin{equation*}
\begin{split}
	\bigg(\frac{1}{1+R}\bigg)^{t - t} \Big( P_{i, t} A_{i, t} \theta_i L_{i, t}^{\theta_i -1} K_{i, t}^{1-\theta_i} -  W_{i, t}  \Big) &= 0 \\
	\implies \theta_i P_{i, t} A_{i, t}  L_{i, t}^{\theta_i -1} K_{i, t}^{1-\theta_i} -  W_{i, t}  &= 0
\end{split}
\end{equation*}
The FOC w.r.t. $K_{i, t+1}$ is: 
\begin{equation*}
\begin{split}
	\bigg(\frac{1}{1+R}\bigg)^{t - t} (-1) + \bigg(\frac{1}{1+R}\bigg)^{t + 1 - t} \Big(  P_{i, t} A_{i, t} L_{i, t}^{\theta_i} (1-\theta_i) K_{i, t}^{-\theta_i} - (-1) \Big) &= 0 \\
	\implies -1 + \frac{1}{1+R} \Big( (1-\theta_i)  P_{i, t} A_{i, t} L_{i, t}^{\theta_i}  K_{i, t}^{-\theta_i} + 1 \Big) &= 0 \\
\end{split}
\end{equation*}
\\
This yields optimality conditions for tradables and nontradables. For the the tradable sector, normalising $P_T = 1$, for capital
\begin{equation*}
\begin{split}
	-1 + \frac{1}{1+R} \Big( (1-\theta_T)  A_{T} L_{T}^{\theta_i}  K_{T}^{-\theta_i} + 1 \Big) &= 0 \\
	 (1-\theta_T)  A_{T} \bigg( \frac{K_{T}}{L_{T}}\bigg)^{-\theta_i}  + 1 &= 1+R \\
	\implies (1-\theta_T)  A_{T} \bigg( \frac{K_{T}}{L_{T}}\bigg)^{-\theta_i}  &= \bar{R} \\
\end{split}
\end{equation*}
\\
and for labour: 
\begin{equation*}
\begin{split}
	 \theta_T A_{T}  L_{T}^{\theta_T -1} K_{T}^{1-\theta_T} -  L_{T}  &= 0 \\
	\implies \theta_T A_{T} \bigg( \frac{K_{T}}{L_{T}}\bigg)^{1-\theta_T}  &= W \\
\end{split}
\end{equation*}
\\
We repeat the exercise for the non-tradable sector (with $P_N$ the relative price of nontraded goods): 
\begin{equation*}
\begin{split}
	(1-\theta_N) P_N  A_{N} \bigg( \frac{K_{N}}{L_{N}} \bigg)^{-\theta_N}  &= \bar{R} \\
	\theta_N P_N A_{N} \bigg( \frac{K_{N}}{L_{N}}\bigg)^{1-\theta_N}  &= W
\end{split}
\end{equation*}
\\
We have four unknowns: $W$, $P_N$, $\frac{K_N}{L_N}$ and $\frac{K_T}{L_T}$.\\
\\
We log-differentiate every equation: we take logs then totally differentiate every variable with respect to itself: $\hat{X} = \text{d } \ln{X}$. 
\begin{equation*}
\begin{split}
	\ln{ \Bigg( (1-\theta_T)  A_{T} \bigg( \frac{K_{T}}{L_{T}}\bigg)^{-\theta_i} \Bigg)}  &= \ln{\bar{R}} \\
	\ln{  (1-\theta_T) } + \ln{ A_{T}}  + \ln{ \bigg( \frac{K_{T}}{L_{T}}\bigg)^{-\theta_i} }   &= \ln{\bar{R}} \\
	\ln{  (1-\theta_T) } + \ln{ A_{T}}  - \theta_T \ln{  \frac{K_{T}}{L_{T}} }  &= \ln{\bar{R}} \\
	\ln{  (1-\theta_T) } + \ln{ A_{T}}  - \theta_T \Big( \ln{ K_{T}} - \ln{ L_{T}} \Big)  &= \ln{\bar{R}} \\
\end{split}
\end{equation*}
Differentiating each variable with respect to itself (which eliminates the constants such as $\bar{R}$ and $\ln{  (1-\theta_T) }$): 
\begin{equation*}
\begin{split}
	\text{d} \ln{ A_{T}}  - \theta_T \Big( \text{d}\ln{ K_{T}} - \text{d}\ln{ L_{T}} \Big)  &= 0 \\
	\implies \hat{A}_{T}  - \theta_T \Big( \hat{K}_{T} - \hat{L}_{T} \Big)  &= 0 \\
\end{split}
\end{equation*}
\\
We repeat this process for the other equations. For the labour equation of the tradable sector:
\begin{equation*}
\begin{split}
	 \theta_T A_{T} \bigg( \frac{K_{T}}{L_{T}}\bigg)^{1-\theta_T}  &= W \\
	\ln{ \Bigg( \theta_T A_{T} \bigg( \frac{K_{T}}{L_{T}}\bigg)^{1-\theta_T} \Bigg) } &= \ln{W} \\
	\ln{  \theta_T } + \ln{ A_{T} } + \ln{\bigg( \frac{K_{T}}{L_{T}}\bigg)^{1-\theta_T} } &= \ln{W} \\
	\ln{  \theta_T } + \ln{ A_{T} } + (1-\theta_T)\ln{\frac{K_{T}}{L_{T}}} &= \ln{W} \\
	\ln{  \theta_T } + \ln{ A_{T} } + (1-\theta_T)\big( \ln{K_{T}} - \ln{L_{T}} \big) &= \ln{W} \\
	\text{d} \ln{ A_{T} } + (1-\theta_T)\big( \text{d} \ln{K_{T}} - \text{d} \ln{L_{T}} \big) &= \text{d} \ln{W} \\
	\implies \hat{A}_{T} + (1-\theta_T)\big( \hat{K}_{T} - \hat{L}_{T} \big) &= \hat{W} \\
\end{split}
\end{equation*}
For the equations of the nontradable sector, the only difference is the inclusion of the variable $P_N$, hence we add $\hat{P}_{N}$: 
\begin{equation*}
\begin{split}
	\hat{P}_{N} + \hat{A}_{N}  - \theta_N \Big( \hat{K}_{N} - \hat{L}_{N} \Big)  &= 0 \\
	\hat{P}_{N} + \hat{A}_{N} + (1-\theta_N)\big( \hat{K}_{N} - \hat{L}_{N} \big) &= \hat{W} \\
\end{split}
\end{equation*}
\\
We now solve for our four unknowns: $\hat{K}_{T} - \hat{L}_{T}$, $\hat{K}_{N} - \hat{L}_{N}$, $\hat{W}$ and $\hat{P}_{N}$. \\
Subtracting the first two equations: 
\begin{equation*}
\begin{split}
	\hat{A}_{T} + (1-\theta_T)\big( \hat{K}_{T} - \hat{L}_{T} \big) - \bigg( \hat{A}_{T}  - \theta_i \Big( \hat{K}_{T} - \hat{L}_{T} \Big)  \bigg) &= \hat{W} - 0 \\
	(1-\theta_T)\big( \hat{K}_{T} - \hat{L}_{T} \big) + \theta_i \Big( \hat{K}_{T} - \hat{L}_{T} \Big) &= \hat{W} \\
	 \hat{K}_{T} - \hat{L}_{T} &= \hat{W} \\
\end{split}
\end{equation*}
Plugging this into the second one: 
\begin{equation*}
\begin{split}
	\hat{A}_{T} + (1-\theta_T)\big( \hat{K}_{T} - \hat{L}_{T} \big) &= \hat{W} \\
	\hat{A}_{T} + (1-\theta_T) \hat{W} &= \hat{W} \\
	\hat{A}_{T} &= \hat{W} -  (1-\theta_T)\hat{W}  \\
	\hat{A}_{T} &= \hat{W} - \hat{W} +  \theta_T\hat{W} \\
	\hat{A}_{T} &=  \theta_T \hat{W} \\
	\implies \hat{W} &= \frac{\hat{A}_{T} }{\theta_T}
\end{split}
\end{equation*}
\\
\\
Subtracting the third from the fourth equation: 
\begin{equation*}
\begin{split}
	\hat{P}_{N} + \hat{A}_{T} + (1-\theta_T)\big( \hat{K}_{T} - \hat{L}_{T} \big) - \bigg(\hat{P}_{N} + \hat{A}_{N}  - \theta_i \Big( \hat{K}_{N} - \hat{L}_{N} \Big) \bigg) &= \hat{W} - 0\\
	\implies  \hat{K}_{N} - \hat{L}_{N} &= \hat{W} 
\end{split}
\end{equation*}
And hence: 
\begin{equation*}
\begin{split}
	\hat{K}_{N} - \hat{L}_{N} = \frac{\hat{A}_{T} }{\theta_T}
\end{split}
\end{equation*}
Plugging this into the fourth equation: 
\begin{equation*}
\begin{split}
	\hat{P}_{N} + \hat{A}_{N} + (1-\theta_N)\big( \hat{K}_{N} - \hat{L}_{N} \big) &= \hat{W} \\
	\hat{P}_{N} + \hat{A}_{N} + (1-\theta_N)\frac{\hat{A}_{T} }{\theta_T}  &= \frac{\hat{A}_{T} }{\theta_T} \\
\end{split}
\end{equation*}
As a result, 
\begin{equation*}
\begin{split}
	\hat{P}_{N} &= \frac{\hat{A}_{T} }{\theta_T} - (1-\theta_N)\frac{\hat{A}_{T} }{\theta_T}  - \hat{A}_{N}\\
	\hat{P}_{N} &= \frac{\hat{A}_{T} }{\theta_T} - \frac{\hat{A}_{T} }{\theta_T} + \theta_N\frac{\hat{A}_{T} }{\theta_T}  - \hat{A}_{N}\\
	\implies \hat{P}_{N} &= \frac{ \theta_N }{\theta_T} \hat{A}_{T}  - \hat{A}_{N}\\
\end{split}
\end{equation*}
\\
We log differentiate the aggregate price index: 
\begin{equation*}
\begin{split}
	P &= P_T^\gamma P_N^{1-\gamma}  = P_N^{1-\gamma} \\
	\implies \ln{P} &= \ln{P_N^{1-\gamma}} \\
		&= (1-\gamma) \ln{P_N} \\
	\text{d} \ln{P}&= (1-\gamma) \text{d} \ln{P_N} \\
	\implies \hat{P} &= (1-\gamma) \hat{P}_{N} \\ 
\end{split}
\end{equation*}
We substitute our expression for $\hat{P}_{N}$ : 
\begin{equation*}
\begin{split}
	\hat{P} &= (1-\gamma) \bigg(  \frac{ \theta_N }{\theta_T} \hat{A}_{T} - \hat{A}_{N}  \bigg)\\
\end{split}
\end{equation*}
Since $\theta_N \geq \theta_T$ and $\hat{A}_{T} >  \hat{A}_{N}$, we know that $\hat{P}>0$. \\
\\
\\
We consider two countries Home and Foreign (denoted by *) that are identical, except for productivity growth $\hat{A}_i$. \\
\\
We log-differentiate the real exchange rate $\xi = \frac{P}{S P^*}$. We use the law of one price for tradables $P_T = S P_T^* = 0$ and the small economy identity: $P_T^* = \bar{P}_T^*$: $P_t = S \bar{P}_T^* = 1 \implies S = \frac{1}{\bar{P}_T^*}$. Hence:
\begin{equation*}
\begin{split}
	\xi &= \frac{P}{\frac{1}{\bar{P}_T^*} P^*} = \frac{P}{\frac{P^*}{\bar{P}_T^*} } \\
	\ln{\xi} &= \ln{\Bigg( \frac{P}{\frac{P^*}{\bar{P}_T^*} } \Bigg)} = \ln{P} - \ln{P^*} \\
	\text{d} \ln{\xi} &= \text{d} \ln{P} - \text{d} \ln{P^*} \\
	\hat{\xi} &= \hat{P} - \hat{P}^* \\
		&= (1-\gamma) \bigg(  \frac{ \theta_N }{\theta_T} \hat{A}_{T} - \hat{A}_{N}  \bigg) - (1-\gamma) \bigg(  \frac{ \theta_N }{\theta_T} \hat{A}_{T}^* - \hat{A}_{N}^*  \bigg) \\
		&= (1-\gamma) \bigg(  \frac{ \theta_N }{\theta_T} \hat{A}_{T} - \hat{A}_{N} - \frac{ \theta_N }{\theta_T} \hat{A}_{T}^* + \hat{A}_{N}^*   \bigg) \\
	\implies \hat{\xi} &= (1-\gamma) \bigg(  \frac{ \theta_N }{\theta_T} \Big( \hat{A}_{T} -  \hat{A}_{T}^* \Big) - \Big( \hat{A}_{N} - \hat{A}_{N}^* \Big)   \bigg) \\
\end{split}
\end{equation*} 
\\
So, a productivity growth advantage in tradables, $\hat{A}_{T} -  \hat{A}_{T}^* > \hat{A}_{N} - \hat{A}_{N}^*$ leads to a real appreciation $\hat{\xi}>0$. This is the Harrod-Balassa-Samuelson effect: countries with higher productivity in tradables compared to nontradables tend to have higher aggregate price levels. \\
\\
Higher productivity in tradeables\\ $\rightarrow$ a higher wage in tradeables\\ $\rightarrow$ higher wage in non-tradeables (through wage arbitrage)\\ $\rightarrow$ higher labour costs in non-tradeables\\ $\rightarrow$ higher prices for non-tradeables goods\\ $\rightarrow$ higher aggregate price level.

%\section{Dornbusch Overshooting Model}

\end{document}
